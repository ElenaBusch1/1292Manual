\chapter{General Instructions}

\section{Purpose of the Laboratory}

The laboratory experiments described in this manual are an important part of your physics course.  Most of the experiments are designed to illustrate important concepts described in the lectures.  Whenever possible, the material will have been discussed in lecture before you come to the laboratory.  \myskip

The sections headed \underline{Applications} and \underline{Lab Preparation Examples}, which are included in some of the manual sections, are \emph{not} required reading unless your laboratory instructor specifically assigns some part.  The Applications are intended to be motivational and so should indicate the importance of the laboratory material in medical and other applications.  The Lab Preparation Examples are designed to help you prepare for the lab; you will not be required to answer all these questions (though you should be able to answer any of them by the end of the lab).  The individual laboratory instructors may require you to prepare answers to a subset of these problems.\myskip

\section{Preparation for the Laboratory}

In order to keep the total time spent on laboratory work within reasonable bounds, the write-up for each experiment will be completed at the end of the lab and handed in \emph{before the end of each laboratory period}.  Therefore, it is \underline{imperative} that you spend sufficient time preparing for the experiment \emph{before} coming to laboratory. You should take advantage of the opportunity that the experiments are set up in the \underline{Lab Library} (Room 506) and that TAs there are willing to discuss the procedure with you.   \myskip

At each laboratory session, the instructor will take a few minutes at the beginning to go over the experiment to describe the equipment to be used and to outline the important issues. This does not substitute for careful preparation beforehand!  You are expected to have studied the manual and appropriate references at home so that you are prepared when you arrive to perform the experiment.  The instructor will be available primarily to answer questions, aid you in the use of the equipment, discuss the physics behind the experiment, and guide you in completing your analysis and write-up.  Your instructor will describe his/her policy regarding expectations during the first lab meeting.\myskip

Some experiments and write-ups may be completed in less than the three-hour laboratory period, but under no circumstances will you be permitted to stay in the lab after the end of the period or to take your report home to complete it.  If it appears that you will be unable to complete all parts of the experiment, the instructor will arrange with you to limit the experimental work so that you have enough time to write the report during the lab period.\myskip

\textbf{Note}: Laboratory equipment must be handled with care and each laboratory bench must be returned to a neat and orderly state before you leave the laboratory.  In particular, you must \underline{turn off} all sources of electricity, water, and gas.

\section{Bring to Each Laboratory Session}

\begin{itemize}
    \item A pocket calculator (with basic arithmetic and trigonometric operations).

    \item A pad of $8.5 \times 11$ inch graph paper and a sharp pencil.  (You will write your reports on this paper, including your graphs.  Covers and staplers will be provided in the laboratory.)

    \item A ruler (at least $10\,\mathrm{cm}$ long).
\end{itemize}

\section{Graph Plotting}

Frequently, a graph is the clearest way to represent the relationship between the quantities of interest.  There are a number of conventions, which we include below.

\begin{itemize}
    \item A graph indicates a relation between two quantities, $x$ and $y$, when other variables or parameters have fixed values.  Before plotting points on a graph, it may be useful to arrange the corresponding values of $x$ and $y$ in a table.

    \item Choose a convenient scale for each axis so that the plotted points will occupy a \underline{substantial} part of the graph paper, but do \underline{not} choose a scale which is difficult to plot and read, such as 3 or 3/4 units to a square.  Graphs should usually be at least half a page in size.

    \item Label each axis to identify the variable being plotted and the units being used.  Mark prominent divisions on each axis with appropriate numbers.

    \item Identify plotted \emph{points} with appropriate symbols, such as crosses, and when necessary draw vertical or horizontal \emph{bars} through the points to indicate the range of uncertainty involved in these points.

    \item Often there will be a theory concerning the relationship of the two plotted variables.  A linear relationship can be demonstrated if the data points fall along a single straight line.  There are mathematical techniques for determining which straight line best fits the data, but for the purposes of this lab it will be sufficient if you simply make a rough estimate visually.  \emph{The straight line should be drawn as near the mean of the all various points as is optimal}.  That is, the line need not precisely pass through the first and last points.  Instead, each point should be considered as accurate as any other point (unless there are experimental reasons why some points are less accurate than others).  The line should be drawn with about as many points above it as below it, and with the `aboves' and `belows' distributed at random along the line.  (For example, not all points should be above the line at one end and below at the other end).
\end{itemize}

\section{Questions or Complaints}

If you have a difficulty, you should attempt to work it through with your laboratory instructor.  If you cannot resolve it, you may discuss such issues with:

\begin{itemize}
    \item one of the laboratory Preceptors in Pupin Room 729;

    \item the Undergraduate Assistant in the Departmental Office -- Pupin Room 704;

    \item the instructor in the lecture course, or the Director of Undergraduate Studies;

    \item your undergraduate advisor.
\end{itemize}

As a general rule, it is a good idea to work downward through this list, though some issues may be more appropriate for one person than another.


