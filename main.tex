\documentclass[letterpaper, 12pt]{book}

\usepackage[pdftex]{graphicx}
\usepackage{epstopdf}
\DeclareGraphicsRule{*}{mps}{*}{}

\usepackage[titletoc]{appendix}
\usepackage{amsmath, amsthm, amssymb}
\usepackage{listings}
\usepackage{float}
\usepackage{enumerate}
\usepackage{hyperref}
\usepackage{fancyheadings}
\usepackage{titlesec}
\usepackage{multicol}
\usepackage{subcaption}
\usepackage{wrapfig}
\usepackage{tocloft}
\usepackage{tikz}
\usepackage{engord}
\usepackage{comment}
\usetikzlibrary{positioning}
\usetikzlibrary{decorations.pathmorphing}
\usetikzlibrary{arrows}
\usetikzlibrary{decorations.markings}
\usetikzlibrary{patterns}
\usetikzlibrary{shapes}
\usetikzlibrary{shapes.geometric}
%\usepackage{fullpage}
\usepackage[left=1in, top=1.2in, right=1in, bottom=1.2in, bindingoffset=0.4in]{geometry}


% Different font in captions
\newcommand{\captionfonts}{\small}

\makeatletter  % Allow the use of @ in command names
\long\def\@makecaption#1#2{%
  \vskip\abovecaptionskip
  \sbox\@tempboxa{{\captionfonts #1: #2}}%
  \ifdim \wd\@tempboxa >\hsize
    {\captionfonts #1: #2\par}
  \else
    \hbox to\hsize{\hfil\box\@tempboxa\hfil}%
  \fi
  \vskip\belowcaptionskip}
\makeatother   % Cancel the effect of \makeatletter

\linespread{1.05}

\renewcommand*\thesection{\arabic{section}}
\renewcommand*\thefigure{\arabic{figure}}
\renewcommand*\theequation{\arabic{equation}}

\setlength{\cftchapnumwidth}{1.2cm}
\renewcommand{\cftchappresnum}{2-}

\setcounter{secnumdepth}{3}
\setcounter{tocdepth}{0}
\setcounter{chapter}{-1}

\pagestyle{fancy}
\renewcommand{\chaptermark}[1]{\markboth{#1}{}}
\renewcommand{\sectionmark}[1]{\markright{\thesection.#1}{}}

\newcommand{\myskip}{\vspace{0.5\baselineskip}}

\lhead[\rightmark]{Experiment 2-\thechapter}
\cfoot{\thepage}
\rhead[Experiment 2-\thechapter]{\leftmark}

%\fancypagestyle{plain}{
    %\fancyhf{}
    %\renewcommand{\headrulewidth}{0pt}
    %\renewcommand{\footrulewidth}{0pt}
%}


\begin{document}

\pagestyle{empty}
\begin{titlepage}
\begin{center}
\textsc{\Huge\bf Experiments in Physics}
\\[5cm]
\textsc{\huge Physics 1292}
\\[0.3cm]
\textsc{\huge General Physics II Lab}
\\[4cm]
\textsc{\large Columbia University}
\\[0.5cm]
\begin{figure}[h]
  \centering
  \includegraphics[height=4cm]{./pic/Columbia-Logo.png}
\end{figure}
\textsc{Department of Physics}
\\[1cm]
\textsc{Spring 2014}
\end{center}
\end{titlepage}

\newpage
\phantom{aaa}
\clearpage
\tableofcontents
\addtocontents{toc}{\protect\thispagestyle{empty}}
\cleardoublepage

\pagestyle{fancy}

\titlespacing*{\chapter}{0pt}{-50pt}{20pt}
\titleformat{\chapter}[display]
    {\bfseries\huge\filcenter}
    {\underline{Introduction 2-\thechapter}}
    {0pt}
    {\huge}

\setcounter{page}{1}
\chapter{General Instructions}

\section{Purpose of the Laboratory}

The laboratory experiments described in this manual are an important part of your physics course.  Most of the experiments are designed to illustrate important concepts described in the lectures.  Whenever possible, the material will have been discussed in lecture before you come to the laboratory.  But some of the material, like the first experiment on measurement and errors, is not discussed at length in the lecture.\myskip

The sections headed \underline{Applications} and \underline{Lab Preparation Examples}, which are included in some of the manual sections, are \emph{not} required reading unless your laboratory instructor specifically assigns some part.  The Applications are intended to be motivational and so should indicate the importance of the laboratory material in medical and other applications.  The Lab Preparation Examples are designed to help you prepare for the lab; you will not be required to answer all these questions (though you should be able to answer any of them by the end of the lab).  The individual laboratory instructors may require you to prepare answers to a subset of these problems.\myskip

\section{Preparation for the Laboratory}

In order to keep the total time spent on laboratory work within reasonable bounds, the write-up for each experiment will be completed at the end of the lab and handed in \emph{before the end of each laboratory period}.  Therefore, it is \underline{imperative} that you spend sufficient time preparing for the experiment \emph{before} coming to laboratory. You should take advantage of the opportunity that the experiments are set up in the \underline{Lab Library (Room 506)} and that TAs there are willing to discuss the procedure with you.\myskip

At each laboratory session, the instructor will take a few minutes at the beginning to go over the experiment and describe the equipment to be used and to outline the important issues. This does not substitute for careful preparation beforehand!  You are expected to have studied the manual and appropriate references at home so that you are prepared when you arrive to perform the experiment.  The instructor will be available primarily to answer questions, aid you in the use of the equipment, discuss the physics behind the experiment, and guide you in completing your analysis and write-up.  Your instructor will describe his/her policy regarding expectations during the first lab meeting.\myskip

Some experiments and write-ups may be completed in less than the three-hour laboratory period, but under no circumstances will you be permitted to stay in the lab after the end of the period or to take your report home to complete it.  If it appears that you will be unable to complete all parts of the experiment, the instructor will arrange with you to limit the experimental work so that you have enough time to write the report during the lab period.\myskip

\textbf{Note}: Laboratory equipment must be handled with care and each laboratory bench must be returned to a neat and orderly state before you leave the laboratory.  In particular, you must turn off all sources of electricity, water, and gas.

\section{Bring to Each Laboratory Session}

\begin{itemize}
    \item A pocket calculator (with basic arithmetic and trigonometric operations).

    \item A pad of $8.5 \times 11$ inch graph paper and a sharp pencil.  (You will write your reports on this paper, including your graphs.  Covers and staplers will be provided in the laboratory.)

    \item (optional) A ruler (at least $10\,\mathrm{cm}$ long).
    \item (optional) A personal laptop with Microsoft Excel for data analysis.
\end{itemize}

\section{Graph Plotting}

Frequently, a graph is the clearest way to represent the relationship between the quantities of interest.  There are a number of conventions, which we include below.

\begin{itemize}
    \item A graph indicates a relation between two quantities, $x$ and $y$, when other variables or parameters have fixed values.  Before plotting points on a graph, it may be useful to arrange the corresponding values of $x$ and $y$ in a table.

    \item Choose a convenient scale for each axis so that the plotted points will occupy a \underline{substantial} part of the graph paper, but do \underline{not} choose a scale which is difficult to plot and read, such as 3 or 3/4 units to a square.  Graphs should usually be at least half a page in size.

    \item Label each axis to identify the variable being plotted and the units being used.  Mark prominent divisions on each axis with appropriate numbers.

    \item Identify plotted \emph{points} with appropriate symbols, such as crosses, and when necessary draw vertical or horizontal \emph{error bars} through the points to indicate the range of uncertainty involved in these points.

    \item Often there will be a theory concerning the relationship between the two plotted variables.  A linear relationship can be demonstrated if the data points fall along a single straight line.  There are mathematical techniques for determining which straight line best fits the data, but for the purposes of this lab, we will be using Microsoft Excel's built-in fitting methods.
\end{itemize}

\section{Error Analysis}

All measurements, however carefully made, give a range of possible values referred to as an uncertainty or error. Since all of science depends on measurements, it is important to understand uncertainties and where they come from. Error analysis is the set of techniques for dealing with them.\myskip

In science, the word ``error'' does not take the usual meaning of ``mistake''. Instead, we will use it interchangeably with ``uncertainty'' when talking about the result of a measurement. There are many aspects to error analysis and it will feature in some form in every lab throughout this course.

\subsection{Inevitability of Experimental Error}

In the first experiment of the semester, you will measure the length of a pendulum. Without a ruler, you might compare it to your own height and (after converting to meters) make an estimate of $1.5\,\mathrm{m}$. Of course, this is only approximate. To quantify this, you might say that you are sure it is not less than $1.3\,\mathrm{m}$ and not more than $1.7\,\mathrm{m}$. With a ruler, you measure $1.62\,\mathrm{m}$. This is a much better estimate, but there is still uncertainty. You couldn't possibly say that the pendulum isn't $1.62001\,\mathrm{m}$ long. If you became obsessed with finding the exact length of the pendulum you could buy a fancy device using a laser, but even this will have an error associated with the wavelength of light.\myskip

Also, at this point you would come up against another problem. You would find that the string is slightly stretched when the weight is on it and the length even depends on the temperature or moisture in the room. So which length do you use? This is a problem of definition. During lab you might find another example. You might ask whether to measure from the bottom, top or middle of the weight. Sometimes one of the choices is preferable for some reason (in this case the middle because it is the center of mass). However, in general it is more important to be clear about what you mean by ``the length of the pendulum'' and consistent when taking more than one measurement. Note that the different lengths that you measure from the top, bottom or middle of the weight do not contribute to the error. \emph{Error} refers to the range of values given by measurements of exactly the same quantity.

\subsection{Importance of Errors}

In daily life, we usually deal with errors intuitively. If someone says ``I'll meet you at 9:00'', there is an understanding of what range of times is OK. However, if you want to know how long it takes to get to JFK airport by train you might need to think about the range of possible values. You might say ``It'll probably take an hour and a half, but I'll allow two hours''. Usually it will take within about 10 minutes of this most probable time. Sometimes it will take a little less than 1hr20, sometimes a little more than 1hr40, but by allowing the most probable time plus three times this uncertainty of 10 minutes you are almost certain to make it. In more technical applications, for example air traffic control, more careful consideration of such uncertainties is essential.\myskip

In science, almost every time that a new theory overthrows an old one, a discussion or debate about relevant errors takes place. In this course, we will definitely not be able to overthrow established theories. Instead, we will verify them with the best accuracy allowed by our equipment. The first experiment involves measuring the gravitational acceleration g. While this fundamental parameter has clearly been measured with much greater accuracy elsewhere, the goal is to make the most accurate possible verification using very simple apparatus which can be a genuinely interesting exercise.\myskip

There are several techniques that we will use to deal with errors throughout the course. All of them are well explained, with more formal justifications, in ``\emph{An Introduction to Error Analysis}'' by John Taylor, available in the Science and Engineering Library in the Northwest Corner Building.

\subsection{Questions or Complaints}

If you have a difficulty, you should attempt to work it through with your laboratory instructor.  If you cannot resolve it, you may discuss such issues with:

\begin{itemize}
    \item One of the laboratory Preceptors in Pupin Room 729;

    \item The Undergraduate Assistant in the Departmental Office -- Pupin Room 704;

    \item The instructor in the lecture course, or the Director of Undergraduate Studies;

    \item Your undergraduate advisor.
\end{itemize}

As a general rule, it is a good idea to work downward through this list, though some issues may be more appropriate for one person than another.

\section{Lab Report Checklist}

This checklist provides a listing of all the relevant tasks you need to complete each lab. Periodically one or two items on this list may not apply, but for nearly all labs, this provides a cohesive list of what each section of your report should include (please note this will not guarantee you a 100\% but all of these elements are required). Whilst the list may appear a little daunting at first, remember, it is \textbf{strongly encouraged} that you write your Objective and Methods section (and perform any Prelab questions) \textbf{before} your lab session. This will ensure that you understand the experiment more fully, and that you are able to complete it on time.
\newline
\newline
\textbf{Prelab:}
\begin{itemize}
    \item[$\square$] I have completed and included the prelab questions
    \item[$\square$] I have completed and included any necessary derivations for this lab
\end{itemize}    
\textbf{Objective:}
\begin{itemize}
    \item[$\square$] \textbf{In my own words}, I have stated the scientific concepts and key parameters tested in this lab
\end{itemize}
\textbf{Methods:}
\begin{itemize}
    \item[$\square$] \textbf{In my own words}, I have explained all the \textbf{key steps} and parameters for any given person to be able to reproduce the experiment
    \item[$\square$] I have listed, labelled, and provided context for any equations to be used
\end{itemize}  
\textbf{Data:}
\begin{itemize}
    \item[$\square$] My raw data is neatly organized, separated from calculated results
    \item[$\square$] Algebra is cohesive and grouped
    \item[$\square$] Data tables and graphs have:
	\begin{itemize}
	    \item[$\square$] Titles
	    \item[$\square$] Axes
	    \item[$\square$] Units
	\end{itemize}
    \item[$\square$] Calculations:
        \begin{itemize}
	    \item[$\square$] Have work clearly shown \textbf{with units}
	    \item[$\square$] Have references to any equations used in Methods
	    \item[$\square$] Are computed properly
	\end{itemize}
    \item[$\square$] Any linear regression calculations are shown clearly and values are labelled
    \item[$\square$] I have answered all questions asked in the manual during the experiment
\end{itemize}  
\textbf{Error:}
\begin{itemize}
    \item[$\square$] I have made \textbf{reasonable} estimates on uncertainties for \textbf{each measurement}
    \item[$\square$] Error Calculations:
        \begin{itemize}
	    \item[$\square$] I have propagated the errors throughout each calculation
	    \item[$\square$] I have clearly shown all steps
	\end{itemize}
    \item[$\square$] I have identified \textbf{reasonable} sources of error, and I have \textbf{shown what kind of effect these sources would have on my final results}
\end{itemize}  
\textbf{Conclusion:}
\begin{itemize}
    \item[$\square$] I have restated the underlying scientific concepts and key parameters noted in Objectives
    \item[$\square$] I have included all key measurements and shown how and why our results are consistent/inconsistent with the scientific concepts (reference any relevant equations)
    \item[$\square$] I have included any relevant error analysis (even if not specifically asked in the experiment)
\end{itemize}  


\titleformat{\chapter}[display]
    {\bfseries\huge\filcenter}
    {\underline{Experiment 2-\thechapter}}
    {0pt}
    {\huge}

%\chapter{Electric Fields}

\section{Introduction}

One of the fundamental concepts used to describe electric phenomena is that of the electric field. We explore the field concept experimentally by measuring lines of equal electric potential on carbon paper and then constructing the electric field lines that connect them.

\section{Theory}

\subsection{Fields}

A \underline{field} is a function in which a numerical value associated with a physical quantity is assigned to every point in space. For example, each point in the United States may be assigned a temperature (that we measure with a thermometer at that position), as shown in figure \ref{fig:temp_field}. Thus, temperature as a function of location is called the ``temperature field''. The temperature field is an example of a scalar field, since the value assigned to each location is a scalar.

\begin{figure}[h]
    \begin{center}
        \includegraphics[width=0.5\textwidth]{./Exp1/pic/image1.png}
    \end{center}
    \caption{Temperature of Different Places in the United States}
    \label{fig:temp_field}
\end{figure}

The electric field (as well as the magnetic field) is a vector field, because a vector (rather than a scalar) is associated with each position. The direction and magnitude of an electric field at any point in space tells you the direction and magnitude of electric force that a  unit test charge would experience at this position. \myskip

A line that is formed by connecting electric field vectors and follows the direction of the field is called a \underline{field line}. A field is considered uniform if the field lines are parallel. The potential associated with such a field increases linearly with the distance traveled along the field lines.

\subsection{Equipotential Contours for Gravity}

A topographical map is a map with lines that indicate contours of the same height. On such a map, a mountain may look like figure \ref{fig:topograph}

\begin{figure}[h]
    \begin{center}
        \includegraphics[width=0.4\textwidth]{./Exp1/pic/image2.png}
    \end{center}
    \caption{A Topographical Map}
    \label{fig:topograph}
\end{figure}

These contours of the same height $(h)$ are equi-height lines or, if you think in terms of potential energy $(U = mgh)$, they are equipotential lines. That is, every point on the contour has the same value of potential energy. So as you move along a contour, you do not change height, and therefore you neither gain nor lose potential energy. Only as you step up or down the hill do you change the energy. \myskip

There are a few general properties of equipotential contours:
\begin{itemize}
    \item They never intersect. (A single point cannot be at two different heights at the same time, and
therefore it cannot be on two different contours.)
    \item They close on themselves. (A line corresponding to constant height cannot just end.)
    \item They are smooth curves, as long as the topography has no sharp discontinuities (like a cliff).
\end{itemize}

\subsection{Analogy between Electric and Gravitational Potentials}

Electric potential is analogous to gravitational potential energy. Of course, gravitational potential energy arises from any object (with mass), whereas the electric potential arises only from charged objects. \myskip

As in mechanics, the absolute value of potential is not important. The differences in potential are the quantities that have physical meaning. In a real problem, it is usually best to choose a reference point, define it as having zero potential, and refer to potentials at all other points relative to the that point. Similarly, topographic height is usually reported relative to the reference at sea level.

\subsection{Field Lines}

Let's exploit our analogy of mountains and valleys a bit further for electric fields. Electric field lines indicate the electric force on a test charge (magnitude and direction). Gravitational field lines tell us about the gravitational force on a test mass (magnitude and direction) -- the direction it tends to roll down the hill and the acceleration it experiences during its journey. So if we carefully track a marble as we roll it down the hill in small increments, we can obtain a ``field-line''.  \myskip

What is the characteristic of a field line? Think about placing a marble on an inclined hillside.  Which way will it roll if you release it? It will always roll in the direction of steepest descent, and this direction is always perpendicular to the contours that indicate equal height (which we have been referring to as equipotential lines). \myskip

So it is for the electric case: if we know the equipotential lines, we can draw the field lines such that each one is perpendicular to each equipotential line. Figure \ref{fig:equi_field} and \ref{fig:field_dipole} both show equipotential lines (dashed lines) and the corresponding field lines (solid lines).

\begin{figure}[h]
    \begin{center}
        \includegraphics[width=0.5\textwidth]{./Exp1/pic/image3.png}
    \end{center}
    \caption{Equipotential Lines and Field Lines of a Single Charge}
    \label{fig:equi_field}
\end{figure}

Keep in mind that when equipotential lines are closer together, the potential is changing faster! This means we have steeper changes in height for gravitational field lines for example.

\begin{figure}[h]
    \begin{center}
        \includegraphics[width=0.6\textwidth]{./Exp1/pic/image4.png}
    \end{center}
    \caption{Equipotential Lines and Field Lines of a Pair of Opposite Charges}
    \label{fig:field_dipole}
\end{figure}

There are also a few rules for field lines:
\begin{itemize}
    \item Field lines always begin and end on charges. (They often terminate on material surfaces, but that is because there are charges on those surfaces.)
    \item They never intersect each other (except at electric charges, where they also terminate).
    \item They always intersect equipotential lines perpendicularly!
    \item They are usually smoothly continuous (except if they terminate).
\end{itemize}

\subsection{Metal Surfaces}

It turns out that for electrical phenomena, metals are equipotentials. So, by using metals, we can impose some unusually shaped equipotential lines and determine the corresponding field lines. \myskip

Metals are always equipotentials because:
\begin{enumerate}
    \item they are excellent conductors of electric charge, and
    \item they have an abundant supply of freely moving negative charges (electrons).
\end{enumerate}

Based on these two properties, we can understand that, if a field line were to penetrate a metal surface, these electrons would feel electric forces to move them through the material in the direction opposite to that of the electric field (since they are negative) until they would be forced to stop upon encountering the metal surface on the other side. Therefore, the free electrons ultimately distribute themselves along the surface of the metal so that there is a net negative charge on one side of the conductor and a net positive charge on the other (from where the electrons have fled). This realignment of the free electrons creates a field that precisely cancels the imposed field resulting in no net field, and the \underline{remaining} free charges inside the conductor then feel no force and don't move! Hence, field lines imposed from outside always end on metal surfaces (which they intersect perpendicularly), and the surface (with the entire interior) is an equipotential.

\subsection{Electric Shielding Theorem}

As just discussed, whenever a piece of metal is placed in an electric field, the entire metal will remain an equipotential. That is, every point in the metal will be at the same potential as every other point. No field lines penetrate through metals. \myskip

What happens if we take an enclosed container of metal of arbitrary shape, say a tin can, and put it into an electric field? Since no field gets through the metal, there must be no electric field inside the can. This means that every point inside the container, as well as all points on the container surface must be at the same potential. (If they were not, then there would be differences in the potential and therefore an electric field.) \myskip

Let's return to our analogy of gravity equipotentials in hills and valleys, with field lines in the direction a marble would roll down the hill. Consider how a frozen lake would look on such a contour map. The surface of the lake has the same gravitational potential energy at all points, therefore if you place a marble on the frozen surface of the lake, the marble will not roll anywhere. In other words, there is no component of the gravitational field along the surface of the lake to push the marble. Similarly, there is no electric field on the inside of a closed metal container to push the charges around, and the entire interior is therefore a constant equipotential surface.

\subsection{Remarks for Experts}

\begin{enumerate}
    \item  The gravitational analogy operates in only two dimensions: the horizontal coordinates describing the surface of the lake. When discussing electrical fields we should, in principle, take into account all three dimensions. The analogy of the lake surface for the electrical case is the entire volume (three dimensions) of the interior of the can. For this experiment, we ``cheat'' a little by looking at electric field within a two dimensional world of slightly conducting paper. But the field and potential arrangements are as one expects for electrostatic phenomena in a two dimensional system.
    \item To shield electric field, we actually don't need a metallic surface that is completely closed; a closed cage of wire mesh (Faraday cage) is sufficient. So a sedan will work like a Faraday cage, but a convertible will not since it is not closed at the top.
\end{enumerate}

\section{Experiments}

In this experiment, we use pieces of slightly conductive paper, electrodes, and metallic paint to determine different equipotential and field configurations. Before we actually start our measurements we will have to draw the shapes on the conductive paper using the metal paint. The metal paint will take a few minutes to dry.

\subsection{Preparation of the Conductive Paper}

Take four pieces of conductive paper -- three full-sized and one thin strip and paint on them with the metallic paint according to the following instructions. Each shape drawn by the metal paint is an equipotential curve. By connecting the pins to areas covered with paint, we can determine the field configurations between different equipotential shapes.\myskip

\begin{minipage}[h]{0.55\textwidth}
    \begin{enumerate}
        \item Leave the first piece of paper blank.
        \item On the second piece of paper paint two parallel plates. Use a piece of cardboard provided to get sharp edges with the paint. Don't use the rulers!
        \item On the third piece of paper paint a closed loop of any shape that does not enclose an electrode.
        \item Take the thin strip and paint a line on each short end of the strip.
    \end{enumerate}
\end{minipage}
\begin{minipage}[h]{0.45\textwidth}
    \begin{flushright}
        \vspace{0.5cm}
        \begin{tikzpicture}[scale=0.8]
            \draw (0,0) rectangle (-3,-2.5);
            \draw[line width=4pt] (-0.7,-0.5) -- (-0.7,-2);
            \draw[line width=4pt] (-2.3,-0.5) -- (-2.3,-2);
        \end{tikzpicture}
        \vspace{0.3cm}

        \begin{tikzpicture}[scale=0.8]
            \draw (0,0) rectangle (3,2.5);
            %\draw[line width=2pt] (0.7,1.25) to [bend left=70] (1.5,2.1) to[bend right=20] (2.3,1.25) to [bend left=30] (1.8,0.5) to [bend right=40] (1.2,1) to [bend right=70] (0.7,1.25);
            \draw[line width=2pt,rounded corners=3pt] (0.7,1.25) -- (1.5,2.1) -- (2.3,1.25) -- (1.8,0.5) -- (1.2,1) -- (0.4,0.6) -- cycle;
        \end{tikzpicture}
        \vspace{0.3cm}

        \begin{tikzpicture}[scale=0.8]
            \draw (0,0) rectangle (5,0.8);
            \draw[line width=4pt] (0,0) -- (0,0.8);
            \draw[line width=4pt] (5,0) -- (5,0.8);
        \end{tikzpicture}
    \end{flushright}
\end{minipage}

\myskip
\noindent Hints:
\begin{enumerate}
    \item You get the best results if you stay at least 1-2$\,\mathrm{cm}$ away from the edges of the paper. Also make sure that the pins are always in contact with the paper. (Otherwise the experiment does not work!) Be careful not to make the holes too big or the pins will no longer be in contact with the paper.
    \item Give the metal paint enough time to dry! First paint all the configurations you will use on the conductive paper. While they dry, get started on the two point-charge arrangement.
    \item Make sure you're not accidentally touching the paper when making measurements!
\end{enumerate}

\subsection{Setup of the Voltage Source}

\begin{itemize}
    \item ``OUTPUT A'' of the power supply will be used in this experiment to provide a constant potential difference of 10 volts.
    \item Turn the ``A VOLTAGE'', the ``A CURRENT'', the ``B VOLTAGE'', and the ``B CURRENT'' control knobs located on the right side of the front panel counterclockwise to the ``MIN'' position.
    \item Set the ``A/B OUTPUT'' switch located on the upper right of the front panel to the ``INDEPENDENT'' setting, and set the ``A/B METER'' switch located between the two meters on the front panel to the ``A'' setting.
    \item Turn the power on by using the button located on the left of the front panel.
    \item Slowly turn the ``A VOLTAGE'' output knob until the voltage reads 10 volts. After doing this, do not make any further changes on the power supply for the remainder of the experiment. (The current meter will stay at a zero reading.)
    \item Connect two cables to the ``A OUTPUT ($0\sim 20\,\mathrm{V}\  0.5\,\mathrm{A}$)'' of the power supply (red and black poles) and then connect these cables to two of the yellow-metallic pins.
\end{itemize}

NOTE: To use the digital multimeter as an appropriate voltmeter for this experiment: turn the multimeter power on, select ``DC'' (button in out position), select ``VOLTS'' (push button in), set range to ``2 VOLTS'', and connect your voltage-measuring cables to the ``$\mathrm{V}$--$\Omega$'' and the ``COM'' jacks.

\subsection{Equipotential and Field Lines}

\begin{itemize}
    \item Connect two cables to the output of the power supply (black and red poles) and two of the yellow-metallic pins as described in the last step of the previous part. These are your output pins.
    \item Take one additional cable and connect one end to the black pole of the multimeter and the other to a different metallic pin. This will be your reference pin. Take the pen-like pin and plug it into the red pole of the multimeter. This pin will be your measuring probe.
    \item Take an empty sheet and put the two output pins on it to act as point-like sources. Push the pins in only a little bit and not all the way through.
    \item Insert the reference pin at an arbitrary position and move the measuring probe until the multimeter shows a value of zero. This means that the two points are at the same potential. Mark this position with one of the red markers provided. Search for more equipotential points until you have enough to draw an equipotential contour connecting them.
    \item Move the reference pin to a new position and construct another equipotential line.
    \item Repeat for a total of about 5 equipotential lines.
    \item Given these equipotential lines, draw about 5 field lines using the yellow pen. Remember that the field lines and equipotential lines always intersect perpendicularly.
    \item Do your equipotential and field lines show the symmetries you would expect for this system? Just what symmetries does/should this system have?
    \item Obtain the equipotential and field lines for the parallel-plate configuration as well. (Stick the two pins from the function generator into the metal paint-covered part of the paper, after the paint has dried!).
    \item Are the equipotential lines as you expect them?
    \item What the major problems and uncertainties in this part?
\end{itemize}

\subsection{Electric Shielding Theorem}

Take the sheet with the conducting loop on it, and set up an electric field. You need the reference pin as in the previous part. Put it somewhere either within the loop or on the rim of it.\myskip

Put the reference pin anywhere within the closed surface or on its rim, and check that any point within the loop has the same potential as the reference pin. This demonstrates that all points within the closed loop have the same potential.
\begin{itemize}
    \item What do you conclude about the shielding theorem? Does it hold or not?
    \item What would happen with your readings if you had drawn the loop badly such that the loop was not perfectly closed?
\end{itemize}

\subsection{Linear Increase in Potential}

The purpose is to verify that the potential increases linearly with distance as you move the probe from one end to the other.\myskip

Take the small strip and connect the two output pins to the metal covered ends. This produces a potential difference between the ends. Take the reference pin and remove the pin from the end. (You only need the cable and not the pin.) Connect it to one of the output pins. Take the measuring probe and measure how the voltage increases along the center of the strip as you move further away from the reference point.\myskip

Plot a voltage vs.\ distance graph. And use LINEST to determine the best-fit line and intercepts.
\begin{itemize}
    \item Do the points fall on a reasonably straight line?
    \item How do you interpret the slope and intercept of this line?
    \item Is the field between the two poles uniform?
    \item What would you observe if you measured the voltage near the outside of the strip rather than along the center?
\end{itemize}

\section{Applications}

\begin{figure}[h]
    \begin{center}
        \includegraphics[width=0.6\textwidth]{./Exp1/pic/image5.png}
    \end{center}
    \caption{Sample Voltage Pattern for ECG}
    \label{fig:ecgpattern}
\end{figure}

The canonical example in medicine for measuring voltage potentials and displaying them (with an oscilloscope) are ECG (Electrocardiography) and EEG (Electroencephalography). The basic idea of these two standard devices is fairly simple: by measuring the potential difference in time between two points (or usually several pairs of points) one gets a nice direct insight into the activity of the heart (brain) and can therefore detect dysfunctions easily. Let us, for example, take a closer look at how an ECG works: If you record the electrical potential between two points along your chest you can record the voltage pattern shown in figure \ref{fig:ecgpattern} (in time).\footnote{Remember: The ECG only shows the electrical polarization of the heart muscle. It does not show the contraction of the heart muscle!} \myskip

As you can see, the pattern observed can be split into different parts: At first the atrial cells are depolarized, giving the first signal (P wave). (This signal is relatively weak due to the small mass of the atrium.) After a delay one gets the QRS complex, which indicates the depolarization (wave) of the ventricles. After another delay one gets the T wave, which comes from the repolarization of the heart. \footnote{The interpretation of the U wave is still not yet 100\% understood} What do all these results, obtained from the heart as a total, mean in terms of processes going on in the single cells? The next figure \ref{fig:ventricule} shows the measurement by your ECG in comparison to the potential in a single cell in a ventricle (obtained using a different method). First we recall that the interior and the exterior of the cell in their standard state have different ion concentrations and are therefore at different electrical potentials. (That is the zero line in the graph.) In the depolarization phase the ion channels in the cell membrane open and a flow of $\mathrm{K}^+$ ions changes the potential inside the cell very rapidly. After this rapid change in potential one gets a plateau, which is mainly due to the inflow of $\mathrm{Ca}^{++}$ ions into the cell. (The $\mathrm{Ca}^{++}$ ions are much bigger than the $\mathrm{K}^+$ ions, because they have a much larger hydrogen cover surrounding them and they diffuse much slower through the ion channels.) Finally in the repolarization phase the ion channels close again and the ion pumps in the cell membrane reestablish the initial ion concentrations. \myskip

\begin{figure}[h]
    \begin{center}
        \includegraphics[width=0.55\textwidth]{./Exp1/pic/image6.png}
    \end{center}
    \caption{Comparison with Ventricular Potential}
    \label{fig:ventricule}
\end{figure}

How can you now take advantage of this method for a diagnosis? First of all you don't want to take the measurements only along one single axis or plane, since e.g.\ if an infarction occurs on the front or back wall of your heart you are probably going to miss it. These days the usual way to get a 3-D picture of the position of the heart axis\footnote{The heart axis is a simplified concept of the locations of the electrical potentials in the heart. One can think of the heart axis as a vector symbolizing the (physical) axis of the heart.} is obtained by measuring with multiple channels simultaneously between the points indicated in figure \ref{fig:channels}. (The contacts with an R are placed on the patients back.) \myskip

\begin{figure}[h]
    \begin{center}
        \includegraphics[width=0.5\textwidth]{./Exp1/pic/image7.png}
    \end{center}
    \caption{Channels of Measurement}
    \label{fig:channels}
\end{figure}

\begin{figure}[h]
    \begin{center}
        \includegraphics[width=0.8\textwidth]{./Exp1/pic/image8.png}
    \end{center}
    \caption{Electric Potentials and Information}
    \label{fig:heartcomp}
\end{figure}

From the location and amplitude of the main vector (axis) one can see for example if the muscle mass is increased on one side of the heart (hypertrophy). In that case the main vector is tilted.\footnote{Also in pregnant women the heart as a total is slightly repositioned and therefore the electrical axis is a little bit off.} Another thing to look for is if the depolarization and repolarization was performed properly. For example if the ion channels in a certain region of the heart are destroyed by an infarction, then the electric potential between the depolarization and repolarization phase does not reach the zero level. By looking at your data you can not only locate the infarction, but also read off additional information, e.g.\ if the infarction killed the tissue through all of the heart wall or only parts of it. (That determines your treatment of the patient!) \myskip

You can see that you can get a lot of useful information if you look at electrical potentials (and their change in time), as shown in figure \ref{fig:heartcomp}. \myskip

Textbook references:
\begin{itemize}
    \item Stein: \emph{Internal Medicine}
    \item Harrison's \emph{Principles of Internal Medicine}
\end{itemize}

\newpage
\section{Lab Preparation Examples}

\noindent\underline{Field Lines}: Given the following equipotential lines, draw 5-10 field lines on each diagram. \myskip

\begin{minipage}[h]{0.95\textwidth}
    1.\begin{center}
        \includegraphics[width=0.7\textwidth]{./Exp1/pic/image9.png}
    \end{center}
\end{minipage}

\begin{minipage}[h]{0.95\textwidth}
    2.\begin{center}
        \includegraphics[width=0.7\textwidth]{./Exp1/pic/image10.png}
    \end{center}
\end{minipage}

\begin{minipage}[h]{0.95\textwidth}
    3.\begin{center}
        \includegraphics[width=0.7\textwidth]{./Exp1/pic/image11.png}
    \end{center}
\end{minipage}

\begin{minipage}[h]{0.95\textwidth}
    4.\begin{center}
        \includegraphics[width=0.7\textwidth]{./Exp1/pic/image12.png}
    \end{center}
\end{minipage}

\noindent\underline{Shield Theorem} \myskip

5. Explain in a few sentences why you might be safe inside your new Volkswagen Beetle even when struck by a bolt of lightning.

\chapter{DC Circuits}

\section{Introduction}

In this lab, we will investigate the mathematical relationship between voltage applied across a resistor and the corresponding current through the resistor. This relationship is knows as Ohms law and is fundamental in describing the flow of electrons through circuits. Additionally, we will demonstrate the validity of Kirchhoff's rules in analyzing circuits with multiple resistors in various geometries and determine simple rules for analyzing resistors in series and in parallel. The basic rules we will explore in this lab are essential for understanding the behavior of more complicated circuit systems used in modern technology.

\section{Theory}
\subsection{Resistance of a Uniform Wire}

Consider a uniform wire of length $l$ and cross-sectional area $A$ with a potential difference $\Delta V = V_b - V_a$ maintained across it as shown in Figure \ref{fig:cross-sec}.\myskip

\begin{figure}[h]
\centering
\includegraphics[width=0.4\textwidth]{./Exp2/pic/wireresistance.png}
\caption{Resistance of a uniform wire.}
\label{fig:cross-sec}
\end{figure}

The current $I$ that this potential difference produces can be obtained once we know the resistance $R$ of this wire:
\begin{equation}
	I = \frac{\Delta V}{R}
\end{equation}

The resistance of the wire is proportional to its length (the longer the wire, the ``harder" it is for electrons to travel from $b$ to $a$) and inversely proportional to its cross-sectional area (the wider the wire, the ``easier" it is for electrons to travel from $b$ to $a$):
\begin{equation}
	R = \rho \frac{l}{A}
\end{equation}

where the constant of proportionality $\rho$ is called the resistivity and is a characteristic of the material the wire is made of.

\subsection{Kirchhoff's Rules}

Simple circuits can be analyzed using the expression $V = IR$ and the rules for series and parallel combinations of resistors. Very often, however, it is not possible to reduce a circuit to a single loop. The procedure for analyzing more complex circuits is greatly simplified if we use two principles called Kirchhoff's rules.\myskip

\begin{enumerate}
	\item Junction Rule: The sum of the currents entering any junction in a circuit must equal the sum of the currents leaving that junction:
	\begin{equation}
		\sum I_{in} = \sum I_{out}
	\end{equation}
	This is basically just the statement of conservation of electric charge. For example, if we have a junction as shown in Figure \ref{fig:junction}, then we have $I_1 = I_2 + I_3$.

	\begin{figure}[h]
	\centering
	\includegraphics[width=0.3\textwidth]{./Exp2/pic/junction.png}
	\caption{A sample junction}
	\label{fig:junction}
	\end{figure}

	\item Loop Rule: The sum of the potential differences across all elements around any closed circuit loop must be zero:
	\begin{equation}
		\sum_{\text{Closed Loop}} V = 0
	\end{equation}
	This rule simply follows from conservation of energy.

\end{enumerate}

When applying Kirchhoff's second rule in practice, we imagine traveling around some loop and consider changes in electric potential, bearing in mind the following sign conventions:\myskip
\begin{itemize}
	\item If a resister is traversed in the direction of the current, the potential difference across the resistor is negative (Figure \ref{fig:signconventions}, (a))
	\item If a resistor is traversed in the direction opposite the current, the potential difference will then be positive (Figure \ref{fig:signconventions}, (b))
	\item If a source of EMF (assumed to have zero internal resistance) is traversed in the direction of the EMF (from $-$ to $+$), the potential difference will be positive (Figure \ref{fig:signconventions}, (c))
	\item If the source is traversed in the direction opposite the EMF (from $+$ to $-$), the potential difference will be negative (Figure \ref{fig:signconventions}), (d))
\end{itemize}

\begin{figure}[h]
\centering
\includegraphics[width=0.6\textwidth]{./Exp2/pic/signconventions.jpg}
\caption{Sign conventions for Kirchhoff's Second Rule}
\label{fig:signconventions}
\end{figure}

In practice, for a given circuit diagram, we first label all the known and unknown quantities and assign a direction to the current in each branch of the circuit. Although the assignment of current directions is arbitrary, you must adhere rigorously to the assigned directions when applying Kirchhoff's rules. After applying Kirchhoff's rules to junctions and loops as necessary, we simply need to solve the resulting equations simultaneously for the unknown quantities. If some current turns out to be negative, that simply means that it direction is opposite to that which we assigned, but its magnitude will be correct!

\section{Procedure: Ohm's Law}
\label{sec:ohmslaw}
We will using a Multimeter to make all our measurements in this experiment. A multimeter has settings to make many different types of measurements across various ranges, so make sure that when using it, you've selected the right setting for your needs.\myskip

Choose three of the resistors that you have been given. Using the chart in Figure \ref{fig:resistancechart}, decode the resistance values and record the value in the first column of Table \ref{tab:ohmslaw}.\myskip

\begin{figure}[h]
\centering
\includegraphics[width=0.8\textwidth]{./Exp2/pic/resistancechart.png}
\caption{A diagram of colorcoded resistors. (I know you can't see the color in this book)}
\label{fig:resistancechart}
\end{figure}

\subsection{Measuring Current}
To measure the current going through our resistor, we must first build an incomplete circuit! That may seem strange, but two probes of the multimeter will act as two ends of the final wire that will finish the circuit, displaying the current passing through.\myskip

\begin{enumerate}
	\item Construct the circuit shown in Figure \ref{fig:currentpic} by pressing the leads of the resistor onto two of the springs in the Experimental Section on the Circuits Experiment Board.

	\item Set the multimeter to the 200 mA range. Connect the circuit through the multimeter and read the current that is flowing through the resistor. Record this value in the second column of Table \ref{tab:ohmslaw}.

	\begin{figure}[h]
	\centering
	\includegraphics[width=0.8\textwidth]{./Exp2/pic/currentpic.png}
	\caption{A picture of measuring currents in circuits}
	\label{fig:currentpic}
	\end{figure}

	\item Remove the resistor and repeat the previous steps for the next two that you picked, recording everything in Table \ref{tab:ohmslaw}. Keep these three resistors handy for the next steps!

	\item What would happen if we completed our circuit with a wire and measured the current with the multimeter at the same two points? Think of Kirchhoff's laws.

\end{enumerate}

\subsection{Measuring Voltages}
Measuring voltage is different! Here we will construct a complete circuit and our two probes will tell us the potential (voltage) difference between those two positions in the system.\myskip

\begin{enumerate}
	\item Disconnect the multimeter and connect a wire from the positive lead of the battery directly into the first resistor you used as shown in Figure \ref{fig:voltpic}. Note that this completes the circuit. Change the multimeter to the $2V$ DC scale and connect the leads as shown also in that figure. Measure your voltage across the resistor and record it in Table \ref{tab:ohmslaw}.

	\begin{figure}[h]
	\centering
	\includegraphics[width=0.8\textwidth]{./Exp2/pic/voltpic.png}
	\caption{A picture of measuring voltage in circuits.}
	\label{fig:voltpic}
	\end{figure}

	\item Repeat the last step for the remaining two resistors.

	\item For each of your data sets, calculate the ratio of voltage divided by the resistance. Record your result in the corresponding column of Table \ref{tab:ohmslaw}. Compare the values you calculate with your measured values of the current.

	\item Construct a graph of current vs. 1/resistance ($I$ vs. $\frac{1}{R}$). Include error bars from the precision of the multimeter.

	\item Does your graph form a straight line?

	\item Draw a line of best fit and determine the slope with error using LINEST.

	\item How does your slope compare to the voltage across each resistor? Do they agree within error?

\begin{table}
\begin{center}
\begin{tabular}{| c | c | c | c |}
\hline
	Resistance ($\Omega$) & Current (A) & Voltage (V) & Voltage/Resistance (V/$\Omega$ = A) \\
	\hline
	& & & \\
	\hline
	& & & \\
	\hline
	& & & \\
	\hline
\end{tabular}
\end{center}
\caption{Resistance, current and voltage measurements for Section \ref{sec:ohmslaw}.}
\label{tab:ohmslaw}
\end{table}

\end{enumerate}

\subsection{Measuring Resistances}
\label{sec:resist}
You already know the coded resistance of each of your resistors from the colored stripes, but now we will verify how accurate those are. Measuring resistance does not require a circuit at all. With your multimeter set to resistance mode, it will act as our power supply (like a battery) so touching the probes to each side of the resistor will send current through and the multimeter's display will tell us its measured resistance.\myskip

\begin{enumerate}
	\item Choose three resistors. They could be the ones from before or entirely new ones. Enter their corresponding sets of colors in Table \ref{tab:resistance}. We will refer to one as Resistor \#1, another as \#2 and the third as \#3.

	\begin{table}
		\begin{center}
			\begin{tabular}{| c | c | c | c | c |}
				\hline
				Color Bands & Coded Resistance ($\Omega$) & Measured Resistance ($\Omega$) & $\%$ Error & Tolerance \\
				\hline
				$\#1$ & & & & \\
				\hline
				$\#2$ & & & & \\
				\hline
				$\#3$ & & & & \\
				\hline
			\end{tabular}
		\end{center}
		\caption{Resistance and error Section \ref{sec:resist}.}
		\label{tab:resistance}
	\end{table}

	\item Determine the coded value of your resistors from the color code in Figure \ref{fig:resistancechart}. Enter the Tolerance value as indicated by the color of the fourth bar in the plot.

	\item Use the multimeter to measure the resistance of each of your three resistors. Enter these values in the table.

	\item Determine the percentage experimental error of each resistance value and enter in the appropriate column.
	\begin{equation}
		\text{\% Error} = \frac{|\text{Measured} - \text{Coded}|}{\text{Coded}} \times 100\%
	\end{equation}

	\item How does the \% error compare to the coded tolerance for your resistors?

\end{enumerate}

\section{Procedure: Resistors in Series and Parallel}
It is rare that we encounter circuits with just a single resistor and power source. This section will walk through how we deal with multiple resistors in different layouts.\myskip

\subsection{Series Circuits}
\begin{enumerate}

	\item Take those same three resistors from Section \ref{sec:resist}. Connect the resistors in a series circuit shown in Figure \ref{fig:voltseries} using the springs to hold the leads of the resistors together without bending them. Connect two wires to the D cell battery and carefully note which side is positive and which is negative.

	\begin{figure}[h]
	\centering
	\includegraphics[width=0.9\textwidth]{./Exp2/pic/voltseries.png}
	\caption{A diagram of the series circuit}
	\label{fig:voltseries}
	\end{figure}

	\item First, disconnect the wires from the battery so we can measure the resistances along the circuit. Use the Multimeter to measure resistance of each individual resistor as well as each series combination by placing the two probes of your multimeter at the beginning and end of where you would like to measure. Make sure your multimeter is in resistance mode! Record your readings in \ref{tab:seriestable}.

	\item Reconnect your battery and measure the voltage across each resistor and each series combination of resistors. Make sure your multimeter is in voltage mode! Record your readings in \ref{tab:seriestable}.

	\begin{table}
	\begin{center}
	\begin{tabular}{| c | c | c |}
	\hline
		Resistor & Resistance ($\Omega$) & Voltage (V)\\
		\hline
		$R_1$ & &\\
		\hline
		$R_2$ & &\\
		\hline
		$R_3$ & &\\
		\hline
		$R_{12}$ & &\\
		\hline
		$R_{23}$ & &\\
		\hline
		$R_{123}$ & &\\
		\hline
	\end{tabular}
	\end{center}
	\caption{Voltage and resistance of series resistors}
	\label{tab:seriestable}
	\end{table}

	\item What is the apparent rule for total resistance when resistors are added up in series? Cite evidence from your data to support your conclusions.

	\item What is the pattern for how voltage gets distributed in a series circuit with equal resistances? Is there any relationship between the size of the resistance and the size of the resulting voltage?

\noindent Now we will modify our circuit slightly to allow us to measure current. Remember, for measuring current, the multimeter will complete the circuit. We will measure the current at four different locations following Figure \ref{fig:seriescurrent}. For a measurement at each location, we will remove the corresponding wire connector and replace it with the two probes of our multimeter. Remember to replace the wires when measuring other locations.

\begin{figure}[h]
\centering
\includegraphics[width=0.9\textwidth]{./Exp2/pic/seriescurrent.jpg}
\caption{A diagram of the series circuit}
\label{fig:seriescurrent}
\end{figure}

	\item Make sure your multimeter is in current mode and measure the four currents as indicated in Figure \ref{fig:seriescurrent}. You should be using the multimeter scale which goes to a maximum of 200 mA. Be careful to observe the direction of the current in your measurements. The red probe of the meter should connect to the positive lead on the circuit and the black probe to the other. Record your results in Table \ref{tab:seriescurrent}.

	\begin{table}
	\begin{center}
	\begin{tabular}{| c | c |}
	\hline
		 & Current (A)\\
		\hline
		$I_0$ &\\
		\hline
		$I_1$ &\\
		\hline
		$I_2$ &\\
		\hline
		$I_3$ &\\
		\hline
	\end{tabular}
	\end{center}
	\caption{Current across series resistors}
	\label{tab:seriescurrent}
	\end{table}

	\item What is the pattern for how current behaves in a series circuit?

\end{enumerate}

\subsection{Parallel Circuits}
\begin{enumerate}
	\item Take the same three resistors and connect them in a parallel circuit following Figure \ref{fig:voltparallel}.

	\begin{figure}[h]
	\centering
	\includegraphics[width=0.4\textwidth]{./Exp2/pic/voltparallel.png}
	\caption{A diagram of the parallel circuit}
	\label{fig:voltparallel}
	\end{figure}

	\item As before, set your multimeter to resistance mode and disconnect your battery. Measure the resistance of each individual resistor as well as one across the whole parallel set of resistors (by placing your probes at the two junction points in Figure \ref{fig:voltparallel}). Record your measurements in Table \ref{tab:paralleltable}.

	\item Reconnect your battery and measure the voltage across each resistor and the parallel combination of resistors. Make sure your multimeter is in voltage mode! Record your readings in \ref{tab:paralleltable}.

	\begin{table}
	\begin{center}
	\begin{tabular}{| c | c | c |}
	\hline
		Resistor & Resistance ($\Omega$) & Voltage (V)\\
		\hline
		$R_1$ & &\\
		\hline
		$R_2$ & &\\
		\hline
		$R_3$ & &\\
		\hline
		$R_{123}$ & &\\
		\hline
	\end{tabular}
	\end{center}
	\caption{Voltage and resistance of parallel resistors}
	\label{tab:paralleltable}
	\end{table}

	\item What is the rule for total resistance when resistors are added up in parallel? Cite evidence from your data to support your conclusions.

	\item What is the pattern for how voltage distributs itself in a parallel circuit? Is there any relationship between the size of the resistance and the size of the resulting voltage?

\noindent Now we will modify our circuit again to allow us to measure current. Review the instructions from before. We will measure the current at five different locations according to Figure \ref{fig:parallelcurrent}.

\begin{figure}[h]
\centering
\includegraphics[width=0.4\textwidth]{./Exp2/pic/currentparallel.jpg}
\caption{A diagram of the parallel circuit}
\label{fig:parallelcurrent}
\end{figure}

	\item Make sure your multimeter is in current mode as before and measure the five currents indicated in Figure \ref{fig:parallelcurrent}. Be careful to observe the direction of the current in your measurements. Record your results in Table \ref{tab:parallelcurrent}.

	\begin{table}
	\begin{center}
	\begin{tabular}{| c | c |}
	\hline
		 & Current (A)\\
		\hline
		$I_0$ &\\
		\hline
		$I_1$ &\\
		\hline
		$I_2$ &\\
		\hline
		$I_3$ &\\
		\hline
		$I_4$ &\\
		\hline
	\end{tabular}
	\end{center}
	\caption{Current across parallel resistors}
	\label{tab:parallelcurrent}
	\end{table}

	\item Are there any patterns to the way currents behave in a parallel circuit?

\end{enumerate}
 %DC circuits - Very old, need lots of work and images
%\chapter{Capacitance}

\section{Introduction}

A capacitor is a device for storing electric charge and energy. For simplicity, an ideal capacitor can be considered as a pair of parallel conducting metal plates, as shown in figure \ref{fig:capacitor}. When a charge $+Q$ is placed on the upper plate and $-Q$ on the lower plate, a potential difference $V$ is established between the plates, and the quantities $Q$ and $V$ are related by the expression:
\begin{equation}
    Q = CV
\end{equation}
where the capacitance $C$ is determined by the size and separation of the plates.

\begin{figure}[h]
    \begin{center}
        \includegraphics[width=0.5\textwidth]{./Exp4/pic/image1.png}
    \end{center}
    \caption{An Ideal Capacitor}
    \label{fig:capacitor}
\end{figure}

\subsection{Discharging a Capacitor}

If a wire is attached to the upper plate and then touched to the lower plate, charge will flow through the wire until the charge $q$ and potential difference $V$ are zero. (We will use $Q$ to indicate the initial charge and $q$ for the time dependent charge.) This process, known as discharging the capacitor, does not occur instantaneously. Instead, as the charge flows from one plate to the other, the potential difference decreases, and the current (or rate of change of charge) gradually diminishes. Since the rate of change is proportional to the amount of charge remaining, the current is initially large and then \emph{decays exponentially} with increasing time. Exponential growth or decay occurs whenever the rate of change of a quantity is proportional to the quantity present at that time. Exponentials are also characteristic of the growth rate of cells or organisms and the decay of radioactive isotopes.

The changing current which discharges a capacitor can be calculated as a function of time by considering the circuit shown in figure \ref{fig:capcircuit1}. The capacitor $C$ is initially charged by a battery of emf $\varepsilon$, placing a charge $Q = C\varepsilon$ on the plates. When the switch is closed, a current $I$ flows in the circuit, and according to Ohm's Law the voltage drop across the resistance is $IR$.

\begin{figure}[h]
    \begin{center}
        \includegraphics[width=0.4\textwidth]{./Exp4/pic/image2.png}
    \end{center}
    \caption{Discharging Circuit of a Capacitor}
    \label{fig:capcircuit1}
\end{figure}

The resistance may be a separate circuit element or merely the resistance of the wire itself. Since the sum of the voltage drops around the circuit must be zero, we obtain the equation:
\begin{equation}
    IR + \frac{Q}{C} = 0
\end{equation}

The equation can be rewritten in terms of charge alone from the definition of current, $\displaystyle I = dQ/dt \ \left(=\lim_{\Delta t\to 0} \frac{\Delta Q}{\Delta t}\right)$,
\begin{equation}
    R\frac{dQ}{dt} + \frac{Q}{C} = 0
    \label{eqn:diffcharge}
\end{equation}

Equation (\ref{eqn:diffcharge}) is a differential equation for $Q$ and has a solution which gives the time dependence of the charge
\begin{equation}
    Q(t) = C\varepsilon e^{-t/RC}
    \label{eqn:solcharge}
\end{equation}

Students familiar with calculus can verify this result by differentiating equation (\ref{eqn:solcharge}) and combining $Q(t)$ and $dQ/dt$ to show that equation (\ref{eqn:diffcharge}) is satisfied. From the definition of current, $I = dQ/dt$, we find that
\begin{equation}
    I(t) = -\frac{\varepsilon}{R}e^{-t/RC}
    \label{eqn:dischargecurrent}
\end{equation}
Note that at $t=0$, when the switch is just closed, $I = -\varepsilon/R$. As time increases, the current $I(t)$ gets smaller and reaches zero as time goes to infinity.

\subsection{Charging a Capacitor}

The process of charging an uncharged capacitor has many similarities with the process of discharging discribed above. In this case, a battery with an emf of $\varepsilon$ volts is connected in series with a resistance $R$ and the capacitance $C$, as indicated in figure \ref{fig:chargecircuit}.

\begin{figure}[h]
    \begin{center}
        \includegraphics[width=0.4\textwidth]{./Exp4/pic/image3.png}
    \end{center}
    \caption{Charging Circuit of a Capacitor}
    \label{fig:chargecircuit}
\end{figure}

When the switch is first closed, there is no charge on the capacitor and thus no voltage across it, and the full voltage $\varepsilon$ appears across $R$, causing maximum current $I$ to flow. As the capacitor becomes charged, the voltage-drop $IR$ (and thus the current $I$) gradually decreases. $I$ approaches zero as the capacitor is charged toward its maximum potential difference of $\varepsilon = Q/C$. \myskip

The equation which indicates the sum of voltage drops around the circuit loop at any time during the charging is:
\begin{equation}
    \varepsilon = IR + \frac{Q}{C}
\end{equation}
Again, this can be converted to a \emph{differential equation}, the solution of which is:
\begin{equation}
    I(t) = \frac{\varepsilon}{R}e^{-t/RC}
    \label{eqn:chargecurrent}
\end{equation}

Comparing equation (\ref{eqn:dischargecurrent}) with (\ref{eqn:chargecurrent}), we note that the current for discharging a capacitor from a given potential $\varepsilon$ decreases in time identically to the current for charging the capacitor through the same resistance to the same final voltage $\varepsilon$. The difference in the sign of $I$ indicates, of course, that the two currents flow in opposite directions.

\subsection{Graphical Presentation of Exponential Decay}

In equations (\ref{eqn:dischargecurrent}) and (\ref{eqn:chargecurrent}), the symbol $e$ stands for the constant $2.71828\cdots$, the \emph{base of natural logarithms}. A graph of the function $y = ae^{-x/b}$ is given in figure \ref{fig:expgraph}, where $y=a$ at $x=0$ and then decays exponentially as $x$ increases, reaching a value of $y = ae^{-1} = 0.37a$ at $x=b$.

\begin{figure}[h]
    \begin{center}
        \includegraphics[width=0.9\textwidth]{./Exp4/pic/image4.png}
    \end{center}
    \caption{Graph of $y=ae^{-x/b}$}
    \label{fig:expgraph}
\end{figure}

Although it is possible to determine the values of $a$ and $b$ from experimental results plotted on such a curve, it is more informative to plot $\log\,y$ versus $x$, or to use \emph{semi-log} graph paper, where the horizontal lines and scales of the $y$-axis are spaced in equal increments of $\log\,y$. Then, taking the logarithm (base 10) of $y=ae^{-x/b}$ yields:
\begin{equation}
    \log\,y = \log\,a - \frac{x}{b}\log e = \log\,a - \left( \frac{\log\,e}{b} \right)x = \log\,a - \left( \frac{0.434}{b} \right)x
    \label{eqn:logslope}
\end{equation}
which is a linear equation in $x$ representing a straight line with slope $-0.434/b$, if $\log\,y$ is plotted against $x$ as in figure \ref{fig:logplot}. \myskip

\begin{figure}[h]
    \begin{center}
        \includegraphics[width=0.8\textwidth]{./Exp4/pic/image5.png}
    \end{center}
    \caption{Plot of $\log\,y$ versus $x$}
    \label{fig:logplot}
\end{figure}

A simple way to determine $b$ from a straight line drawn through the experimental points on such a plot is to note the values of $x$ at which the line crosses values of $y$ differing by a factor of 10. Then, using the values shown in figure \ref{fig:logplot},
\begin{equation}
    \Delta(\log\,y) = \log\frac{a}{100} - \log\frac{a}{10} = (\log\,a - \log\,100) - (\log\,a - \log\,10) = -2+1 = -1
\end{equation}
Since from equation (\ref{eqn:logslope}) the slope $\Delta(\log\,y)/\Delta x$ equals $-0.434/b$, therefore
\begin{equation}
    \frac{-1}{\Delta x} = \frac{-0.434}{b},\quad\text{or}\quad b = 0.434\,\Delta x
\end{equation}
In the present experiment, $y$ repensents the current $I$; $x$ is the time $t$; $a$ is the initial value of $I$; and $b$ is the time constant $RC$. \myskip

\begin{minipage}[h]{0.7\textwidth}
    \underline{Note}: You may produce a graph like the one in figure \ref{fig:logplot} by either: 1) Plotting $\log\,y$ on the $y$ axis of normal graph paper; or 2) Plotting $y$ on \underline{semi-log} graph paper (shown in the right). Semi-log graph paper (provided in the lab) has the $y$ axis already logarithmically compressed, so you do not need to calculate $\log\,y$ for each point -- you can simply plot the $y$ values themselves. With semi-log graphs you can verify quickly whether the data behaves exponentially or not; $y = 1/x$, for example, will \underline{not} produce a straight line on a semi-log plot. Note also that on semi-log graphs $y$ approaches 0 only exponentially -- there is no $y=0$.
\end{minipage}
\begin{minipage}[h]{0.3\textwidth}
    \begin{flushright}
        \includegraphics[width=0.8\textwidth]{./Exp4/pic/imagelog.png}
    \end{flushright}
\end{minipage}
%TODO: Put in the semi-log paper graph

\section{Procedure}

\subsection{Large RC-charging}

Figure \ref{fig:largerccharge} shows the circuit to be wired for measuring the current $I$ as a function of time for charging a capacitor $C$ through a resistance $R$ to a voltage $\varepsilon$. For this part of the lab you will use the bank of three large capacitors which are attached to a piece of wood with three switches and the label ``$30\,\mathrm{MFd}$''. (In this case the prefix ``M'' in MFd indicates micro, $\mu$, or $10^{-6}$.) Be sure to use a low-voltage (15 volts max.) power supply and to connect the positive terminal on the power supply to the positive terminal on the microammeter. Before closing the switch to the power supply and starting each new measurement, momentarily connect a low resistance across the capacitor in order to start with no charge on the capacitor plates.\myskip
\begin{figure}[h]
    \begin{center}
        \includegraphics[width=0.8\textwidth]{./Exp4/pic/image6.png}
    \end{center}
    \caption{Large RC-charging Circuit}
    \label{fig:largerccharge}
\end{figure}

Measure $I$ at a series of time intervals after closing the switch. Plot the results on semi-log paper and determine the $RC$ time constant.\myskip

Repeat the above for a different value of $C$. (Note that the ``capacitor'' supplied is in fact a bank of capacitors with the provision for varying $C$ by switching in different numbers of the capacitors in parallel).

\subsection{Large RC-discharging}

Use the circuit shown in figure \ref{fig:largercdischarge} to measure the discharge current through the resistance $R$ of the same capacitors $C$ used in Part 1. You will use the same equipment as you used in Part 1 of the lab. \myskip

\begin{figure}[h]
    \begin{center}
        \includegraphics[width=0.65\textwidth]{./Exp4/pic/image7.png}
    \end{center}
    \caption{Large RC-discharging Circuit}
    \label{fig:largercdischarge}
\end{figure}

Before closing the switch to begin the measurement, momentarily connect the power supply directly across the capacitor in order to give it an initial charge. Then disconnect the power supply, close the switch, and measure $I$ as a function of time. Again, plot the results on semi-log paper and determine $RC$. Compare with the results found in Part 1 for the same values of $C$.

\subsection{A Relaxation Oscillator}

A neon bulb has the property of having a very high resistance (almost infinite) until the voltage applied to it is high enough to ``break down'' the gas, at which point the bulb lights and its resistance becomes very low. In this part of the lab you will use a high voltage power supply. Thus, \emph{it is essential that you are sure to use the small capacitor with a value of $0.082\,\mathrm{\mu F}$} rather than the bank of capacitors you use in the first two parts of the lab. Wire a neon bulb in parallel with the capacitor in the circuit of figure \ref{fig:relaxcirc}.\myskip

\begin{figure}[h]
    \begin{center}
        \includegraphics[width=0.75\textwidth]{./Exp4/pic/image8.png}
    \end{center}
    \caption{Relaxation Oscillator Circuit}
    \label{fig:relaxcirc}
\end{figure}

\textbf{CAUTION}: Since a relatively \underline{high-voltage} power supply is necessary for this part of the experiment, it is important not to touch any exposed metal parts of the circuit while the power supply is connected to the circuit and turned on, whether or not the switch is closed. Since the capacitor stores charge, \emph{it may be charged even if the voltage supply has been removed}. Be sure to discharge the capacitor fully by simultaneously touching an insulated wire to each end of the capacitor before touching any of the metal parts of the circuit.\myskip

When the switch is closed, the capacitor will start to be charged as in Part 1, with the time constant $RC$, and the high resistance of the neon bulb will have negligible effect on the circuit. However, when the capacitor is charged to sufficiently high voltage, the neon bulb will light. The capacitor will then be quickly discharged through the bulb. If $R$ is sufficiently large, there will not be sufficient current to keep the bulb lit after the capacitor is discharged. The bulb will then be extinguished; it will return to a state of high resistance, and the charging process will start again. For a given applied voltage $\varepsilon$, and a neon bulb with a given breakdown voltage, the period of this repetitive ``oscillation'' will thus be determined by the value of $RC$.\myskip

Measure the period for different combinations of $R$ and $C$, and verify its dependence on the product $RC$.

\subsection{Demonstration of Short RC Time (to be set up by Lab Instructor)}

If the $RC$ time constant of a circuit is too short to be detected using an ammeter, the effect can be displayed on an oscilloscope as illustrated in figure \ref{fig:capaoscil}. In this case, it is convenient to use a square-wave generator, which switches a voltage on and off at a variable repetition rate.\myskip

\begin{figure}[h]
    \begin{center}
        \includegraphics[width=0.8\textwidth]{./Exp4/pic/image9.png}
    \end{center}
    \caption{Using an Oscilloscope to Study the Time-dependence of Voltage}
    \label{fig:capaoscil}
\end{figure}

Use the smallest values of $R$ and $C$ available. From observation of the oscilloscope trace, plot $I$ versus time on semi-log graph paper and determine the value of $RC$. Compare the result with the value of $RC$ calculated from the labels on the circuit elements. (Remember that the total $R$ of the circuit includes any internal resistance in the square-wave generator).

%TODO: fix the layout problems

%\chapter{The Magnetic Field}

\section{Introduction}

In this lab, we will determine the strength of the magnetic field in the gap of an electromagnet in two ways: first, by measuring the force applied on a current-carrying rod; and second, by measuring the effects of changing magnetic flux through a coil that is being inserted or removed from that gap. In doing so, we will also be verifying both Faraday's law and Lenz' law! \myskip

\underline{\textbf{CAUTION:}}
\begin{itemize}
  \item Always reduce the current through the electromagnet to zero before opening the circuit of the magnet coils.
  \item Remove wrist watches before placing hands near the magnet gaps.
\end{itemize}

\section{Theory}
\subsection{Force on a Current-carrying Wire}
Consider a rod of length $L$, held horizontally and normal to the direction of a uniform, horizontal magnetic field $B$. If a current $i$ is passed through the wire, as indicated in Figure {\ref{fig:force}}, then there will be a vertical force $F=iLB$ on the wire rod.
\begin{figure}[h]
\centering
\includegraphics[width=0.6\textwidth]{./Exp5/pic/image1.png}
\caption{Force on a Current-carrying Wire}
\label{fig:force}
\end{figure}

\subsection{Induced EMF in a Coil}
According to Faraday's Law of induction, a changing magnetic flux through a coil induces an EMF (electromagnetic force) $\varepsilon$ given by
\begin{equation}
  \varepsilon=N\frac{\Delta \Phi}{\Delta t}
\end{equation}
where $\Phi = \int \vec{B} \dot \vec{dA}$ is the flux of magnetic field $B$ through a coil of area $A$ and perpendicular to that area. $N$ is the number of turns in the search coil. For this experiment, the area of the coil is constant and the magnetic field is assumed to be uniform, so the average EMF is given by
\begin{equation}
  \varepsilon= -NA\frac{\Delta B}{\Delta t}
\end{equation}
The negative sign in Faraday's Law comes from the fact  that the EMF induced in the coil acts to oppose any change in the magnetic field. This is summarized as Lenz' Law. It is important to remember that EMF, despite being called a "force" is actually a potential and is measured in volts. Voltage will be induced as the coil enters and leaves magnetic field and its direction will be determined using Lenz' law.

\section{Procedure}
\subsection{Force on a Current-carrying Wire}

The experimental set-up is shown in Figure {\ref{fig:set-up}}. The horizontal magnetic field $B$ is produced in the air gap of a ``C-shaped'' iron electromagnet. The strength of $B$ is determined by the current in the magnet coils $I$, which is supplied by an adjustable low-voltage power supply.\myskip
\begin{figure}[h]
\centering
\includegraphics[width=0.8\textwidth]{./Exp5/pic/image2.png}
\caption{Experimental Set-up}
\label{fig:set-up}
\end{figure}

A more detailed drawing of the balance and electro-magnet arrangement is shown in Figure {\ref{fig:measureforce}}.
\begin{figure}[h]
\centering
\includegraphics[width=0.8\textwidth]{./Exp5/pic/image3.png}
\caption{Setup of the Wire and Balance for Force Measurement}
\label{fig:measureforce}
\end{figure}

Note that the current $i$ through the horizontal conductor is supplied by a separate power supply. The current balance is constructed out of conducting and insulating materials such that current can enter through one side of the knife-edge fulcrum, flow through one side of the balance arm to the horizontal conductor, and then flow back through the other side of the balance arm and out through the other knife-edge. \myskip

The current $i$ in the balance is provided by the HP E3610A power supply, which can operate either in constant voltage or constant current mode. The voltage dial sets the \emph{maximum voltage} the device will supply to the circuit; the current dial likewise sets the \emph{maximum current}, and if the circuit tries to draw more current, the power supply will reduce the voltage until it reaches whatever value is needed to maintain the maximum current (by $V = IR$). \myskip

To use the power supply in constant current mode, begin with the current dial turned all the way down (counter-clockwise) and the voltage dial turned all the way up (clockwise). Set the range to 3 Amps, and connect the leads to the $+$ and $-$ terminals. You can now set the current to the desired level. Note that the digital meters on the power supply show the \emph{actual} voltage and current being supplied, so you will not normally see any current unless the leads are connected to a complete circuit. (If you want to set the current level without closing the circuit, you can hold in the CC Set button while you turn the current dial.) Once you close the circuit, the voltage adjusts automatically to maintain the constant current level, and the CC (Constant Current) indicator light should be on.

\begin{itemize}
  \item Set the current $I$ through the electromagnet at 5 amperes. Place a small number of weights on the balance and determine the value off $i$ necessary to reach equilibrium. Repeat for at least five different weights.

  \item With Microsoft Excel, plot the weight used to balance the scale vs. the balance current $i$. Include error bars.

  \item Draw a line of best fit and determine the slope with error using LINEST.

  \item From your slope, determine the magnetic field strength $B$ with error.

  \item Repeat the above steps (steps 1-4)  for two other magnet currents $I$ (for a toal of three current data sets). You do not need to do error analysis for these measurements, but plot each of your results on the same graph.

  \item Draw a diagram similar to Figure \ref{fig:measureforce} and indicate the directions of $i$, $F$ and $B$ for your setup.

  \item Discuss potential sources of error.
\end{itemize}

\subsection{Induced EMF in a Coil}

\subsection{Experimental Apparatus}

A charge integrator (the Magnetic Field Module shown in Figure {\ref{fig:module}}) is used to measure the $\Delta Q$ produced by the EMF induced in the search coil. A capacitor in the module stores the charge $\Delta Q$, and the voltage across this capacitor (read on the external voltmeter shown) is proportional to $\Delta Q$. Therefore
\begin{equation}
  V=K\Delta Q=K\frac{N}{R}\Delta \Phi
\end{equation}
where $K$ is a constant that depends on the capacitance and gain of the integrator circuit. Instead of trying to calculate a value of $KN / R$ in terms of the components, it is more direct to calibrate the combination of the search coil and the integrator circuit by measuring $V$ for a known $\Delta \Phi$. This known magnetic flux can be created by passing a measured current $I_{\mathrm{sol}}$ through a long air-core solenoid of $n$ turns per meter and with a cross-sectional area of $A_{\mathrm{sol}}$ so that
\begin{equation}
  \Delta \Phi_{\mathrm{sol}}=B_{\mathrm{sol}}A_{\mathrm{sol}}=\mu_{0}nI_{\mathrm{sol}}A_{\mathrm{sol}}
\end{equation}
where $\mu_{0}$ is the permeability of free space ( $\mu_{0} = 4\pi \times 10^{-7}\, \mathrm{T}\cdot  \mathrm{m} / \mathrm{A} $).

\subsection{Procedure}
Connect the apparatus as shown in Figure {\ref{fig:module}}. Turn on the power supply. Before any measurements are made, depress the shorting switch, then release the shorting switch and turn the drift adjust control to minimize the drift in the output voltage as observed on your meter. The shorting switch must be used to discharge the integrating capacitor prior to each measurement. Also, the drift adjust setting should be checked occasionally. If the gain setting on the Magnetic Field Module is changed, the drift adjust control must be reset.\myskip

Magnetic field measurements are made by inserting or removing the search coil from the region containing the field to a field-free region or by leaving the search coil stationary and turning the field on or off.\myskip

Since the magnetic field in the large iron-core electromagnet is much greater than the field in an air-core solenoid, the Magnetic Field Module was designed with two gain settings. In the gain=100 position, the module is 100 times as sensitive as in the gain=1 position. For measuring fields generated by the air-core solenoid, set the gain at 100. For fields generated by the large electromagnet, set the gain at 1.\myskip

Connect the solenoid to the $+$ and $-$ terminals of the HP Power Supply. A three position (on-off-on) reversing switch is part of the solenoid circuit. Flip the switch to one of the on positions and adjust the power supply current such that two amperes is flowing through the solenoid.\myskip

Slide the search coil over the solenoid and, while holding the search coil at the center of the solenoid, discharge the Magnetic Field Module and then turn the current through the solenoid off using the three position switch (or turn the current from off to on). Take several readings and record the voltage on the integrator and the current through the solenoid. Then the result is
\begin{equation}
  V_{\mathrm{sol}}=100\cdot K\cdot \frac{N}{R}\cdot \Delta \Phi_{\mathrm{sol}}=100\cdot K\cdot \frac{N}{R}(\mu_{0}nI_{\mathrm{sol}})A_{\mathrm{sol}}
\label{eq:vsol}
\end{equation}

Now set the gain to 1 on the Magnetic Field Module, and readjust the drift controls. Set the current for the large electromagnet to 5 amps, and use the search coil to measure the resulting $B$. Move the search coil \emph{gently and smoothly} into the region of the magnetic field (do not move the coil hastily as you may damage it by striking against the magnet itself). Record $V_{\mathrm{mag}}$ . The corresponding equation is:
\begin{equation}
  V_{\mathrm{mag}}=K\cdot \frac{N}{R}A_{\mathrm{coil}}B_{\mathrm{mag}}
\label{eq:vmag}
\end{equation}

Combine the results of Eq({\ref{eq:vsol}}) and Eq({\ref{eq:vmag}}) to determine the value of $B$ for the large electromagnet when the current is 5 amps, and then do the same for the other magnet currents of 4, 3, and 2 amps. Compare these values for $B$ with those obtained in Part~I.

%\chapter{$e/ m$ of The Electron}
\section{General Discussion}
The ``discovery'' of the electron by J. J. Thomson in 1897 refers to the experiment in which it was shown that ``cathode rays'' behave as beams of particles, all of which have the same ratio of charge to mass, $e/m$. Since that time, a number of methods have been devised for using electric and magnetic fields to make a precise measurement of $e/m$ for the electron. When combined with the value of the electron's charge, which is measured in the Millikan Oil Drop Experiment, the determination of $e/m$ leads to an accurate value of the mass of the electron. In the present experiment, electrons are emitted at a very low velocity from a heated filament, then accelerated through an electrical potential $V$ to a final velocity $v$, and finally bent in a circular path of radius $r$ in a magnetic field $B$. The entire process takes place in a sealed glass tube in which the path of the electrons can be directly observed. During its manufacture, the tube was evacuated, and a small amount of mercury was introduced before the tube was sealed off. As a result, there is mercury vapor in the tube. When electrons in the beam have sufficiently high kinetic energy ($10.4\,\mathrm{eV}$ or more), a small fraction of them will collide with and ionize mercury atoms in the vapor. Recombination of the mercury ions, accompanied by the emission of characteristic blue light, then occurs very near the point where the ionization took place. As a result, the path of the electron beam is visible to the naked eye. \myskip

The tube is set up so that the beam of electrons travels perpendicular to a uniform magnetic field $B$. $B$ is proportional to the current $I$ through a pair of large diameter coils (so-called ``Helmholtz Coils'') in which the coil separation is selected to produce optimum field uniformity near the center.

\section{Experimental Apparatus}
\subsection{The Sealed Glass Tube}
\begin{figure}[h]
\centering
\includegraphics[width=0.8\textwidth]{./Exp5/pic/image1.png}
\caption{Diagram of the Interior of the Sealed Glass Tube}
\label{fig:tube}
\end{figure} 

Figure {\ref{fig:tube}} shows the filament surrounded by a small cylindrical plate. The filament is heated by passing a current directly through it. A variable positive potential difference of up to 100 volts is applied between the plate and the filament in order to accelerate the electrons emitted from the filament. Some of the accelerated electrons come out as a narrow beam through a slit in the side of the cylinder. The entire tube is located inside a set of coils, which produce a uniform magnetic field $B$ perpendicular to the electron beam. The magnitude of the field can be adjusted until the resultant circular path of the electron beam just reaches one of the measuring rods. These rods are located along a cross bar, which extends from the cylinder in a direction perpendicular to that in which the electron beam was emitted--i.e., along a diameter of the circular orbits. 

\subsection{The Helmholtz Coils and Uniform Magnetic Field}
The magnetic field produced at the position of the electron beam by a current $I$ flowing through the coils must be computed. For a single turn of wire of radius $R$, the field on the axis at a distance $x$ from the plane of the loop is given by:
\begin{equation}
  B'=\frac{\mu_{0}R^{2}I}{2(R^2+x^2)^{3/2}}.
\end{equation}

For the arrangement in Figure {\ref{fig:coils}}, there are two loops with $N$ turns each, separated by a distance equal to the coil radius $R$. The coils contribute equally to the field at the center:
\begin{equation}
  B_{I}=\frac{\mu_{0}R^2NI}{\big[R^2+(R/2)^2\big]^{3/2}}=\frac{4\pi\times 10^{-7}NI}{R(1+\frac{1}{4})^{3/2}}=\mathrm{constant}\times I\, \mathrm{Tesla}
\end{equation}
where $N = 72$ is the number of turns of each coil and $R = 33\, \mathrm{cm}$ is the radius of the coils used.\myskip
\begin{figure}[h]
\centering
\includegraphics[width=0.8\textwidth]{./Exp5/pic/image2.png}
\caption{Helmholtz coils used to produce a uniform magnetic field.}
\label{fig:coils}
\end{figure} 

This arrangement, called a pair of \textbf{Helmholtz coils}, yields a remarkably uniform field in the region at the center.\myskip

The net field $B$ in which the electrons move is not $B_I$ alone, but the resultant of the earth's field $B_e$ and $B_I$ . If the equipment is oriented so that the field of the Helmholtz coils is parallel to that of the earth, and if the current through the coils causes $B_I$ to be directed opposite to $B_e$ , then
\begin{equation}
  B=B_{I}-B_{e}
\label{eq:b}
\end{equation}

\subsection{The Trajectories of the Electrons in the Glass Tube}
If an electron of charge $e$ and mass $m$ starts nearly from rest and is accelerated through a potential difference $V$ to a final velocity $v$, then
\begin{equation}
  \frac{1}{2}mv^2=eV\quad \mathrm{or}\quad \frac{e}{m}=\frac{v^2}{2V}
\label{eq:ev}
\end{equation}

If the electron then enters a uniform magnetic field $B$ which is perpendicular to its velocity, it will move in a circular orbit of radius $r$, where
\begin{equation}
  \frac{mv^2}{r}=evB\quad\mathrm{or}\quad \frac{e}{m}=\frac{v}{Br}
\label{eq:eb}
\end{equation}

If it were possible to measure the velocity directly, then $e/m$ could be determined by measurements of either the electric or magnetic field alone. Since a direct measurement of $v$ is not feasible in this experiment, $e/m$ can be determined from the combination of electric and magnetic fields used. Specifically, by eliminating $v$ from equations ({\ref{eq:ev}}) and ({\ref{eq:eb}}), $e/m$ can be expressed directly in terms of $V$, $B$, and $r$.\myskip

Instead of determining $e/m$ from a single measurement of $r$ for given values of $V$ and $B$, however, it is preferable to measure the variation of $r$ with $B$ (or $I$) at fixed values of $V$. In particular, the data can be presented in convenient form by plotting the \textbf{curvature} $1/r$ as a linear function of $I$.
\begin{equation}
  \frac{1}{r}=\sqrt{\frac{e}{m}\frac{1}{2V}}B_{I}-\sqrt{\frac{e}{m}\frac{1}{2V}}B_{e}
\label{eq:r}
\end{equation}

Derive ({\ref{eq:r}}) from ({\ref{eq:b}}), ({\ref{eq:ev}}), and ({\ref{eq:eb}}) for your report \underline{before} coming to laboratory.\myskip

Note that equations ({\ref{eq:ev}}), ({\ref{eq:eb}}), and ({\ref{eq:r}}) apply strictly only to electrons with trajectories on the \emph{outside edge} of the beam -- i.e., the most energetic electrons. There are two reasons why some electrons in the beam will have less energy:
\begin{enumerate}
\item There is a small potential difference across the filament caused by the heating current. Only electrons leaving the negative end of the filament are accelerated through the whole potential difference $V$.
\item Some of the electrons in the beam will lose energy through collisions with mercury atoms.
\end{enumerate}

\section{Procedure}
\subsection{Orientation of the Coil and Tube Setup}

For reasons already explained, we would like to orient the Helmholtz coils such that their axes are parallel to the ambient magnetic field.\myskip

\emph{Please exercise extreme care in the following section as you align the Helmholtz coils: the cathode ray tube is very delicate and may break if the coil support is not very firmly secured and falls.  It is recommended that two people handle the coil frame at all times while the coils are being aligned.  Also take care not to touch the dip needle, which is easily bent.}\myskip

In order to align the coils so that their axis is aligned with the ambient magnetic field proceed as follows:
\begin{itemize}
    \item With the coils in the horizontal position, rotate the horizontal arm of the dip needle support so that the needle itself and the plastic protractor are horizontal.  Avoid touching the needle or plastic protractor, and instead turn the arm using the attached lever.
    \item Allow the compass needle to come to a rest.  It is now pointing in the direction of the horizontal component of the ambient field.
    \item Rotate the entire frame by turning the base, until the compass needle is aligned along the $90^\circ$-$270^\circ$ line on the protractor.  The horizontal axis of the cathode ray tube (coaxial with the metal rod) is now aligned with the horizontal component of the ambient field.
    \item Rotate the horizontal arm of the dip needle support so that the needle itself and the plastic protractor are now in a vertical plane.  Avoid touching the needle or plastic protractor, and instead turn the arm using the attached lever.
    \item Allow the needle to come to a rest.  It is now pointing in the direction of the ambient field.
    \item Loosen the wingnut and \textbf{gently} raise one side of the coils.  You want to increase the angle until the dip needle is aligned along the $0^\circ$-$180^\circ$ line on the protractor.  \textbf{Hold the wooden frame rather than the coils as you raise the setup}.
    \item Securely \textbf{tighten the wingnut so that the coils remain in position}.  One person should be supporting the frame while a second person tightens the wingnut.
    \item The coil axis should now be aligned (coaxial) with the ambient magnetic field.
\end{itemize}

\subsection{Preliminary Adjustments}
The supplies and controls for the Helmholtz coils and the filament are permanently wired on a board and are designed to minimize the possibility of damage to the tube or coils. Locate each control, and note the qualitative effects observed when the control is varied. In particular:
\begin{enumerate}[(a)]
\item Figure {\ref{fig:tube}} shows that the filament and its associated lead wires form a small loop. Since a $4\,\mathrm{amp}$ current is required to heat the filament, this loop creates a measurable field. The filament coil reversing switch permits you to study the effect of this field. The effect can be minimized in the experiment by rotating the tube slightly in its mounting so that the electrons come out parallel to the plane of the coils.
\item Note the direction of the coil current for each position of its reversing switch by using the dip needle to check the direction of the resultant field. Knowing the field direction, check the sign of the charge of the particles in the deflected beam. Also, determine whether the earth's field adds to or subtracts from the coil field.
\item The beam will have a slight curvature in the earth's field when the coil current is zero. Make a preliminary measurement of the earth's field by adjusting the coil current to remove this curvature. The special Meter Switch and low current meter ($200\, \mathrm{mA}$) will enable you to measure the relatively small current needed, and the straight line trajectory can be checked by comparison with the light emitted from the filament.
\end{enumerate}

\subsection{Measurement of the Circular Orbits}
With the accelerating voltage at an intermediate value, the current in the Helmholtz coils can be adjusted so that the outside edge of the beam strikes the outside edge of each bar in turn. Measure field current as a function of radius for the highest voltage $V$ which allows you to adjust the beam with respect to all five bars.\myskip

For one measurement, test the reproducibility of the current setting as an aid to error analysis. The tube manufacturer supplies the following values for the \emph{diameters} from the filament to the \emph{outside} of each bar in succession:
\begin{gather*}
6.48\,\textrm{cm},\quad 7.75\,\textrm{cm},\quad 9.02\,\textrm{cm},\quad 10.30\,\textrm{cm},\quad 11.54\,\textrm{cm}
\end{gather*}

\subsection{Calculations}
Plot a graph of $1/r$ versus $I$, and draw the straight line that gives a best fit to the five measured points. Use equation ({\ref{eq:r}}) to calculate $e/m$ from the slope of this line. Write your report up to this point and then proceed to the following:

\subsection{Further Considerations}
\begin{enumerate}[(a)]
\item Calculate the percentage difference between your value of $e/m$ and the accepted value which is $1.758 \times 10^{11}\, \mathrm{coulombs} / \mathrm{kg}$. What do you think is your largest source of error? Can you account for this much error by a numerical estimate?
\item Calculate the actual maximum velocity of the electrons in your beam.
\item Make another run with a lower accelerating voltage. Do you find a consistent value of $e/m$?
\item Compare the intercept of your graph with the value of coil current you obtained by balancing the earth's field. If these numbers are not roughly the same, you may have made an error. Note that this current is not an important number, but it makes a good check on your technique. Physicists frequently check the consistency of their data by computing numbers which they do not ``need''.
\item Calculate the actual value of $B_e$ .
\item Test the maximum error in your readings which could be caused by a change in the field of the nearest neighboring coil.
\end{enumerate}

%\chapter{Optics I: Reflection, Refraction, and Lens Equation}

\section{Introduction}

This experiment is the first of two experiments introducing the basic ideas of geometrical optics. These experiments will introduce you many new physics concepts so it is important to read the theory section before your lab section. \footnote{The material in these labs is discussed in the assigned lecture course textbook (Fundamentals of Physics, 10th Ed., Halliday, Resnick \& Walker, Chapters 34-38) In particular, we adopt the sign conventions used in the course textbook.}  They are designed to present the ideas of geometrical optics from an empirical point of view.\myskip

In this first experiment, we utilize a simple model describing light as a bundle of rays.  We then explore the behavior of these rays as they reflect from smooth surfaces and as they are transmitted, or refracted, through transparent media.  We will experimentally verify Snell's law of refraction and observe the phenomenon of total internal reflection as a consequence of Snell's law.\myskip

We will subsequently use the concept of light rays to describe ray diagrams and derive the lens equation\footnote{Many approximations are made with ``thin lenses''. However, even the most sophisticated treatments of optics usually begin with these ``thin lens approximations''.}. These are powerful tools to draw conclusions based on geometrical optics. In this experiment, we will verify the lens equation for real and virtual images. In the next experiment, we will use these tools to build a magnifying glass, microscope, and telescope.\myskip

Why is optics important (besides the fact that the human eye works by the rules we describe here)? Optical instruments are used in many real-life situations in which an image of a system is needed.  Such cases apply whether the system is large or small, and whether easily accessible or not.  For example, whenever you want insight into how a biological system works, you will likely need to use imaging methods based on geometrical optics. Even if you are only interested in the final data from some fancy imaging system, it will be essential for the quality of your work that you know how to evaluate what you are seeing and what the limitations of the method are.\myskip

\emph{Remark}: It is strongly recommended that you look over the setup in the Lab Library before coming to the lab. It is difficult to understand optical phenomena in the abstract so simply reading the lab manual might not be enough to fully prepare for this lab.  Be sure to stop by the Lab Library and let the TA show you the apparatus!

\section{Theory}

\subsection{Geometrical Optics}

Geometrical optics is a simplified way of describing light phenomena. It is valid as long as we do not consider cases in which light passes through small pinholes or slits, or examine the edges of shadows. In your lecture and lab this semester, you will encounter effects such as diffraction and interference, which are much more complicated and cannot be accurately explained by the theory of geometrical optics. The assumption for geometrical optics is that rays of light propagate along straight lines until they are reflected, refracted, or absorbed at a surface. \myskip

A very simple, but important, application of geometrical optics is X-rays. (X-rays are a type of `light', with wavelengths that are much shorter than those characteristic of visible light.)  The shadow image of the skeleton made in an X-ray may be understood by simply propagating straight lines from the source to the detector, except for those rays absorbed by some tissue (e.g. bone) in the path.

\subsection{Light as Rays}

The concept of light rays may become more plausible to you if you imagine being in a dark, dusty room with light from the outside entering through a small hole. You can ``see'' a ray of light traveling in a straight line from the hole to wherever it hits the wall.  (Your eye actually sees the light scattered by dust particles along the path of the ray.)\myskip

We describe light using the following simple model. A light source emits rays of light in all possible directions. Each ray propagates along a \emph{straight line} until one of the following happens. It is:
\begin{enumerate}
    \item	reflected if it encounters a reflective material.
    \item	refracted if it encounters a transparent material.
    \item	absorbed if it encounters a non-reflecting, non-transparent material.
\end{enumerate}
For most materials in the real world, combinations of these effects occur.\myskip

\begin{figure}[h]
    \begin{center}
        \includegraphics[width=0.5\textwidth]{./Exp6/pic/basicsnelllaw.png}
    \end{center}
    \caption{Reflection and Refraction of Light Ray Across a Surface}
    \label{fig:rayref}
\end{figure}

In geometrical optics, we assume that each ray of light travels along a straight line indefinitely unless it strikes a boundary, like a mirror or an interface between air and another material. A source of light, such as a flashlight, produces many rays that travel nearly parallel to each other on their way to the illuminated spot on the boundary. If a ray is not absorbed when it strikes the boundary, it splits into parts as shown in the figure above. Both the reflected and refracted rays again travel in straight lines until they encounter another boundary. The angles $\theta_{i}$, $\theta_{r}$, and $\theta_{t}$ are measured with respect to the normal (perpendicular) to the boundary surface.\myskip

Although in actuality a light ray that encounters a surface experiences all three of these effects, there is usually one that dominates. If the boundary is a metallic mirror, only the reflected ray is relevant, and if the boundary is a transparent medium such as water, only the refracted ray is relevant. In sections 2.4 and 2.6, we describe the quantitative relationships between the incident ray and the reflected and refracted rays. In particular, we write the relations between the angles $\theta_{i}$ and $\theta_{r}$, and between the angles $\theta_{i}$ and $\theta_{t}$ shown in Figure \ref{fig:rayref}.

\subsection{Absorption/Color (a brief aside)}

The colors we perceive in light are directly correlated with the wavelength of the light ray (for example, blue light has a shorter wavelength than red light). White light is the response your brain perceives if light of all colors strikes your eye's receptors.  \myskip

Materials that absorb light can either decrease the reflected intensity for all colors equally (which means that the light simply appears less bright than before) or they can selectively decrease the intensity of some colors. An example of the second case is white light from the sun falling onto the grass where the grass absorbs all colors except green (which it reflects). This is why grass appears to you as green. If an object absorbs all the light, it appears to be black.\myskip

All materials absorb to some extent, even when the light appears to pass through or reflects. (The best commercial mirrors reflect about 99.99\% of the incoming light.)

\subsection{Reflection}

The ray reflected from a surface emerges at an angle equal to the angle of incidence of the original light ray.  Quantitatively, this law of reflection is expressed in terms of the angles shown in the figure as:
\begin{equation}
    \theta_{r} = \theta_{i}
\end{equation}
There are two general types of reflection -- diffuse and specular. Diffuse reflection is exemplified by the reflection from the surface of this page. On a microscopic scale, the surface of the paper is quite rough; consequently, parallel rays striking even nearby parts of the paper's surface are characterized by different angles of incidence. Each of the initially parallel rays is reflected in a different direction.  Therefore, when we shine a flashlight on a piece of paper, irrespective of our position, we see a bright spot on the paper.\footnote{Only a few of the light rays are reflected into our eyes!}  In this case, it is not feasible to determine the relationship between $\theta_{i}$ and $\theta_{r}$ for any particular ray.\myskip

If parallel rays reflect from a surface which is very flat and smooth (such as that found on a mirror or on window glass), there is a unique angle of incidence and therefore a unique angle of reflection. This type of reflection is referred to as specular or regular. A flashlight beam reflected specularly will only be observed if it is viewed along the direction of reflection. We restrict our attention here to specular reflection, since only in that case the relationship between the incident and reflected rays can be understood well enough to be used in optical instruments.  \myskip

You may have seen fresh snow glitter in the sunlight, and wonder which type of reflection that is. The explanation is that some of the small flat surfaces of the snowflake are smooth and act like mirrors that reflect several rays in the same direction, which just happens to be where your eye is. But the totality of the surface of white snow is irregular and reflection from the snow more often looks much like the diffuse reflection from this page.\myskip

\emph{Remark for Experts}: There is of course no \underline{perfectly} smooth surface in nature. Every surface is rough on the atomic level. To get specular reflection, it is sufficient that surface irregularities are small compared to the wavelength of light. So when you polish anything, from optical instruments to your furniture, the desired successful appearance is accomplished when you have made all surface irregularities small compared to the wavelength of light (which is about $5\times 10^{-5}\,\mathrm{cm}$, or $500\,\mathrm{nm}$). Since atoms are much, much smaller than that, there is no contradiction.

\subsection{Real and Virtual Images}

When you look in a mirror, you see your own image. It looks as if a copy of you is standing behind the mirror. But if you put a screen behind the image at the position where your copy seems to stand, you would obviously not capture the image of yourself on the paper. Your copy in the mirror or any image you cannot capture on a screen at its apparent location is called a virtual image. If you can record the image on a screen at its apparent location, it is called a real image. Real images are caused by the actual convergence of light rays at a given point; virtual images are caused by light rays that diverge in such a way that they seem to, but do not actually, converge at a certain point.\myskip

Mirrors produce virtual images. In the next two labs, we will deal with many different examples of real and virtual images.

\subsection{Refraction and Snell's Law}

Light waves (rays) propagate through vacuum with a fixed velocity $c$ equal to about $300,000\,\mathrm{km/s}$ or $3\times 10^8\,\mathrm{m/s}$. One of the consequences of our understanding of electromagnetism, detailed in Einstein's theory of special relativity, is that nothing can travel faster than this speed.  \myskip

As the waves travel through any material, interactions of the light with the atoms result in a velocity $v$ that is smaller than $c$. The ratio of these speeds is called the refractive index $n$, and is a specific constant associated with the medium:
\begin{equation}
    n = \frac{c}{v}
\end{equation}
For visible light traversing through most transparent media, the index varies roughly between 1 and 2.5, depending on the material. Although in actuality, light travels through air slightly slower than it does through a vacuum, for our purposes we will consider the refractive index to be 1. Typical glass, for example, has a refractive index of about 1.5.\myskip

When a ray of light in air encounters a medium, the change in wave velocity requires it to change direction.\footnote{The directional change follows from a simple physical argument.}  The new angle relative to the normal, shown in the figure, is the angle of refraction $\theta_{t}$ given by Snell's Law
\begin{equation}
    n_{i}\sin \theta_{i} = n_{t} \sin \theta_{t},
\end{equation}
which reduces to the following equation when the incident ray travels through vacuum/air
\begin{equation}
    \sin \theta_{t} = \frac{\sin \theta_{i}}{n_{t}}
\end{equation}

For example, if an incoming ray at an angle $\theta_{i} = 15^\circ$ encounters glass (with an refractive index of $n = 1.5$), then the refracted ray will change direction and travel through the glass at an angle of $\theta_{t} = 9.9^\circ$.  Since $n$ is always larger than unity, the refracted ray entering a material is closer to the normal than the incident ray.\myskip

\emph{Remark for Experts - A Physical Argument for Snell's Law}: Fermat's principle states that a traveling light wave will follow the path of least time.  Imagine two points, where light is moving from Point A to Point B.  Now if the space between them is empty the light will travel at $c$ for the entire duration, and thus the path of least time is equivalent to that of shortest distance, which we know to be a straight line.  Thus in the case of empty space Fermat's principle reduces to the trivial answer that light will move in a straight line.  However, when the speed at which the light travels is not constant for the entire journey, the picture can change dramatically.  If we now let Points A and B be inside materials with indices of refraction $n_{1} = \frac{c}{v_{1}}$ and $n_{2} = \frac{c}{v_{2}}$, respectively, we can imagine a theoretically infinite number of paths the light can take between the two.  If we now calculate the time it takes the light to traverse each path, we would find the least amount of time occurs on the path where $n_{1}\sin \theta _{1} = n_{2}\sin \theta _{2}$ or Snell's Law!  Note that if we divide each side by $c$ we now get
\begin{equation}
	\label{eq:snell2}
	\frac{1}{v_{1}}\sin \theta _{1} = \frac{1}{v_{2}}\sin \theta _{2},
\end{equation}
which relates the speeds of light in each medium to the path the light follows.  You may wonder, does this have applications outside of light?  The answer: yes!  Imagine you're playing Frisbee at the beach and your friend throws it over your head and into the water.  You're on the sand, where you know you run at speed $v_{1}$; likewise you know you swim at speed $v_{2}$.  If you want to minimize the time it takes you to retrieve your Frisbee, just follow the path defined by Equation \eqref{eq:snell2} (of course, it may take you longer once you carry out the necessary calculations and measurements).  The only difference with light is it \textit{must} ``choose" this path, whereas you have the free will to consider other options.

\subsection{Total Internal Reflection and Critical Angle}

Snell's Law describes the paths light rays follow when traversing interfaces between two media.  Though in the previous section we considered light passing from a lower index of refraction to a higher, it is equally valid to consider the reverse. We would therefore see that when light passes from glass to air, the ray exits with a larger $\theta_{t}$ than $\theta_{i}$.  In this example Snell's Law can be reduced to: $n \sin \theta_{i} = \sin \theta_{t}$.  The refracted ray must have a larger angle to the normal than the incident ray, but this angle must always be less than $90^\circ$ in order for the refracted ray to emerge on the opposite side of the interface from the incident ray.  Therefore, at some maximum incident angle the refracted ray travels along the surface; for angles beyond this, no refracted ray emerges and the ray is reflected entirely back into the glass! This angle is called the critical angle, $\theta_c$, and from Snell's Law is defined by
\begin{equation}
    \sin\theta_c = \frac{1}{n}.
\end{equation}
An easy way to determine the index of refraction of a medium is by measuring the critical angle and using this equation. As you can see, total internal reflection can only happen when a light beam passes from a medium with a higher refractive index to a medium with a lower refractive index.

\subsection{Lenses}
\label{sec:lenses}
Lenses are made of glass, or a similar transparent material which refracts light at its surfaces. They are shaped so that they cause a bundle of parallel rays to converge to or (seem to) diverge from, a single point, called the focal point. The focal length, $f$ , is defined as the distance between the lens and the focal point. The focal length is a fixed characteristic of a given lens, and depends only on the lens material and shape. A lens that converges light is called convex, and its value of $f$ is positive. A lens that diverges light is called concave, and its value of $f$ is negative. We deal only with converging lenses in this lab.
\begin{figure}[h]
\centering
\includegraphics[width=0.8\textwidth]{./Exp6/pic/image1.png}
\caption{Converging Lens}
\end{figure}

\emph{Remark}: Sometimes lenses have a number printed on them which indicates the focal length, in millimeters. A positive number indicates a convex/converging lens while a negative number indicates a concave/diverging lens.\myskip

\emph{Remark for Experts}:  If a bundle of parallel rays falls on the lens but not along the axis of the lens, then the focal point will be shifted upward or downward along a line perpendicular to the axis (in the ``focal plane''). Note that the focal distance along the axis stays the same.
\begin{figure}[h]
\centering
\includegraphics[width=0.6\textwidth]{./Exp6/pic/image2.png}
\caption{Focal Plane}
\end{figure}

\subsection{Ray Diagrams}
Ray diagrams allow us to trace the paths of rays. They make it easier to understand how images are formed and what lenses do. \myskip

To draw ray diagrams, follow a few simple rules illustrated in the figure below:
\begin{enumerate}
  \item Mark the focal points of the lens on each side of the lens.
  \item If two rays from a source point intersect at another point, then all rays from that source point will intersect at the second point.
  \item When drawing ray diagrams for convex lenses, it is most convenient to use three specific kinds of rays from the tip of the object ($P$) labeled $a$, $b$, and $c$ in the figure and described here:
  \begin{enumerate}
    \item Rays that enter the lens parallel to the axis pass through the focal point behind the lens.
    \item Rays that pass through the focal point in front of the lens leave the lens parallel to the axis of the lens.
    \item Rays going through the center of a lens will go through the lens in a straight line and do not bend. (These are the only rays that have no net refraction from the lens)
  \end{enumerate}
\end{enumerate}

\begin{figure}[h]
\centering
\includegraphics[width=0.8\textwidth]{./Exp6/pic/image3.png}
\caption{Ray Diagram}
\end{figure}

\subsection{Lens Equation}
\label{sec:lens}
The ray diagram described in the previous section provides a graphical method for locating images. Using the simple geometrical argument given here, with the notation indicated in Figure \ref{fig:lenseq}, we derive the lens equation. This equation relates the image distance $S'$, the object distance $S$, and the focal length $f$.
\begin{figure}[h]
\centering
\includegraphics[width=0.8\textwidth]{./Exp6/pic/image4.png}
\caption{An image $P'$ of an object $P$ is formed by a lens}
\label{fig:lenseq}
\end{figure}

Since triangle $COP$ is similar to triangle $CO'P'$:
\begin{equation}
  \frac{S'}{S}=\frac{CO'}{CO}=\frac{O'P'}{OP}
  \label{eq:one}
\end{equation}

Since triangle $F_2O'P'$ is similar to triangle $F_2CR$:
\begin{equation}
  \label{eq:two}
  \frac{S'-f}{f}=\frac{O'F_{2}}{CF_{2}}=\frac{O'P'}{RC}=\frac{O'P'}{OP}
\end{equation}

Combining equations ({\ref{eq:one}}) and ({\ref{eq:two}}):
\begin{equation}
  \frac{S'}{S}=\frac{S'-f}{f}
\end{equation}
which reduces to:
\begin{equation}
  S'f=SS'-Sf
\end{equation}
and finally, dividing by $fSS'$ and rearranging, we obtain the thin lens equation:
\begin{equation}
  \frac{1}{f}=\frac{1}{S}+\frac{1}{S'}.
\end{equation}

The sign convention that we have used in deriving this equation, sets $f$ is positive for a converging lens, $S$ is positive on the side of the lens where the object is placed, and $S'$ is positive on the other side of the lens. Sign convention is \emph{incredibly important to understand for this experiment}. For $S>f$, $S'$ is also positive and a real image exists on the other side of the lens, as shown in the figure above. If the object were placed inside the focal length, i.e. if $S<f$, then the solution of the lens equation would give a negative value for $S'$ and the image would appear on the same side of the lens as the object.\myskip

The magnification of the image is the ratio of the image height to the object height. Eq. ({\ref{eq:one}}) shows that this magnification is numerically equal to $S'/S$, the ratio of image distance to object distance. The figure above also shows that the real image, formed when $S>f$, is inverted.

\section{Experiments}

\begin{figure}[h]
    \begin{center}
        \includegraphics[width=0.8\textwidth]{./Exp6/pic/equipment.jpg}
    \end{center}
    \caption{Equipment for the Experiment}
    \label{fig:equip4}
\end{figure}

\subsection{Law of Reflection and Snell's Law}
\label{sec:refsnell}
In the first part of the lab we verify the Law of Reflection and Snell's Law with the materials seen in Figure ~\ref{fig:equip4}.  To begin, set the D-shaped acrylic lens in in the similarly-designed outline on the Ray Table.  Rotate the knob on the front of the Versatile Light Source until one light ray emerges, and arrange it so it perpendicularly crosses the flat side of the center of the acrylic lens (you can arrange it to align with the ``Normal" arrow on the Ray Table).  Now rotate the Ray Table such that the ray still crosses the center of the lens, but is no longer orthogonal, as in Figure \ref{fig:slaw}.  You should see transmitted and reflected rays.  Using the angle markings on the Ray Table, measure and record these angles along with that of the incident ray.  Repeat this process for two other incident angles.\myskip

\begin{figure}[h]
\centering
\includegraphics[width=0.8\textwidth]{./Exp6/pic/snelllawdiagram.png}
\caption{This diagram demonstrates what should happen as a bundle of light rays incident on the face of a D-shaped acrylic lens at angle $\theta_{i}$ are reflected and refracted at angles $\theta_{r}$ and $\theta_{t}$, respectively.}
\label{fig:slaw}
\end{figure}

Consider $\theta_{i}$ vs. $\theta_{r}$:
\begin{enumerate}
	\item Do the incident and reflected rays intersect at the reflecting edge of the acrylic?
	\item Estimate a reasonable uncertainty for these angles, considering the technique used to construct and measure them.
	\item Are the two angles equal within uncertainty?
	\item With Microsoft Excel, plot $\theta_{i}$ vs. $\theta_{r}$ with error bars and draw construct a best-fit line.
  \item Determine the slope and intercept of your best-fit line using the LINEST method. What would you expect for the values of the slope and intercept? Why?
  \item What were the main errors in performing this experiment?
\end{enumerate}
\myskip

\noindent{Consider $\sin \theta_{i}$ vs. $\sin \theta_{t}$:}
\begin{enumerate}
    \item Calculate the refractive index $n$ explicitly from each of the six pair of rays in your data. Calculate the mean and estimate the uncertainty of a single measurement using the 2/3 method.
    \item Does your measured index of refraction for acrylic agree with the accepted value of approximately 1.49 within error?
    \item Discuss the main sources of error in determining the index of refraction.
\end{enumerate}

\subsection{Total Internal Reflection and Critical Angle}

\begin{figure}[h]
\centering
\includegraphics[width=0.8\textwidth]{./Exp6/pic/tirdiagram.png}
\caption{This demonstrates how one achieves total internal reflection in the case where $n_{1}>n_{2}$ and thus $\theta_{i}<\theta_{t}$.  Note because a line running from a circle's center to any point is necessarily orthogonal to the circumference, some simple geometry allows us to easily determine the $\theta_{i}$ that leads to the critical angle $\theta_{c}$.  It is recommended you study this figure until you are convinced of this.}
\label{fig:tirdiagram}
\end{figure}

In the previous section, we found $\theta_{i} > \theta_{t}$ as the light passed from air to acrylic (Snell's Law guaranteed this because $n_{\rm air} < n_{\rm acrylic}$).  Total internal reflection may only occur when light passes from a medium with a higher index of refraction to one with a lower ($n_{i} > n_{t}$).  For this portion of our experiment we want to setup a scenario in which light passes from acrylic to air.  Figure \ref{fig:tirdiagram} shows a schematic for how we can accomplish this.\myskip

Begin with the ray of light from the Versatile Light Source orthogonally penetrating the curved portion of the D-shaped acrylic lens along the "Normal" line that is transcribed on the Ray Table.  You should see the ray of light emerge perpendicularly to the flat surface of the acrylic along the opposite side's ``Normal" tracing.  Next begin rotating the Ray Table.  In doing so, you should notice the emerging ray of light from the acrylic's flat side begins to bend.  Though the effect is initially subtle, eventually you should reach an angle where the refracted light is now at $\theta_{t}=90^{\circ}$.  This incident angle has reached the critical angle $\theta_{c}$ - the point beyond which we have total internal reflection, and no refracted light escapes from the acrylic.

Measure the angle $\theta_{c}$ at which total internal reflection occurs.
\begin{enumerate}
    \item   From $\theta_{c}$ determine the index of refraction for the acrylic lens with error.  Error can be found by propagating uncertainty in $\theta$ using
    \begin{equation}
      \sigma_{\sin(\theta)} = \cos(\theta) \sigma_\theta
    \end{equation}
    \item	Does this value for $n$ agree with your findings in Section \ref{sec:refsnell} and with the accepted value within error?
    \item Discuss the main sources of error in measuring the index of refraction in this section.
\end{enumerate}

\noindent{\emph{An alternate way to achieve total internal reflection}: Begin with the ray of light from the Versatile Light Source orthogonally aligned with the flat face of the D-shaped acrylic lens, as we did with Snell's Law.  This time instead of rotating the Ray Table, simply slide it sideways so that the light beam remains perpendicular to the flat face of the lens, but now crosses on either the left or right side of the center.  You should notice that the light ray remains perpendicular to the face once in the acrylic.  However, it will now hit the curved portion of the D-shaped lens at an angle less than $90^{\circ}$, and as a result is refracted out the back.  Eventually you will reach a distance where you notice the light is no longer refracted, but appears to bounce around the circular portion of the lens, due to recurring total internal reflections.}\myskip

\subsection{Measuring the Focal Length of a Lens}
\label{sec:lengthI}
\begin{figure}[h]
\centering
\includegraphics[width=0.7\textwidth]{./Exp6/pic/image8.png}
\caption{Measuring the Focal Length}
\label{fig:focusedray}
\end{figure}
Using a lens and a piece of paper, focus the image of an object outside the laboratory window (such as a tree or light outside the window) onto the paper, like illustrated in Figure \ref{fig:focusedray}.  If it is dark outside or the window blinds are closed, you can use a lightbulb in the room. Based on the fact that light from a distant source may be safely assumed to contain only parallel rays, the distance between the lens and the paper should be the focal length of the lens. You may already suspect that this method is not very precise and that you will get large uncertainties.

\begin{enumerate}
\item Measure this distance and estimate uncertainty for two lenses of your choosing.
\item Take note of the orientation of the image on your paper.
\item How could you improve this simple procedure?
\end{enumerate}

\subsection{Lens Equation}
\label{sec:lensequation}

Set up your optical bench in the following way: stand the Versatile Light Source upright on one end of the optical bench such that the crossed arrow target points down the bench. Place one convex lens on the bench a distance that is greater than its focal length $f$ from the Light Source.  Finally, place the Viewing Screen on the opposite side of the lens and move it forward and backward until the crossed arrow target comes into focus. By measuring $S$ and $S'$, you can verify the lens equation.
\begin{enumerate}
\item Repeat this measurement of $S$ and $S'$ for four more initial positions of the lens, each time readjusting the screen until the new arrow is clear. Be sure to include uncertainties.
\item With Microsoft Excel, plot $1/S'$ vs. $1/S$, with error bars. Draw a best-fit line and determine the slope and intercept with the LINEST method. You should be able to determine the focal length $f$ from the intercept!
\item Is your value of $f$ consistent with your value from the previous measurements? Is the value of the slope of the line what it is supposed to be?
\item Comment on how well the line fits your data points.
\item Which of the two methods used so far should give you the best estimate for $f$? Explain why.
\item Discuss the main sources of error in determining $f$. How would you improve this experiment?
\end{enumerate}

\section{Applications}

\emph{Swimming pool experience}:\vspace{0.6\baselineskip}

The next time you go swimming, do the following experiment (safely!):\myskip

From underwater, look at people or trees outside the water. You will see that you can see them when you view with your eyes at small angles relative to the \underline{normal} to the water surface. But as you look at larger angles, you find that the water surface suddenly appears weird, somewhat like liquid mercury. This occurs because the water surface reflects like a mirror so that you are not able to observe anything outside the pool. This is a consequence of total internal reflection. (If you will not be swimming for a while, you can also check this by looking at the water-air surface of a fish tank from below.)\vspace{0.6\baselineskip}

\noindent\emph{Medical diagnoses}:\vspace{0.6\baselineskip}

A standard method of medical imaging utilizes ultrasonic sound echocardiography: ultrasonic sound waves produce images of inner body organs (along with many other applications). A wave of ultrasonic sound sent into the body is refracted and reflected as it passes through parts of the body of different density. The wave reflected from a specific organ is observed by a detector and produces an image. The manner in which the ultrasonic sound waves get reflected and refracted is similar to the case for light that we have considered.\myskip

An important difference between light and sound waves is that sound waves propagate faster in water than in air. (With light, it is the other way around!)  This means that, for sound, water has a lower refractive index than air! Therefore, total internal reflection occurs going from air to water.  To enable the sound waves to get into the body requires having no air between the source and the body, usually accomplished by putting a layer of gel between the sound head and the body. This also means you cannot send ultrasonic sound through air-filled organs (like the lungs).

The picture below shows an echocardiogram of a human heart:
\begin{figure}[h]
    \begin{center}
        \includegraphics[width=0.5\textwidth]{./Exp6/pic/image021.png}
    \end{center}
    \caption{Echocardiogram of a Human Heart}
    \label{fig:heartgram}
\end{figure}

LA = left atrium, LV = left ventricle, RA = right atrium, RV = right ventricle.\myskip

(Picture from: Marvin Berger: Doppler Echocardiography in Heart Disease.)\myskip

\noindent\emph{The human eye}:\vspace{0.6\baselineskip}

The most important application of lenses (to us) is the human eye! The retina is located a fixed distance from the lens. We need the ability to focus objects located at different distances in front of the eye onto the retina. These specifications require us to have an adjustable lens (with variable focal length). Adjusting the focal length is accomplished by deformation of the lens through contraction of the ciliary muscles. If the eye views a distant object, the muscles are relaxed and the lens is rather flat, with a long focal length. If the eye must examine a nearby object, the muscles contract and the lens becomes rounder, with a shorter focal length. With advancing age, the lens loses its flexibility so that the eye loses much of its ability to adapt to objects at close distances. (A common misperception is that this can be compensated by ``eye exercises'', which would be the case if the problem were muscles. But the problem is not in lost vigor of the muscles, but in decreased flexibility of the lens!)\myskip

\begin{figure}[!h]
\centering
\includegraphics[width=0.7\textwidth]{./Exp6/pic/image11.png}
\caption{Human Eye}
\end{figure}

The two most common optical defects are nearsightedness and farsightedness. In a nearsighted (myopic) eye, the focal length is too short even when the ciliary muscles are completely relaxed. Thus, parallel rays from a distant object come to focus in front of the retina and fail to form a sharp image on the retina. Vision of distant objects is blurred. Eyeglasses with diverging lenses correct this condition. \myskip

In a farsighted (hyperopic) eye, the focal length is excessively long, even when the ciliary muscles are fully contracted. Hence, rays from a nearby object converge toward an image beyond the retina and fail to form a sharp image on the retina. Eyeglasses with converging lenses can correct this condition.\myskip

Reference: see Physics, by Ohanian, for further information.\myskip

Picture from: Daniel Malacara: Geometrical and Instrumental Optics.

\section{Lab Preparation Problems}

\noindent \underline{Absorption}:\myskip

1. How many times does a ray of light get reflected on a parallel pair of mirrors (which reflect $99\%$ of the incoming light) such that the total intensity is down to $50\%$? \myskip

\noindent \underline{Reflection}: \myskip

2. Explain in your own words the difference between specular and diffuse reflection using the following system and observations: \myskip

On a day with no wind, the surface of the sea can be very smooth and shiny. On the other hand, the tops of the waves appear white in a strong storm. \myskip

\noindent \underline{Refractive index and Snell's Law}:\myskip

3. What is the refractive index if a medium can slow down light to $v = 2/3\,\mathrm{c}$? \myskip

4. If you measure an incoming ray from the air to be at an angle of 15 degrees to the normal, what angle will it have in the medium with an refractive index of $n = 1.2$? \myskip

5. If you measure $\theta_{i} = 25$ degrees (air) and $\theta_{t} = 20$ degrees, what is the refractive index of the medium in which you measured $\theta_{t}$?\myskip

6. Given the following pairs of values for $\theta_{i}$ and $\theta_{t}$, draw a graph of $\sin \theta_{t}$ vs. $\sin \theta_{i}$. Determine the value of n using only the slope of the graph and not the individual values from the table.
\begin{table}[h]
    \centering
    \begin{tabular}{|l|l|l|l|l|l|}
        \hline
        $i$ in degrees & 0 & 15 & 30 & 45 & 60 \\ \hline
        $r'$ in degrees & 0 & 11.5 & 22.6 & 33.0 & 41.8 \\ \hline
    \end{tabular}
\end{table}

7. Given the following measured values for the refractive index, what is the mean and uncertainty (using the 2/3 rule)?
\begin{table}[h]
    \centering
    \begin{tabular}{|l|l|l|l|l|l|l|}
        \hline
        $n=$ & 1.24 & 1.21 & 1.29 & 1.23 & 1.27 & 1.26 \\ \hline
    \end{tabular}
\end{table}

\noindent \underline{Total Internal Reflection and Critical Angle}: \myskip

8. Given an refractive index of diamond $n = 2.42$, what is its critical angle? \myskip

9. You want to measure the refractive index of water. For that you take a glass of water or go to your fish tank and look upwards at the air-water boundary through the side of your glass. When look at the boundary with a small enough angle, the surface will be (almost) $100\%$ reflective and you cannot see out through the boundary. With this setup, you measure the critical angle to be 60 degrees. What does this data give for the refractive index of water? Is it the same as the value you find in books?

% \chapter{Optics II: Applications of Lenses}
\section{Introduction}
In this experiment, we will apply our knowledge of lenses and geometrical optics to build a microscope and telescope. Microscopes are common in the medical field, and allow doctors to closely examine things that our eyes cannot resolve. Understanding the basic operation is important to knowing when they are useful, and how they can be improved. Telescopes allow a user to view a far away object as if it were closer, again allowing us to resolve details that the naked eye cannot. Both devices are useful in a wide variety of fields. Throughout this experiment, we will be using the `thin lens' approximation\footnote{Many approximations are made with ``thin lenses''. However, even the most sophisticated treatments of optics usually begin with these ``thin lens approximations''.}.

\section{Theory}
Why do we need optical instruments? The human eye is incredible in its ability to capture a wide range of colors and focus over a wide range of distances. However, there are some objects that are either too small, or too far away for the eye to resolve. Doctors and scientists require optical instruments to take improve our natural ability.

\subsection{The Near Point}
In the human eye, light is focused on the retina by the elastic lens. Though the lens can change its focal length (through contractions of the ciliary muscles), the range of focus is limited. The closest point at which the eye can get a sharp image of an object is called the near point. The near point is different for every person, but the value of $25\, \textrm{cm}$ is an average over a random segment of the population. We use the standard reference of $25\, \textrm{cm}$ here. Young people with good eyesight may have a near point as close as $10-15\, \textrm{cm}$, and very young children have an even smaller one. You will determine the near point for your eyes in the course of this experiment.

\subsection{Resolution}
In art museums, you probably observed the following effect. As you look at an impressionist painting from a distance, the picture appears sharp and easily identifiable. But as you approach the picture, you observe that it is made up of a number of smaller objects, large dots or blobs, and is quite coarse-grained.\myskip

What happened? Your retina consists of an array of light receptors, somewhat like the pixels in a video camera. When you are far enough away from the picture, all light rays from an object like a big dot fall onto a single receptor. Your visual impression from a long distance therefore consists of various sharp points. As you approach the picture, rays from this object begin to strike several receptors and you recognize that the blob is an extended object. If you look at the same picture, but shrunk by a factor of 10, you observe that you can get much closer before seeing the individual blobs. Such effects are related to the resolution limits of the eye.\myskip

The figure on the next page illustrates how an image appears on the retina. Note that the image is inverted on the retina. We actually �see� objects upside down. (Our brains do the work of inverting it back!) When we view an object, its size is determined by how big the object�s image is on our retina. \myskip

Figure \ref{fig:resolution} below, we see that the size of the retinal image is directly proportional to $\tan\alpha=h/s$. Clearly, there is a minimum value of $\alpha$ that can be resolved on the retina.
\begin{figure}[h]
\centering
\includegraphics[width=0.8\textwidth]{./Exp7/pic/image1.png}
\caption{The Formation of Images on the Retina}
\label{fig:resolution}
\end{figure}


\subsection{Need for Magnification}

Suppose a person with a near point of $25\, \textrm{cm}$ examines the object shown in the figure above. To see the most detail, the person should view the object so that it appears to be as large as possible while remaining in focus. As discussed above, the size of the retinal image varies as $\tan\alpha=h/s$. This means that the retinal image gets larger as s gets smaller (i.e., as the object is brought closer to the eye). The largest focused image this person can see is obtained when $s = 25\, \textrm{cm}$. The size of the object when viewed at $25\, \textrm{cm}$ is \underline{defined} to have a magnification of one. Is there any way to see more detail on the object, or, equivalently, can the object be made to appear larger? The answer is yes, and the simplest means of making the object appear larger is to use a magnifying glass.

\subsection{Virtual Images vs. Real Images}
\label{sec:vimages}

In the lens experiments we have been covering so far, it has been possible to place a screen at the location of the image and since the light rays focus at that screen, a clear image is formed on the screen. Those images are real. However, consider a ray diagram with a lens, where the object is placed between the focal point and the lens, as in Figure \ref{fig:virtualimage}.

\begin{figure}[h]
\centering
\includegraphics[width=0.7\textwidth]{./Exp7/pic/image5.png}
\caption{Convergent Lens: Virtual Image}
\label{fig:virtualimage}
\end{figure}

When the object is placed exactly at the focal point, the rays exit parallel (and the image is infinitely far away). (Remember that this is how we define the focal point, and how we determined the focal length in Experiment 2-6)  As the object gets even closer to the lens, as shown in Figure \ref{fig:virtualimage}, the rays must diverge as they exit. If you extend the outgoing rays backward (behind the lens), you find that they intersect behind the original object. A virtual image is formed. It cannot be captured on a screen, but you can see it as you look through the lens - the object appears at the position of the virtual image and enlarged.

This same phenomena happens when you look into a mirror and see a clear reflection of yourself. The rays originating from your body never really converge, so it is not possible to see the image on a screen.

\subsection{Parallax}

In everyday life, we are able to tell distance by a combination of seeing how large things are relative to other objects, and how they move relative to each other. But what happens when we are not sure what the size of an object is supposed to be? In optics experiments, both virtual and real images can be magnified in non-intuitive ways, so using the size of the image we see compared to the original object is not a good way to tell which object is closer or further. We must use the method of parallax. Parallax is the apparent change in distance between to objects that depends on where the observer is.\myskip

Imagine walking along a road and seeing two trees in the distance, as in Figure \ref{fig:parallax}. At point A in the left image, you see an apparent distance between the two trees given by L$_A$. But as you walk to point B, you would notice that that distance changes dramatically, and the trees only look lined up when you are standing directly in front of them. However, if the trees are in nearly the same location, the apparent change between  L$_A$ and L$_B$ is very small as you walk along the road.\myskip

\begin{figure}[h]
\centering
\includegraphics[width=0.7\textwidth]{./Exp7/pic/image_parallax.png}
\caption{An example of parallax. In the image on the left, the difference in observed distance between the trees changes dramatically as you move along the road, hence the large difference between L$_A$ and L$_B$. In the image on the right, where the trees are nearly in the same location,  L$_A$ and L$_B$ are basically unchanged.}
\label{fig:parallax}
\end{figure}


\subsection{The Magnifying Glass}

A magnifying glass is a simple converging lens with short focal length that can make an object's image on the retina larger. How does it work? One limitation to the detail you see on an object is the limited angular resolution, which is due to the limited size of the individual receptors in your eye. (There is nothing we can do about this.) But the other limitation is that we cannot make the retinal image larger by bringing the object closer than the near point. There is a way to overcome this limitation, and it is used in all optical magnifying devices.\myskip

Consider the object and lens of focal length $f$ shown in Figure \ref{fig:virtualinf}.\myskip
\begin{figure}[h]
\centering
\includegraphics[width=0.8\textwidth]{./Exp7/pic/image2.png}
\caption{Virtual Image at infinity}
\label{fig:virtualinf}
\end{figure}

The object is placed a focal length $f$ away from the lens. Using the lens equation, we find that a virtual image is formed at infinity as shown. (If you feel uncomfortable thinking of an image at infinity, you might prefer to think of it as simply a very large distance.)\myskip

If the rays of light from the ends of the object that pass through the center of the lens are extended to the left, they also pass through the ends of the image. (If this is not obvious, return to Experiment 2-6 and review the rules for ray tracing). This means that the angle $\alpha'=\tan^{-1}h/f$ is the same for both the object and the virtual image. Remember what an image represents - looking at the object through the lens can be represented as looking directly at the image. Since the image is very far away, its size does not appear to change as the eye (to the right of the lens) is moved small distances back and forth. \myskip

What actually happened when you put the converging lens one focal length in front of the object? The rays leaving the lens that actually hit the eye (not the ray shown in the figure) will be nearly parallel and therefore the rays appear to come from an infinitely far object.\myskip

When we view the image, its size is proportional to $\tan\alpha'=h/f$. The \emph{angular magnification} of the magnifying glass is defined as:
\begin{equation}
  M=\frac{\tan\alpha'}{\tan\alpha}=\frac{h/f}{h/25\ \mathrm{cm}}=\frac{25\ \mathrm{cm}}{f\ \mathrm{cm}}
\end{equation}

The \textbf{magnification $M$} represents the ratio of the \underline{apparent} size of the object when viewed using the magnifying glass to that of the object when viewed directly at a distance of $25\, \textrm{cm}$.\myskip

If $f$ is less than $25\, \textrm{cm}$, the lens increases the apparent size of the object and permits you to see more detail. Viewing an object with a magnifying glass of focal length $12.5\, \textrm{cm}$ produces a retinal image twice as large as that formed when the object is viewed directly $25\, \textrm{cm}$ away.\myskip

If a lens of focal length $25\, \textrm{cm}$ is used as a magnifying glass, the apparent size of the object is the \underline{same} as that seen when looking directly at the object $25\, \textrm{cm}$ from the eye. Using the lens, however, permits your eye to relax because it views parallel rays, just as if the object were located very far away.\myskip

In practice, one does not always place the object at the focal point of the magnifying glass, but it is at that point that the magnification is greatest. Magnification can also be equivalently defined as the ratio of the height of the image to that of the object. By drawing similar triangles, it should be clear that this implies that
$$
M= \frac{h_i}{h_o} = \frac{-s_i}{s_o}
$$

Note that we follow the sign conventions used in the course textbook here--you may wish to review Section 34-7 of Fundamentals of Physics or the discussion in Experiment 2-6.

\subsection{The Microscope}
\label{sec:theorym}
For various reasons, a single lens magnifier (such as a magnifying glass) can only provide good images with relatively small magnifications. To achieve higher magnification, a device known as a compound microscope can be used. \myskip

A compound microscope employs two lenses as shown in the next figure. The object to be viewed is placed at a distance $s_{o1}$, just beyond the focal length, $F$, of the first lens, called the objective lens. The light collected by the objective lens forms a real (enlarged) image of the object at $s_{i1}$. (The image is real because a screen placed at $s_{i1}$ would display an actual inverted image of the object.) The second lens, the eyepiece, has a focal length $f$ and acts as a magnifying glass that is used to examine the image $s_{i1}$. The position of the image formed by the first lens becomes the object for the second lens, and the distance $s_{o2}$ is defined as in Fig. \ref{fig:micro1}. Adjusting the position of the eyepiece adjusts the position of the final image. \myskip

As discussed in the preceding section, the observer looking through the eyepiece (magnifier) sees a virtual image $s_{i2}$ of the real image.
\begin{figure}[h]
\centering
\includegraphics[width=0.8\textwidth]{./Exp7/pic/image9a.png}
\caption{A Compound Microscope}
\label{fig:micro1}
\end{figure}

How big does the original object appear? Since magnification represents a multiplicative factor, we simply have to calculate the magnification of the original image $I_1$, by the first lens, then multiply that by the magnification of the second lens. \myskip

The total magnification of the microscope is then:
\begin{equation}
	M=M_{eyepiece}M_{objective} =  \left(\frac{-s_{i1}}{s_{o1}}\right) \left(\frac{-s_{i2}}{s_{o2}}\right)
\end{equation}

\subsection{The Telescope}
\label{sec:telescope}
The optical system of a refracting telescope has the same elements as those of a compound microscope. In both instruments, the real image formed by an \underline{objective} lens is viewed through an \underline{eyepiece}. However, the object of a telescope is distant, the incoming rays are (approximately) parallel and the objective lens forms the first image $I_1$, near its focal point $F$, as shown in the figure below. (By contrast, for the microscope $s'\gg F$.)\myskip
\begin{figure}[h]
\centering
\includegraphics[width=0.8\textwidth]{./Exp7/pic/image4.png}
\caption{Telescope}
\end{figure}

\textbf{The important point here is that although the object is very far away, it still subtends a small angle in our naked eye (otherwise we wouldn't be able to see it)}. The assumption that it is very far away is to make sure that there will be an image $I_1$ at the focal point of the objective lens, which again serves as the object for the eyepiece. In order for $I_2$ -- the final virtual image made by the eyepiece -- to be near infinity (for relaxed viewing), the eyepiece needs to be located so that $I_1$ is also near its focal point $f$. Therefore, the distance between the objective lens and the eyepiece -- the length of the telescope -- is the sum of the focal lengths $F+f$.\myskip

The angular magnification of the telescope is easily derived by noting that $\alpha$, the very small angle that the distant object would subtend for the \underline{unaided eye}, is essentially the same as the angle subtended by the image $I_1$ from the objective lens. Then from the figure above we have:
\begin{equation}
  \tan a=\frac{y}{F}
\end{equation}
and angle $A$, subtended by $I_1$ at the eyepiece, we have:
\begin{equation}
  \tan A =\frac{y}{f}
\end{equation}
so:
\begin{align}
  M_{\mathrm{telescope}}&=\frac{\tan A}{\tan a}=\frac{y/f}{y/F}\\
  M_{\mathrm{telescope}}&=\frac{F}{f}
\end{align}


\section{Procedure}

\subsection{The Near Point}
In the first part of the lab, you will measure the near point of each eye. Tape the paper grid pattern onto the screen and place the screen on the optical bench. Place your head so that one eye is at the end of the  optical bench. Move the screen as close to your eye as possible, consistent with the requirement that the screen remains in focus. The distance s is the near point for that eye. Repeat for the other eye.
\begin{enumerate}
\item Is the near point the same for both eyes?
\item How do these values for your eyes compare to the standard value of $25\, \textrm{cm}$?
\end{enumerate}

If the values you obtain are different from the conventional $25\, \textrm{cm}$ value, it should not be surprising; as you compare your values other people in the class, you will probably see a wide variation. So as to make quantitative comparisons of the instruments that we will be using in the following sections, we use a standardized reference near point of $25\, \textrm{cm}$. We choose this value since the �average� person cannot get closer to an object than this and still keep it in focus.

\subsection{Virtual Images}

In this part of the lab, we will study virtual images formed by concave lenses. Concave lenses have negative focal lengths, and are therefore sometimes referred to as ``diverging lenses''. You will then use
another lens to form a real image of the virtual image. In this way you can identify the
location of the virtual image.\myskip

Set up the -150 mm lens and light source as in Fig. \ref{virt1}, with the crossed arrow object on the light source pointing towards the lens.\myskip

\begin{figure}[h]
\centering
\includegraphics[width=0.7\textwidth]{./Exp7/pic/virtual-1.png}
\caption{Virtual Image set-up}
\label{virt1}
\end{figure}

\begin{enumerate}
\item Look through the lens toward the light source. Describe the image. Is it upright or inverted? Does it appear to be larger or smaller than the
object?

\item Which do you think is closer to the lens: the image or the object? Why do you think so?

\item Using the lens equation (see Experiment 2-6, and also on the next page) and remembering that the concave lens has a negative focal length, calculate the position of the virtual image.

\noindent Place the +200 mm lens anywhere between 50 cm and 80 cm on the optical bench and then move the screen so that the image is focused (Fig. \ref{virt2}).
\begin{figure}[h]
\centering
\includegraphics[width=0.7\textwidth]{./Exp7/pic/virtual-2.png}
\caption{Placing a second lens to form a real image from the virtual one.}
\label{virt2}
\end{figure}

\item Why are you able to see the image on the screen? What is serving as the object?

\noindent Now, you will discover the location of the virtual image by replacing it with the light source. Remove the negative lens from the bench, and note that the image on the screen goes out of focus. Slide the light source to a new position so that a clear image is formed on the screen. (Do not move the positive lens or the screen.) Make sure you write down the positions of all components. (See Figure \ref{virtu3})

\begin{figure}[h]
\centering
\includegraphics[width=0.7\textwidth]{./Exp7/pic/virtual-3.png}
\caption{Remove the first lens, then move the light source to the position of where the virtual image was.}
\label{virtu3}
\end{figure}

\item Using the new location of your light source, what was the position of the virtual image?

\item Does your measurement agree with the calculation using the lens equation? Note that since our lenses aren't ideal thin lenses, there will be some small  difference from your prediction.

\end{enumerate}

\subsection{The Microscope}
\myskip
\begin{figure}[h]
\centering
\includegraphics[width=0.8\textwidth]{./Exp7/pic/image9a.png}
\caption{Measuring the Magnification of the Microscope}
\label{fig:microscope}
\end{figure}

A microscope magnifies an object that is close to the objective lens. The microscope in this experiment will form an image in the same place as the object (see Figure \ref{fig:microscope}).\myskip

As discussed earlier, the magnification, M, of a two-lens system is equal to the product of the magnifications
of the individual lenses. Magnification is defined to be the negative of the ratio of the image distance to the object distance, as in the lecture course textbook. In a microscope, the eyepiece uses the image of the objective lens as its object. Therefore, the total magnification of the microscope can be given as:

\begin{equation}
\label{eq:magnification}
	M=M_{eyepiece}M_{objective} =  \left(\frac{-s_{i1}}{s_{o1}}\right) \left(\frac{-s_{i2}}{s_{o2}}\right)
\end{equation}

In this experiment, we would like to achieve a magnification of \textbf{-2} (be careful with sign conventions in the equation above).\myskip

First, tape the paper grid pattern to the screen to serve as an object. Then, use the 100mm lens as the objective lens and the 200mm lens as the eyepiece, as shown in Fig. \ref{fig:microscope}.

The goal here is to create a real image with the objective lens, and then calculate where to place the eyepiece so that the final virtual image is at the same position as the original object.

Set up the Objective lens (f=100 mm) 30 cm away from the screen ($s_{o1}$). Given this, and the thin lens equation:
$$
\frac{1}{f}=\frac{1}{s_{o}}+\frac{1}{s_i},
$$
you can calculate where the image ($s_{i1}$)from the objective should show up. In order to measure the magnification by eye, we need to place the eyepiece so that the virtual image is formed exactly where the original object is. If we do this, then we can accurately compare the magnified grid to the unmagnified one.\myskip

Combine the two lens equations for each lens into Equation (\ref{eq:magnification}) to eliminate the unknown variables $s_{i1}$ and $s_{i2}$, then solve for $s_{o2}$. You should arrive at the following equation (please include a derivation; don't just copy the results).
\begin{equation}
  s_{o2} = \frac{f_1 f_2}{s_{o1} - f_1} \frac{1}{M} + f_2
\end{equation}

Place the eyepiece at the calculated distance and look through your microscope. As you move your head from side to side, you should notice that there is little to no parallax between the lines on the actual object and the lines that you see through the center of the microscope.

\begin{enumerate}
\item Estimate the magnification by comparing the magnified and unmagnified grids. Quote an error and compare this to the magnification you attempted to get.

\item Is the image inverted or upright? How do you explain this?

\item Carefully draw a ray diagram of the experiment and indicate the expected type (real or virtual) of each image.

\item Create and fill out a table like Table \ref{tab:microscope}.

\end{enumerate}
\begin{center}
\begin{tabular}{|l|l|}
\hline
Position of Objective Lens & \hspace{4 cm} \\ \hline
Position of Eyepiece Lens &                    \\ \hline
Position of Screen &                    \\ \hline
$s_{o1}$ & \\ \hline
$s_{i1}$ & \\ \hline
$s_{o2}$ & \\ \hline
$s_{i2}$ & \\ \hline
Observed Magnification & \\ \hline
\label{tab:microscope}
\end{tabular}
\end{center}

\subsection{The Telescope}
As we have seen before, the telescope works on a similar principle to the microscope, except now, our object distance is effectively infinite.
\myskip
\begin{figure}[h]
\centering
\includegraphics[width=0.8\textwidth]{./Exp7/pic/image10a.png}
\caption{Schematic of a telescope.}
\end{figure}

We assemble a telescope and measure its magnification. Since the final object is formed near infinity, we can compare far away objects with a magnified image, without worrying too much about where exactly the image is formed. Comparison of these permits an estimate of the magnification.\myskip

Using the same 200 mm and 100 mm lenses from before, construct a telescope using your knowledge from reading previous sections. Think carefully about which lens should be the objective and which lens should be the eyepiece.\myskip

Use your telescope to look at a grid that is placed far away (You may have to place your screen off the optical bench to make the ``object at infinity'' condition valid). Once you have found a clear image looking through the telescope, open your other eye so that you can look simultaneously at the enlarged image with one eye and the unenlarged image with the other eye. If you position the telescope just right, you can see the two grids side by side and estimate the magnification M of your telescope.

\begin{enumerate}
  \item Record your estimate of the telescope magnification by eye with error. Describe how you quantitatively determined the error in magnification.

  \item Calculate the magnification for your telescope. Does your measured M agree with your theoretical value?

  \item Discuss the main sources of error in measuring the magnification.

  \item Is the image inverted? Explain this.

  \item Carefully draw a ray diagram of the experiment and indicate the expected type (real or virtual) of each image.

  \item Create and fill out a table like \ref{tab:telescope}.

\end{enumerate}

\begin{center}
\begin{tabular}{|l|l|}
\hline
Position of Objective Lens & \hspace{4 cm} \\ \hline
Position of Eyepiece Lens &                    \\ \hline
Position of Screen &                    \\ \hline
$s_{o1}$ & \\ \hline
$s_{i1}$ & \\ \hline
$s_{o2}$ & \\ \hline
$s_{i2}$ & \\ \hline
Observed Magnification & \\ \hline
\label{tab:telescope}
\end{tabular}
\end{center}

\section{Applications}
It may seem strange that we provide yet another example based on the human eye, since you already had that in the last experiment. As with many real-life applications, you are only told part of the story at first (with the complexities left to later). The eye should better be described as a system consisting of two lenses.\myskip

The index of refraction of the fluid within the eye is about 1.34 (nearly the same as water). The index of refraction of the solid lens is about 1.44. The difference in the indices of refraction is not large, so the refraction at the interface between them is not strong. Therefore, though the lens provides adjustments in focal length necessary to form the image on the retina at different object distances, the main contribution of focusing the incoming light is refraction at the cornea. This tissue layer separates the eye fluid (or humor) from the air. The difference in indices of refraction between air (1.00) and the aqueous humor (1.34) is much larger and creates substantial refraction. This means that the eye is more precisely treated as a system of two lenses, in which the second lens is variable so as to fine tune the focal length of the system.\myskip

Several different optical malfunctions (e.g. farsightedness/nearsightedness) were described in the application part of the previous experiment. These can also arise from malformations of the eyeball. As described before, using an additional lens either in the form of glasses or contact lenses can compensate these. \myskip

But a more recent approach is to change the focal length of the first lens (cornea) directly. A high intensity laser beam can be used to evaporate thin layers of the cornea and thus provide the cornea with a new shape. If the curvature of the cornea is increased, the light is focused more (corrects farsightedness) or if you decrease the curvature, the light is focused less (corrects nearsightedness).

\section{Lab Preparation Problems}
\noindent\underline{Magnification and Magnifying Glass}:\myskip

1. The angle subtended by an image with an optical device is $\alpha'= 10^\circ$ and without it, the object subtends $\alpha= 7^\circ$. What is the magnification?\myskip

2. What is the magnification of a magnifying glass with a focal length of $10\, \textrm{cm}$?\myskip

3. Assume your eyes have a focal length of $25\, \textrm{cm}$. What is the advantage for your eyes if you still use a magnifying glass of focal length $25\, \textrm{cm}$ to read a book instead of holding the book at a distance of $25\, \textrm{cm}$?\myskip

\noindent\underline{Microscope}:\myskip

4. Assume you have a microscope and measure $s' = 30\, \textrm{cm}$ and $s = 10\, \textrm{cm}$. Your second lens has a focal length of $15\, \textrm{cm}$. What is the magnification?\myskip

5. Assume your microscope has a focal length of $F = 10\, \textrm{cm}$ for the first lens and $f = 10\, \textrm{cm}$ for the second lens. What is the magnification if you choose $s = 15\, \textrm{cm}$? What if you choose $s = 12\, \textrm{cm}$?\myskip

6. You build a microscope with a first lens of $F = 5\, \textrm{cm}$ and a second lens with $f = 10\, \textrm{cm}$. You separate these two lenses by a distance of $25\, \textrm{cm}$. What will the magnification of your microscope be? \myskip

Hint: You first have to calculate $s'$ and then $s$!\myskip

\noindent\underline{Telescope}:\myskip

7. You build a telescope with two lenses. One has a focal length of $5\, \textrm{cm}$, the other has a focal length of $2\, \textrm{m}$. In what order do you want to place these two lenses. Which should be the objective lens, which the eyepiece? \myskip

8. For the telescope described in 7, what is the magnification?\myskip

9. You want to build a telescope that can magnify far away objects in the following way. Everything that appears to be separated at an angle of $1'$ (1/60 of a degree) when viewed without the telescope will appear to be separated by $1^\circ$ when viewed through the telescope. You have a $5\, \textrm{cm}$ lens for the eyepiece. How do you have to choose the objective lens such that the magnification is as described? How long will your telescope then be? (The length of your telescope is the separation of the two lenses.)
 %Optics2 - Totally restored
% \chapter{Polarization and Interference}
\section{Introduction}
You have already investigated some wave properties last semester, so here we will expand upon your knowledge of waves and address three new concepts: polarization, interference and diffraction. Our experiment will be conducted with light waves, however the general concepts apply to many other waves as well.

\section{Theory}
\subsection{Polarization}
Polarization occurs only in transverse waves\footnote{Examples of transverse waves are oscillations on a string, water waves, or the electromagnetic radiation we treat here. Sometimes you may also hear the term polarization used in the context of transverse- and longitudinal polarization.}. By definition, the oscillation is perpendicular to the direction of wave propagation in a transverse wave. Since in three-dimensional space there are three independent dimensions, there are two possible perpendicular directions needed to describe any oscillatory motion perpendicular to the direction of motion. If you think about a transverse wave in a string, the oscillation could be up-and-down or left-and-right, while the wave front propagates forward along the string. The two independent and perpendicular directions are called the two polarization states. (Or in coordinates, if a wave propagates along the $z$-axis, there are two polarizations perpendicular to the wave motion--one along the $x$-axis and one along the $y$-axis.) Polarization cannot occur in longitudinal waves because the direction of oscillation has only one possibility, along the direction of motion.\myskip

You may argue that you can make oscillations happen on a string in more than just the two directions described above--and you are right! For example, you can make oscillations diagonally from the upper left to the lower right or you can even make circular oscillations by moving your hand continuously in a circle, producing a corkscrew pattern along the string\footnote{This wave is sometimes called left- or right- circular polarization.}. But no matter which wave you make, the oscillations can always be represented as two superimposed waves with displacements along the two mutually perpendicular polarization axes. Any vector can always be described by decomposing it into its $x$- and $y$- components. Here, the relevant vector is the oscillating displacement from the $z$-axis. The amplitudes of the resolved $x$ and $y$ components indicate how much of the wave is in each polarization state.\myskip

For electromagnetic radiation, the oscillating quantities are the electric (and magnetic) fields. These undulate perpendicular to the direction of the wave. Like waves in a string, electromagnetic waves are transverse waves and have polarization properties. The polarization state describes the axis along which the electric field in the wave oscillates. Visible light is electromagnetic radiation in a particular frequency-wavelength regime, with large frequency of oscillation ($\sim 10^{15}\,\mathrm{Hz}$) and small wavelength ($\sim 0.5\times 10^{-6}\, \mathrm{m}$).\myskip

\subsection{Polarizer}
Usually light is produced (e.g. incandescent light from a candle or a light bulb) by hot electrons and atoms, which are moving or are oriented in random directions, so that the wave is \textbf{unpolarized}. This means that the electric field associated with the wave oscillates in random directions (though always at right angles to the direction of the propagation of the light). So there is no favored direction for polarization. It is as if you were moving your hand in random directions at the end of a string. Note that this is very different from the cases of waves in a string described above. Even in circular polarization the oscillations were not random.
\begin{figure}[h]
\centering
\includegraphics[width=0.5\textwidth]{./Exp8/pic/image1.png}
\caption{Polarization Filter and Creation of Polarized Light}
\label{fig:creation}
\end{figure}

A polarization filter, shown in Figure {\ref{fig:creation}}, is a device with a specified direction, called the polarization axis. The filter absorbs all the wave's incoming electric field perpendicular to the polarization axis and transmits the electric field parallel to the polarization axis. So no matter what the polarization of the light shining into a polarizer might be, the light coming out of the polarizer is always polarized in the direction specified by the polarization axis\footnote{Such light is called linearly polarized light.}. An unpolarized wave just prior to the polarizer has an electric field ($\mathbf{E}$) pointing randomly in all directions perpendicular to the wave direction, as shown in Figure {\ref{fig:detection}}. The polarized wave after the polarizer has only half the energy content, and the $\mathbf{E}$-field is oscillating solely along the polarization axis. This electric field has the same value as the component along that axis just before the polarizer.
\begin{figure}[h]
\centering
\includegraphics[width=0.8\textwidth]{./Exp8/pic/image2.png}
\caption{Detection of Polarized Light}
\label{fig:detection}
\end{figure}

\subsection{Interference--The Double Slit}
As you've learned in class, Young's double slit experiment is a demonstration of interference and diffraction. A double slit, as illustrated in Figure {\ref{fig:double}}, permits what remains of an incoming wave (from the left) to travel to a distant screen (to the right) along two different paths with different lengths. The light waves from the two slits interfere, resulting in an interference pattern of bright (constructive) and dim (destructive) patches as viewed on the screen. Let's see how the double slit works using Figure {\ref{fig:double}}.
\begin{figure}[h]
\centering
\includegraphics[width=0.8\textwidth]{./Exp8/pic/image3.png}
\caption{Young's Double Slit Experiment}
\label{fig:double}
\end{figure}

We shine light from the left onto the double slit, which allows two light waves to propagate from the two slits. Straight-ahead they will always be in phase since they travel the same distance to the screen. But when the two waves propagate at an angle $\theta$, they cover different distances to reach a specific point on the screen. (We assume that the two rays are parallel, a good approximation since the screen is effectively infinitely far away compared to the distance between the two slits.)\myskip
\begin{figure}[h]
\centering
\includegraphics[width=0.3\textwidth]{./Exp8/pic/image4.png}
\caption{Geometry of the Double Slit}
\label{fig:geometry}
\end{figure}

As you can convince yourself by applying the rules of trigonometry to the triangle in Figure {\ref{fig:geometry}}, the difference in path ($\Delta l$) between the two rays is, for slit separation, $d$,
\begin{equation}
 \Delta l =d\sin\theta
\end{equation}

So what quantitatively determines a point of maximum intensity on the screen? Maximum intensity occurs if the two waves are precisely in phase at that point on the screen. This follows from what you learned about constructive and destructive interference in the lab on standing waves. As you saw, two waves can add up constructively or destructively. In the case of maximal constructive interference, the waves combine at the screen so that a crest on one wave coincides with a crest on the other wave and a trough coincides with a trough. In the case of maximal destructive interference, the waves combine so that a crest on one wave coincides with a trough on the other wave. \myskip

The waves will be in phase when they meet at the screen if the path difference between the two waves is an exact number of wavelengths, $m$, where $m$ is an integer called the order of the maximum. This requirement, expressed mathematically, is
\begin{equation}
 \Delta l =m\lambda
\end{equation}

Note that if $m=0$, there is no difference in the path lengths, resulting in a maximum straight ahead, $\theta=0$, as shown in Figure {\ref{fig:interference}}. Subsequent maxima occur for subsequent positive and negative integers, $m=\pm 1, \pm 2, \pm 3, \dots$. \myskip

Combining the two equations for $\Delta l$, we get for the specific angles, $\theta_{m}$ , that give maximal constructive interference:
\begin{equation}
 d\sin \theta_{m}=m\lambda
\end{equation}

If we know the order $m$ of a particular maximum, we can determine the wavelength of the light incident on the double slit by measuring the angle of the maximum on the screen. Conversely, we can determine the order of a particular maximum, $m$, from the wavelength of the light and the angle.
\begin{figure}[h]
\centering
\includegraphics[width=0.6\textwidth]{./Exp8/pic/image5.png}
\caption{Interference of Light Coming from Two Small Slits}
\label{fig:interference}
\end{figure}

\subsection{Intensity of the Double-Slit Maxima}
You will soon note that the intensities for subsequent maxima are different. What causes this? Just as the double slit creates an interference pattern, each single slit creates a diffraction pattern due to interference from waves coming from different parts of a single slit as described in section 39-6 of the lecture course text\footnote{Sears, Zemansky, and Young, \textbf{College Physics \engordnumber{7} edition}, Addison-Wesley, p. 903-906.}. This occurs because each individual slit has a finite width. The overall (more realistic) pattern shown in Figure {\ref{fig:intensity}} is a superposition of the single slit and double slit interference patterns. The narrow spikes (of interest for this experiment) are due to the double slit interference and the envelope, or large-scale pattern, is due to the single slit diffraction caused by each slit.
\begin{figure}[h]
\centering
\includegraphics[width=0.8\textwidth]{./Exp8/pic/image6.png}
\caption{Intensity Pattern of the Double Slit. The Many Peaks of the Narrow Spikes Correspond to the Maxima in $d\sin\theta_{m}=m\lambda$}
\label{fig:intensity}
\end{figure}

\subsection{Light from a Laser}
In the double slit experiment we use a laser as the light source. Laser light differs from incandescent or fluorescent light in two important ways. First, lasers emit intense light of a single wavelength or frequency (monochromatic). Second, laser light is coherent, meaning that all light waves are in phase.\myskip

Let's illustrate the differences between coherent and incoherent light through a simple familiar example. Contrast a crowd of people cramming the city streets during rush hour with that of a marching band at a parade. In the midtown crowd, even if everybody were to step at the same frequency (which they don't!), each pedestrian's steps are independent of what others are doing. But the marchers at a parade move in step, or in phase, with a similar frequency, taking their steps at exactly the same time. Laser light is like the (coherent) marching band at a parade, whereas light from a light bulb is like the (incoherent) pedestrian crowd at Times Square.\myskip

Although incandescent light sources\footnote{This is easily visible in the colors produced by soap bubbles.} can produce diffraction and interference patterns, a laser is better suited to illustrate coherent phenomena.

\section{Experiment}
\subsection{Polarization}
Suppose a wave has linearly polarized light characterized by an electric field vector, $\mathbf{E}_{0}$, at an angle, $\phi$ , relative to the direction of a polarizer. The electric field that is transmitted through the polarizer is $\mathbf{E}=\mathbf{E}_{0}\cos\phi$, the component that was not absorbed. Since the energy intensity in the wave is proportional to $\mathbf{E}^2$, then the intensity passing through two polarizers will be proportional to the square of the cosine of the angle φ between the two polarizing axes.\footnote{This relation is sometimes referred to as Malus' Law, after Etienne Malus, who also discovered the polarization of reflected light in the early nineteenth century.}
\begin{equation}
 I=I_{\mathrm{max}}\cos^2\phi
\label{eq:i}
\end{equation}

In this experiment, we use an incandescent unpolarized light source and then polarize it by first passing it through a polarizer. The electric field vector $\mathbf{E}_{0}$ will then point along the same direction as the polarizing axis of the first polarizer. As the now polarized light falls on the second polarizer, only the component of $\mathbf{E}_{0}$ along the polarizing axis of the second polarizer will make it through unaffected. \myskip

As Figure {\ref{fig:polarized}} illustrates, an electric field strength of only $\mathbf{E}_{0}\cos\phi$ will make it through the second polarizer, so that the new intensity will be given by Equation {\ref{eq:i}} above. A human eye or a photometer can detect this change in light intensity.
\begin{figure}[h]
\centering
\includegraphics[width=0.8\textwidth]{./Exp8/pic/image7.png}
\caption{Creation and Detection of Polarized Light}
\label{fig:polarized}
\end{figure}

\subsection{Double Slit}
In this part of the experiment we use a laser instead of incandescent light source. Laser light shines onto a double slit, which produces an interference pattern of bright and dark lines on the wall or a piece of paper beyond the double slit. \myskip

We know that the angle, $\theta_{m}$, at which the $m^{\mathrm{th}}$ order bright spot appears is given by
\begin{equation}
 d\sin\theta_{m}=m\lambda
\end{equation}

For small $\theta_{m}$ (in radians), the approximation $\sin\theta \approx \tan\theta \approx \theta$ is valid\footnote{See for yourself by punching $\sin 0.1$ and $\tan 0.1$ into your calculator (using radiant measure)!}. So on a screen far away from the double slit, we observe maxima located close to the optical axis (where the $0^{\mathrm{th}}$ order maximum hits the screen) approximately given by
\begin{equation}
\sin\theta_{m}\approx\tan\theta_{m}=\frac{x_{m}}{L}
\end{equation}
where $x_m$ is the distance between the $m^{\mathrm{th}}$ order maximum and the $0^{\mathrm{th}}$ order maximum, and $L$ is the distance between the double slit and the screen. We then find
\begin{equation}
\frac{x_{m}}{L}=m\frac{\lambda}{d}
\end{equation}
or for adjacent maxima
\begin{equation}
 \frac{\Delta x}{L}=\frac{\lambda}{d}
\end{equation}
with $\Delta x$ equal to the distance between the maxima. Since $\Delta x$ and $L$ are easy to measure, we can straightforwardly determine $d$ (or $\lambda$ ) given $\lambda$ (or $d$).

\section{Specifics of the Experiment}
\underline{SAFETY NOTE:}

\emph{Although the laser is of low intensity for most situations, it can be dangerous in certain circumstances unless used carefully. In particular, \textbf{do not} use the laser in such a way that it can shine into any person's eye (The warning label on the laser states ``Do not stare into Beam!''). When you are not actually using the laser, turn it off or close the beam shutter at the front of the laser.}
\begin{figure}[h]
\centering
\includegraphics[width=0.8\textwidth]{./Exp8/pic/image8.png}
\caption{Equipment Components}
\label{fig:setup}
\end{figure}

\section{Equipment}
\indent\indent\underline{Linear Translator:}
A holder for the fiberoptic cable that can be moved transversely in small increments, allowing a positioning with better than $0.1\,\mathrm{mm}$ precision. \myskip

\underline{Fiberoptic Cable:}
Cable made from glass, quartz, plastic or a similar material, which carries light waves. The cable uses total internal reflection at the walls to provide very high efficiency (or low losses) between the cable ends.\myskip

\underline{Photometer:}
Device to measure the total power contained in light impinging on it. It uses the photoelectric effect to convert the photons to electrons and various electron amplification techniques to produce a measurable electric current.

\subsection{Polarization}
\begin{enumerate}
\item Make sure only the dim incandescent ceiling lights in the room are on.

\item The linear translator should be placed at the far end of the track opposite the laser. Make sure that the linear translator is in the middle position (at $2.5\,\mathrm{cm}$).

\item Place the fiber-optic cable in the hole of the linear translator.

\item Place the incandescent light source about $10$-$20\,\mathrm{cm}$ from the linear translator such that the light shines towards the linear translator.

\item Place the polarizers (numbers 18 and 19) in the holders.

\item Rotate each polarizer such that the white mark on the holder aligns with the $0^{\circ}$ angle reading on the polarizer.

\item Place the two polarizers between the incandescent light source and the linear translator.

\item Set the photometer to the lowest sensitivity (the 1000 setting) and use the zero adjust knob to make sure that the pointer is at the 0 position when the incandescent light source is turned off.

\item Turn the incandescent light source back on and adjust the sensitivity of the photometer such that the needle is at the highest position without being at the maximum reading. The value of the sensitivity corresponds to the maximum value on the analog scale.

\item Record this measurement as the intensity for $0^{\circ}$ difference between the two polarizing axes.

\item Begin by rotating one of the polarizers by 10 degrees and measure the intensity. Continue measuring the intensity for angles between $0^{\circ}$ and $90^{\circ}$ in increments of 10 degrees. If the readings become too small you may have to switch the photometer to a setting with a higher sensitivity. Make sure you are always keeping one polarizer aligned at $0^{\circ}$.

\item Plot the relative intensity ($I$ divided by $I_0$, the intensity at $0^{\circ}$) vs. the $\cos^2\theta$ where $\theta$ is the angle between the axes of your polarizers. Be sure to include error bars on both axes and a line of best-fit. You can determine the error in $I/I_0$ by examining the precision of the photometer. You can determine the error in $\cos^2\theta$ using the following equation:
\begin{equation}
 \sigma_{\cos^2\theta} = 2\cos(\theta)\sin(\theta)\sigma_{\theta}
\end{equation}

\item Determine the slope and intercept o your plot with error found using LINEST. What values do you expect for the slope and intercept? Do your measured values agree within error?

\item What reading do you obtain when the polarizers are at an angle of $90^{\circ}$ relative to each other? (This is called the ``noise'' of your measuring device.) What reading would you expect if there was no noise?

\item If you had a lot of background light, how could you reduce the influence it would have on your results (by changing the setting or by changing your data analysis)?

\item If you had aligned the $90^{\circ}$ mark instead of the $0^{\circ}$ mark of both polarizers with the white mark on the holder would your results have been any different? What relative intensity would you expect if the angle between the two polarizers was $180^{\circ}$? $270^{\circ}$?

\item What are the main sources of error? Which do you think contributes most?
\end{enumerate}

\subsection{Double Slit}
\begin{enumerate}
\item Before you start this section, be sure you have read the laser safety note above.

\item Your TA will demonstrate the interference pattern by projecting it on the wall. Note down your observations!

\item Remove the incandescent light source, the polarizers, and the holders from the track, and the fiberoptic cable from the linear translator.

\item The laser should already be mounted on the laser holder at the far end of the track. Rotate the dial on the linear translator so that it is in the middle position.

\item Turn on the laser. It should propagate through the center of the hole in the linear translator, producing a bright red dot on the wall. If the laser is not properly aligned, alter the position of the laser by adjusting \emph{only the back legs} on the laser holder. Make sure you avoid lifting the laser and shining it in your classmates' eyes.

\item Place the red filter (number 32) onto the linear translator. Carefully slide the fiberoptic cable back into the hole in the linear translator so that the front of the cable sits right against the red filter.

\item Place the slide with the slits (number 13) in a holder and place the holder right in front of the laser.

\item Adjust the slide position in the holder so that the laser propagates through slit pattern A.

\item Note the interference pattern on the front of the linear translator.

\item Adjust the photometer sensitivity to an appropriate setting. Try turning the knob on the linear translator. This should cause the intensity reading on the photometer to change. You should notice the intensity has various maxima.

\item Move the linear translator as far as possible, starting from the 0 position of the linear translator. Record the position and intensity for each maximum. As you turn the knob on the linear translator, take care not to shift the position of the linear translator on the bench.

\item Where do you see the $0^{\mathrm{th}}$ order maximum (the tallest peak from Figure \ref{fig:intensity})? Why might it be located somewhere other than the middle position of the linear translator?

\item Measure and record the distance $L$ from the position of the slits to the tip of the fiberoptic cable.

\item With Microsoft Excel, plot the position of the maxima $x_m$ vs. the order $m$. Only include error bars for the position. Determine the slope of your line of best-fit and use it to calculate the wavelength of the laser light with error using LINEST. Note that the slit separation for slit A is 0.250 mm.

\item With Microsoft Excel, pot the relative intensity, $I/I_0$ vs. $x_m$. Explain the shape of your graph.

\item What is the purpose of using the red filter?

\item Discuss the main sources of error in this experiment.
\end{enumerate}

\section{Applications}
Some chemical molecules have a property called chirality. This means that the mirror image of the molecule is not identical to the original. For example, no matter how good you might be at 3D puzzles, you cannot arrange images of the molecule and of its mirror image to superimpose precisely. A famous example is the DNA double helix, a highly complex molecule, which carries all genetic information. The molecule always twists in one direction, while its mirror image always twists in the opposite direction. Many such chiral molecules are optically active, and rotate the polarization-axis of transmitted or reflected polarized light\footnote{For instance, the glucose dextrose turns the polarization axis to the right: ``dexeter'' is Latin for ``right''.}. If you shine polarized light into such a sample, you will see that the light leaving the sample has its polarization axis rotated a few degrees clockwise or counterclockwise, depending on the substance.\myskip

This effect can be used to determine the concentration of chiral molecules in a solution. For example, you can determine the glucose concentration in a blood specimen in a non-chemical way. The calculation of the concentration is particularly simple, since the change in angle of the polarization axis depends only on the concentration of the chiral substance in the solution, the distance the light travels through the sample, and a constant specific for the substance. Calculating the glucose concentration, therefore, requires only a measurement of the rotation in polarization vector and known constants -- a simple task for a rudimentary computer.

\newpage
\section{Lab Preparation Examples}
\underline{Polarization:}
\begin{enumerate}
\item You look at the sky through a polarizer and as you turn the polarizer you see that the sky appears darker or lighter, depending on the position of the polarizer. What does this observation tell you?

\item You look at a light source through a polarizer. As you turn the polarizer you see no change in intensity. Is the light emitted by this source polarized?

\item Unpolarized light passes through two polarizers whose polarization axes are aligned. You note down the observed intensity as $I_{0}$. Now you turn the polarization axis of one of the polarizers by $60^{\circ}$. What intensity do you observe now?

\item Now you add, after the second polarizer, a third polarizer. The third has a polarization axis of $30^{\circ}$ relative to the second polarizer. What intensity do you observe now?

\item Does the result depend on which of the two polarizers you rotate in question 3?

\item Would your answer to question 5 change if the light source produced polarized light? Explain!

\item You have two polarizers with an angle of $90^{\circ}$ between their polarization axes. Adding a third polarizer between these two polarizers will typically produce light after the final polarizer. At what angle (relative to the first polarizer's axis) would you observe the maximum intensity after the third polarizer?
\end{enumerate}

\underline{The Double-Slit:}

While performing a double slit experiment, with $L = 1.00\pm 0.01\,\mathrm{m}$, $d = 0.5\pm 0.1\, \mathrm{mm}$, you obtain the following locations for maxima:
\begin{table}[h]
 \centering
 \begin{tabular}{|c|c|c|c|c|c|c|c|c|c|}
  \hline
  order&-4&-3&-2&-1&0&1&2&3&4\\
  \hline
  Position (mm)&-4.1&-3.2&-2.0&-0.9&0.0&1.3&2.1&3.2&3.9\\
  \hline
 \end{tabular}
\end{table}
\begin{enumerate}\setcounter{enumi}{7}
\item Calculate x for all of the consecutive maxima and use the 2/3 estimate to determine the uncertainty of x. Now determine the wavelength of the light (including uncertainty).
\item Using the data above, determine x from a plot of position vs. order. Given the wavelength as $\lambda = 600\,\mathrm{nm}$ and $L = 1.00\pm 0.01\,\mathrm{m}$, what is the spacing between the two slits?
\end{enumerate}

\section{Appendix for Experts: How a Laser works (Not required!)}
Quantum mechanics tells us that light is produced when electrons in an atom make a transition from a higher to a lower state, or energy level. Whenever an electron makes a
transition between states, the difference in energy is emitted (or absorbed) as a discrete energy package called a photon\footnote{See the experiment on the Photoelectric Effect.}, which carries (with other photons) the electromagnetic wave. The transition from a higher to a lower state can happen in two different ways. First, it can happen randomly, i.e. the electron just falls down spontaneously. This is what happens when a candle produces light. Electrons that make such random transitions result in emitted photons which are incoherent, or out of phase with one another. \myskip
\begin{figure}[h]
\centering
\includegraphics[width=0.5\textwidth]{./Exp8/pic/image9.png}
\caption{Process of Stimulated Emission}
\label{fig:stimulatedemission}
\end{figure}

The second way this process can happen is called stimulated emission. If you already have a field of photons that are in phase and have the same frequency, and that frequency corresponds exactly to the transition energy, the emission is no longer random. When an electron falls to the lower energy level, a photon is emitted precisely in phase with the already existing photons, as shown in Figure {\ref{fig:stimulatedemission}}.\footnote{Photons are of a particle type called bosons. Such particles are highly ``social'' and like to be in exactly the same energy state and the same phase with their friends. This contrasts with another particle type, called fermions, which hate being in the same state as their associates (Pauli Exclusion Principle). Electrons are fermions, and this antisocial property accounts for the unique state of each electron in an atom. If electrons were bosons, all electrons in an atom would be in the ground state and we would not have chemistry!} In our analogy of soldiers marching on parade, new entrants to the parade are required to start walking in step with the others (or they would miss out on the parade).\myskip
\begin{figure}[h]
\centering
\includegraphics[width=0.5\textwidth]{./Exp8/pic/image10.png}
\caption{The basic three - level scheme for laser operation.}
\label{fig:transition}
\end{figure}

For stimulated emission to work, the upper atomic state must be full of electrons and the lower one must be almost empty. This is called an inversion, since atoms (at ordinary temperatures) tend to be in their lowest energy state. The inversion is actually achieved by pumping the electrons from the lower state into a third transient state, with an energy even higher than the second laser state, so that the electrons fall down to replenish the upper laser state. So real lasers use at least three energy levels; efficient lasers usually use even more\footnote{Semiconductor lasers (or laser diodes) work somewhat differently, but the underlying concepts of stimulated emission and inversion are the same.}.\myskip

Finally, the setup is placed in a cavity with a mirror at one end and a semi-transparent\footnote{These semi-transparent mirrors are still highly reflective. They reflect about 99$\%$ of incident light to keep most of the light inside the cavity.} mirror at the other end. This increases the efficiency of the laser by maintaining a high density of laser light within the cavity. The little leakage emitted from the semitransparent mirror is actually what we use as the laser beam.
 %not done - Images Restored
%\chapter{The Spectrum of the Hydrogen Atom}
\section{Introduction}
In this experiment we will observe the discrete light spectrum one observes from a gas discharge lamp. We will see that the spectrum consists of a collection of sharp, single colored lines. We will be able to measure the wavelength of the light emitted quite precisely, usually better than 1 part in a thousand. Therefore it is crucial to make all calculations to 5 significant figures.

\section{Theory}
\subsection{The Spectrum of the Hydrogen Atom}
According to the classical theory of electro magnetism, atoms should radiate continuously and the electrons should fall into the nuclei within a short timespan. Obviously this is not the case since we have stable atoms. Therefore the classical description seems to be wrong. \myskip

Niels Bohr suggested an alternative theory for atoms, suggesting that the electrons can only exist in certain energy states and can only jump between these discreet states discontinuously. This discontinuity explains why we see a collection of sharp lines in a gas discharge lamp, and not a continuous spectrum as in a usual light bulb. \myskip

Bohr was able to derive the following formula for the energy of the energy levels only using natural constants
\begin{equation}
  E_{n}=-\frac{2\pi^2 me^4k^{2}_{e}}{n^2h^2}=-\frac{me^4}{h^2}\bigg(\frac{1}{8\varepsilon^{2}_{0}}\bigg)\frac{1}{n^2}
\end{equation}
Therefore the energy one gets for a transition from an initial level $n_i$ to a final level $n_f$ is $\Delta E = E_{n_{i}} - E_{n_{f}}$.\myskip 

This energy is then set free in the form of light and one can relate it to the wavelength via
\begin{equation}
  \frac{1}{\lambda}=\frac{f}{c}=\frac{\Delta E}{hc}=\frac{E_{n_{i}}-E_{n_{f}}}{hc}=R\bigg(\frac{1}{n^2_{f}}-\frac{1}{n^2_{i}}\bigg)
\label{eq:lambda}
\end{equation}
where $R$ is called the Rydberg constant, which is given by
\begin{equation}
  R=\frac{2\pi^2me^4k^2_{e}}{h^3c}=1.0974\times 10^{7}\, \mathrm{m}^{-1}
\end{equation}

\subsection{Resolving a Light Spectrum with a Grating}
A grating is a collection of small parallel slits. In our case the slits are so fine that we cannot see them any more. A grating has the nice property that light of different wavelength gets diffracted in different directions. Therefore one can resolve a continuous light spectrum (containing light of various wavelengths) into its components. \myskip

How does a grating do this trick? To understand this it is sufficient if we look at only two slits with a distance $d$ between them. ($d$ is also called the lattice constant)\myskip

We look at two light rays coming from the two slits. If the two rays travel perpendicular to the two slits the two waves will always be in phase, since they travel the same distance. But if the two rays propagate at an angle $\theta$ to the normal the two rays will have to travel different distances to reach the eye. (we always assume that the two rays are parallel, which is a good approximation since the eye is almost infinitely far away compared to the distance between the two slits.)
\begin{figure}[h]
\centering
\includegraphics[width=0.3\textwidth]{./Exp9/pic/image1.png}
\caption{Geometry of the Double Slit}
\label{fig:slit}
\end{figure} 

As you may convince yourself by looking at the graph, the difference in path between the two rays is
\begin{equation}
  \Delta l =d\sin\theta
\end{equation}

As in the lab about standing waves we can have different possibilities how these two rays add up. They can add up to form a node or they can add up to form an anti node (or something in between). To form a node every mountain shall be made lower and every valley shall be made higher. Therefore each hill from one wave should meet a valley from the other ray and vice versa. To get an antinode and therefore a maximum in the total motion of the wave every hill should meet a hill and every valley should meet a valley.
\begin{figure}[h]
\centering
\includegraphics[width=0.8\textwidth]{./Exp9/pic/image2.png}
\caption{Young's Double Slit}
\end{figure} 

How do we now achieve maximum intensity (node)? We always get this if the two waves are in phase which means that there fits an exact number of wavelength $m$ in the difference between the two waves, i.e.
\begin{equation}
  \Delta l=m\lambda
\end{equation}
$m$ is also called the order. If we now put the two equations for $\Delta l$ together, we get
\begin{equation}
  d\sin\theta=m\lambda
  \label{eq:dsin}
\end{equation}
\begin{figure}[h]
\centering
\includegraphics[width=0.6\textwidth]{./Exp9/pic/image3.png}
\caption{Interference of Light Coming from Two Small Slits}
\end{figure} 

This tells us that given the order $m$ we can determine the wavelength of the light we see through the grating by reading off the angle. This also works the other way around. (We can also get m by count up the orders starting with 0 at an angle of 0 degrees, and then count how often a particular color repeats as you go to higher or lower angles.)

\subsection{How to read a Vernier Scale}
With a usual ruler you can read distances up to a mm resolution. But if you would like to have a measuring device to read with a $0.1\, \mathrm{mm}$ resolution you have a problem. In principle you can scratch such a fine scale in the ruler, but in practice you can easily imagine that due to the width of the marks themselves such a scale would be almost impossible to read. That is why we use a trick and introduce a Vernier scale.\myskip
A Vernier scale works in the following way:

\begin{enumerate}[a)]
\begin{figure}[h]
\centering
\includegraphics[width=0.5\textwidth]{./Exp9/pic/image4.png}
\end{figure} 
\item If the zero points of the vernier and main scales are aligned, the first vernier division is 1/10 of a main scale division short of a mark on the main scale; the \engordnumber{2} vernier division is 2/10 of a main scale division short of the next mark on the main scale, etc., and the \engordnumber{10} vernier mark coincides with a main scale mark.

\begin{figure}[h]
\centering
\includegraphics[width=0.5\textwidth]{./Exp9/pic/image5.png}
\end{figure} 
\item If the vernier scale is now moved to the right until the \engordnumber{4} vernier division is lined up with the nearest main scale division to its right, the distance the 0 point of the vernier scale has moved past the 0 point on the main scale is 4/10 of a main scale division. Thus the vernier scale gives the fraction of a main scale division that the zero point of the vernier has moved beyond any main scale mark.
\begin{figure}[h]
\centering
\includegraphics[width=0.5\textwidth]{./Exp9/pic/image6.png}
\end{figure} 

\item In this last example, the zero line of the vernier scale lies between $1.5\,\mathrm{cm}$ and $1.6\,\mathrm{cm}$ on the main scale. The fraction of a division from $1.5\,\mathrm{cm}$ to the zero line can be determined as follows: Line number 6 of the vernier scale coincides with a line on the main scale, so the zero line on the vernier has moved 6/10 of a main scale division away from the main scale line 1.5. Hence, the reading, to the nearest hundredth, is 1.56. The maximum uncertainty is now $\pm 0.01\,\mathrm{cm}$, corresponding to the smallest division on the vernier scale. The use of the vernier has reduced the maximum uncertainty of the caliper measurement from $\pm 0.1\,\mathrm{cm}$ (without the vernier) to $\pm 0.01\,\mathrm{cm}$.
\end{enumerate}

In this experiment we will not use a simple Vernier scale since we do not measure distances, but we will use an angular Vernier scale to measure sub-divisions of angles. If you look at the main scale you will see that the scale is divided in half-degree steps. (You have slightly longer marks for the full degrees and shorter marks for the half degrees.) For angles the next smaller unit is not 1/10 th of a degree but minutes, which is 1/60 th of a degree. But since we can already read the scale up to 1/2 degree we need not have a Vernier scale with 60 devisions but with 30 divisions.\myskip

So you first use the 0 mark on the Vernier scale to read off the angle in degrees and if you are in between 0 and 30 minutes or 30 and 60 minutes. Then you look which of the
markes on the Vernier scale exactly matches a mark on the other scale. So e.g. if the 0 mark on the Vernier scale is between 20.5 and 21 degrees and the \engordnumber{13} mark on the
Vernier scale matches a mark on the degree scale we would have a reading of $20 + 1/2+ 13/60$ degrees or 20 degrees and 43 minutes.

\section{Experiments}
\begin{figure}[h!]
\centering
\includegraphics[width=0.8\textwidth]{./Exp9/pic/image7.png}
\caption{Schematic of the Spectrometer}
\end{figure} 

\subsection{Adjusting the Spectrometer}
The first part of the experiment will basically include the procedure to set up the equipment. This should be done with as much care as possible. Only then we will be able to measure the wavelength on the limit of our apparatus. If you don't set up the spectrometer correctly you will get systematical errors, which screw up your data. 

\subsection{Obtaining the Lattice Constant}
 After adjusting the spectrometer we will measure the yellow line of a Helium discharge lamp. Since we know that the wavelength of this light is $\lambda = 5.8756\times 10^{-7}\,\mathrm{m}$, we can determine the lattice constant of the grating quite accurately. Even though the lattice has written on it 600 lines/mm (which is only an approximate value anyway), we want to get the lattice constant with 5 relevant digits and not just 3, and this means that we have to measure it!

\subsection{Measuring the Spectrum of Hydrogen Atoms}
With the lattice constant determined in the previous part we now measure the wavelength of the light emitted from the Hydrogen discharge lamp. 

\section{Step by Step List}
\subsection{Adjusting the Spectrometer}
\begin{itemize}
\item Take the grating out off the holder and close the green knob.

\item Rotate the yellow knob such that the slit is about half open.

\item Look trough the eyepiece and turn the purple focusing ring until you see a sharp image of the slit.

\item Loosen the red knob and move the telescope tube until the cross hair is in the middle of the slit. Tighten red knob.

\item Now open the green knob and turn the tabletop such that the 0 mark from the Vernier scale with the magnifying glass is lined with either the 180 or the 360 from the outer scale. (Always use only this Vernier scale and don't switch to the other one in between)

\item Close the green knob (and don't open it again for the rest of the experiment!).

\item Now you can fine adjust the relative position of the inner and outer scale by turning the blue knob. Make sure that the lineup between the 0 on the Vernier and the 180/360 is done as careful as possible. (For some of the spectrometers there is a small mark to the left of the 0 mark on the Vernier scale. Make sure that you line up the 0 mark and not this extra mark with the 180/360.)

\item Now put the grating in the holder such that it is perpendicular to the telescope tube-collimator tube line. Close the white screw to lock the grating.
\end{itemize}

\subsection{Obtaining the Lattice Constant}
\begin{itemize}
\item Switch on the Helium lamp and line the spectrometer up such that you can see the slit well illuminated by the lamp as you look through the spectrometer.

\item Put the black cardboard over the front end of your collimator tube and you can use the black piece of cloth over the spectrometer to block light from the surrounding. Also this part and most important the next one should be performed in the dark, therefore switch off the light and use the lamps provided (they are supposed to be that dim) to read the scale.

\item Open the red knob and move the telescope tube to the left until the crosshair is in the center of the yellow line. (You should first see a few blue and green lines, then the isolated yellow line and then red lines. The yellow line should be somewhere around 20 degrees)

\item Note down the angle in degrees and minutes at which you see the \engordnumber{1} order of the yellow line. Do the same on the right side and average these two numbers.

\item Use the average and plug it into the grating equation ($m=1$) to determine the lattice constant $d$ (at least 5 relevant figures).

\item How many lines per mm does this lattice constant correspond to?

\item Can you also see the \engordnumber{2} order and \engordnumber{3} order yellow lines on either side?
\end{itemize}

\subsection{Measuring the Spectrum of Hydrogen Atoms}
\begin{itemize}
\item Now switch off the helium lamp and set up the spectrometer for the hydrogen lamp.

\item The light purple line you see in the middle is the $0^{\mathrm{th}}$ order and not yet one of the lines we are going to measure.

\item There are 4 visible lines for the Hydrogen spectrum. One red (furthest out), one greenish blue, one purple blue and one dark purple. The dark purple line is very
faint and you may not be able to see it, so don't despair.

\item Measure the angles of the 4 lines on both sides and average the angles.

\item How many orders do you see on either side? (look e.g. for the red line and count how often it appears as you go further out)

\item Do the higher orders overlap? (i.e. does a new order start before the old one ended?)
\end{itemize}

\subsection{Analyzing the Data}
\begin{itemize}
\item Put together formulas ({\ref{eq:lambda}}) and ({\ref{eq:dsin}}) and solve the resulting formula for $n_i$. 
\item Use your data and the value for $R$ from above and $n_f=2$ to determine $n_i$.
\item $n_i$ should be an integer number labeling from which initial atomic shell the electrons in the hydrogen atoms fell to the \engordnumber{2} atomic shell.
\item Are the results integers (or close to them)?
\item Are the numbers (integers) the one they are supposed to be?
\item For which color did the electrons jump from the lowest shell? Explain why you could have predicted that anyway!
\item Within how many \% have you been able to measure the data? Compare that to other experiments you have performed in this course!
\item Discuss briefly the factors that determine the uncertainty in your measurements. Which of them are random, which systematic?
\end{itemize}

\newpage
\section{Lab Preparation Examples}
\underline{Hydrogen Spectrum:}
\begin{enumerate}
\item What is the wavelength of emitted if an electron in an hydrogen atom makes a transition from $n_i=3$ to $n_f=1$?

\item For a hydrogen atom give all the transitions that fall within the visible range of the light spectrum (i.e. between $\lambda_{\mathrm{min}}=400\,\mathrm{nm}$ and $\lambda_{\mathrm{max}}=800\,\mathrm{nm}$). Give the transitions in the following form:
A transition from energy level $n_i=3$ to $n_f=1:3 \rightarrow 1$.\myskip

Hint: It may be simplest to calculate the energies $\lambda_{\mathrm{min}}=400\,\mathrm{nm}$ and $\lambda_{\mathrm{max}}=800\,\mathrm{nm}$ correspond to. Then make a list of $E_{n}$ and choose the differences between the energy
level that fit in the calculated range.

\item What wavelengths do the transitions in the previous problem correspond to?
\item If you look at the light of an object at high temperature (e.g. a normal light bulb), you will find that it usually emits a continuous spectrum without any gaps or lines in it. (This is due to the effect that the atoms in a solid (or plasma) are so close together that they disturb each other so strongly that the initially sharp transition lines of the atoms get so broadened that they appear as a continuous spectrum.)\myskip

But if you now look at the spectrum of the sun (or another star), you will find a continuous spectrum that has a number of sharp black lines, so no (or almost no) light of that frequency reaches you; the light of that frequency is missing in the spectrum. These black lines are at exactly the same positions where you would see the colored lines in emission spectra (that is what we do in lab) of the elements present on the sun. For example since hydrogen is present on the sun we will find black lines at the
wavelength calculated in 3.\myskip

Can you give a short explanation of how we get these inverse spectra (compared to the ones we see in the lab from emission spectra) from the sun? 
\end{enumerate}

\noindent\underline{Spectrum with a Grating:}
\begin{enumerate}\setcounter{enumi}{4}
\item What is the lattice constant $d$ if the lattice has 400 lines per mm?
\item If you have a grating with 600 lines per mm at what angle would you observe the $m=1$ maximum for a wavelength of $600\,\mathrm{nm}$?
\item If you have $d = 10^{-1}\, \mathrm{m}$ and $\lambda = 1\,\mathrm{cm}$. at what angles will you see the maxima for $m=1, 2, 3$?
\end{enumerate}

%\chapter{Absorption of Beta and Gamma Rays}

\section{Introduction}

In this lab, we will study the behavior of gamma and beta rays as they pass through solid matter. Specifically, we will use beta ray absorption to measure the range of beta particles from a specific source and from that determine the energy of beta decay in Thallium 204. Additionally, we will determine the absorption coefficient of gamma radiation in lead from a Cesium 137 source.\myskip

The radioactive sources used in this experiment are all extremely weak and therefore can be handled without any danger of overexposure to radiation. However, it is always prudent not to handle any radioactive source more than is necessary.

\section{Theory}
\subsection{Nuclear Decay}
All nuclei heavier than lead (and many isotopes of lighter nuclei) have a non-zero probability of decaying spontaneously into another nucleus plus one or more lighter particles. One of the decay products may be an alpha-particle (two protons and two neutrons--the stable nucleus of a helium atom). Alternatively, a nucleus with more neutrons than it can maintain in stability may decay by emission of a beta-particle (a fast-moving electron or positron) which corresponds to the conversion of a neutron to a proton. These beta electrons may emerge with a kinetic energy of up to several MeV.\myskip

\begin{figure}[h]
\centering
\includegraphics[width=0.5\textwidth]{./Exp10/pic/image1.png}
\caption{Typical Nuclear Energy Level Diagram.}
\label{fig:level}
\end{figure}

After alpha or beta decay, the residual nucleus may be left in an excited state. In this case, a transition to a state of lower energy of the same nucleus will occur almost immediately with the emission of a photon (also called a gamma ray). The spectrum of photons emitted from the various excited states of a nucleus will have discrete frequencies $f$, corresponding to transitions $\Delta E=hf$, between discrete energy levels of the nucleus. The spectrum from an excited nucleus is thus \textbf{analogous} to the line spectrum of visible radiation from an atom due to excited electrons, with the notable difference that the MeV energy changes of the nucleus are approximately $10^6$ times as large as energy changes in transitions between atomic states (where $\Delta E_{\mathrm{atomic}}\approx$ several eV). See Figure {\ref{fig:level}}.\myskip

\begin{figure}[h]
\centering
\includegraphics[width=0.65\textwidth]{./Exp10/pic/betaspectrum.jpg}
\caption{Beta Ray (Electron) Energy Spectrum}
\label{fig:betaray}
\end{figure}

In early experiments on beta-decay, it was observed that each decay was not a simple one in which an electron and the recoil nucleus came off with equal and opposite momentum. The electrons, in fact, were emitted with a continuous spectrum of energies (as in Figure {\ref{fig:betaray}}). It was subsequently suggested by Pauli and Fermi that in each decay, another particle of zero mass and charge, called the neutrino, was emitted. Experimental verification of the neutrino has been obtained by observation of its rare interaction with matter.

\subsection{Detection of Charged Particles (the Geiger Counter)}
When an energetic charged particle traverses matter, it will give electrostatic impulses to nearby electrons and thus ionize atoms of the material. Most methods for detecting nuclear particles rely on observing the results of this ionization. For example, in a photographic emulsion, a trail of ionized grains will show up when the film is developed; in a cloud or bubble chamber, a trail of droplets or bubbles forms along the wake of ionization left by a charged particle; and in a scintillation counter, ionized molecules will very rapidly radiate visible light. In the \textbf{Geiger counter}, which will be used as a detector in this experiment, the ionization produced by a charged particle causes a violent electrical discharge.
\begin{figure}[h]
\centering
\includegraphics[width=0.5\textwidth]{./Exp10/pic/image3.png}
\caption{The Geiger Counter.}
\label{fig:counter}
\end{figure}

As shown in Figure {\ref{fig:counter}}, a Geiger counter comprises a metal cylinder (cathode) with insulating ends supporting a fine axial wire (anode). When a charged particle ionizes the gas (e.g., argon) in the tube, the resulting free electrons will move toward the positively charged anode wire, accelerated more and more by the rapid increase in electric field near the wire. When an electron acquires kinetic energy greater than the ionization energy of the gas molecules, it can create by collision a new ion and electron, which in turn can accelerate and create another ion, etc., thus initiating an avalanche of charge. The process, called the \textbf{Townsend avalanche}, is possible only if the voltage maintained between anode and cathode is sufficiently high.\myskip

The basic counter circuit, shown in Figure {\ref{fig:geiger}}, supplies a positive high voltage of up to 900 volts to the center wire. When an avalanche occurs, current flows through resistor $R$, the counter side of $R$ drops in potential, and this negative pulse is fed through capacitor $C$ to a stage of amplification and then to a scaling device.\myskip
\begin{figure}[h]
\centering
\includegraphics[width=0.8\textwidth]{./Exp10/pic/image4.png}
\caption{Geiger Counter Circuit.}
\label{fig:geiger}
\end{figure}

For a large enough value of $R$, the voltage drop across R would be sufficient to stop the discharge after a count has been recorded. This method of quenching the discharge, however, has the disadvantage of creating a long time constant, $\tau = RC$, for reestablishing voltage across the counter, i.e., of creating a long ``dead time'' during which the passage of another particle cannot be detected. The quenching can be achieved for a lower value of $R$ by adding halogen vapor to the gas to help absorb ion pairs. Nevertheless, the resultant dead time is substantial and must be taken into account when the particle flux is high.

\subsection{Energy Loss and Range of Beta Particles}
An incident particle in matter will continually lose kinetic energy as it ionizes atoms in the material (Figure {\ref{fig:ionizing}}) until it eventually comes to rest. The distance a particle can travel through a material before losing all of its kinetic energy and stopping is called its \textbf{range}. \emph{For a particle of known charge and mass, there will be a unique range associated with each incident energy}. A formula can be theoretically deduced for the rate of energy loss--and hence the range--of a particle (of known mass, charge, and initial velocity) in a particular ``stopping'' material (of known electron density and ionization potential).
\begin{figure}[h]
\centering
\includegraphics[width=0.5\textwidth]{./Exp10/pic/image5.png}
\caption{Ionizing Action of Incident Electron.}
\label{fig:ionizing}
\end{figure}

In each interaction with atomic electrons, however, an incident electron may be scattered through small or relatively large angles, and as it traverses the material it may follow a rather tortuous, winding path (especially at low energies). Therefore, the actual path of the electron may be considerably longer than the observed distance that it penetrates into the material. For this reason, the incident electron range is not sharply determined, and the theoretical calculation is of limited usefulness for electrons of less than $1\, \mathrm{MeV}$ energy.\myskip

In this experiment, therefore, we will use an approximate empirical relationship between range and energy for low energy electrons:
\begin{equation}
  r=\frac{0.412\, \mathrm{g}/\mathrm{cm}^2}{\rho}E^{1.29}
\label{eq:re}
\end{equation}
where $r$ is in cm, $E$ is in MeV, and $\rho$ is the density of the stopping material in $\mathrm{g} / \mathrm{cm}^3$. Note that the density of Aluminum is $2.702\,\mathrm{g} /\mathrm{cm}^{3}$.\myskip

(This result is described in: L. Katz and A. S. Penfold, ``Range-Energy Relations for Electrons and the Determination of Beta-Ray End-Point Energies by Absorption,'' Revs.Modern Phys. \textbf{24}, 1 (1952).)

\subsection{Absorption of Gamma Rays}
Gamma rays, or high-energy photons, can interact with matter by three distinct processes:
\begin{enumerate}[1)]
\item Compton Scattering: This refers to a photon-electron collision in which the energy lost by the scattered photon is given to the recoil electron.
\item Photoelectric Effect: The photon is absorbed by the atom as a whole, releasing an electron with kinetic energy equal to $E_{\gamma} - E_{b}$ , where $E_{\gamma}$ is the photon energy and $E_{b}$ is the relatively small binding energy of the electron in the shell from which it is released.
\item Pair Production: If the photon has energy greater than $1.02\, \mathrm{MeV}$, it can create an electron-positron pair in the neighborhood of a nucleus. The radiative source used in this experiment does \textbf{not} emit photons with energy greater than $1\, \mathrm{MeV}$.
\end{enumerate}

The probability of each of the three processes taking place in a given thickness of material depends on the energy of the photon and the atomic structure of the material. The \emph{total} probability for interaction of photons in lead (i.e., the sum of the probabilities of the three processes) varies with photon energy as indicated in Figure {\ref{fig:linear}}. The coordinate plotted on the graph is $\mu$, the \textbf{total linear absorption coefficient} in units of $\mathrm{cm}^{-1}$. It is defined by the equation:
\begin{equation}
  \frac{d N}{d x}=\mu N
\end{equation}
where $N$ is the number of incident photons and $dN$ is the number removed from the beam (i.e., absorbed) in an absorber of thickness $dx$ (in cm). Note that $dN$ and $dx$ are the calculus equivalents of infinitesimally small values of $\Delta N$ and $\Delta x$, respectively. As in any process where the rate of decrease is proportional to the number present (such as the discharge of a capacitor), the solution of this differential equation is:
\begin{equation}
  N(x)=N_{0}e^{-\mu x}
\label{eq:nx}
\end{equation}
where $N(x)$ is the number of photons passing through $x\,\mathrm{cm}$ of absorber and $N_0=N(x)$ at $x = 0$ , and $e$ is the base of natural logarithms.
\begin{figure}[h]
\centering
\includegraphics[width=0.8\textwidth]{./Exp10/pic/image6.png}
\caption{Linear Absorption Coefficient μ for gamma rays in lead as a function of energy.}
\label{fig:linear}
\end{figure}

\section{Procedure}
\subsection{Adjustments and Measurement of Errors in Counting}
\subsubsection{High Voltage Variations}

Every Geiger tube that is in good working order has a plateau region in which its counting rate is relatively insensitive to changes in the high voltage supply. Follow these steps to find the optimal measurement voltage for your Geiger counter.
\begin{enumerate}
  \item Place a source under the tube and increase the high voltage from its lowest value until the tube just begins to count.
  \item Take 30 second counts while raising the voltage in 50 volt steps.
  \item At some point, the curve of counting rate vs. voltage levels off. You will know you have hit the plateau if for a 100-volt increase, your count rate only goes up a very small amount (less than 10\%).
\end{enumerate}
Do not raise the voltage further as this may damage the tube due to continuous breakdown. Set the high voltage to a value on this plateau for the remainder of the experiment. If this procedure is followed correctly, high voltage variations may be ignored as a source of error.

\subsubsection{Statistical Accuracy}

Particles decay randomly in time from a radioactive source (over a period short compared to the half-life). The probability distribution for measuring a given number of these counts in a given time interval is an almost bell shaped curve (a Poisson Distribution) centered on $N_0$, the most probable value. The distribution will have a standard deviation about the peak of $\pm \sqrt{N_{0}}$ . If $N$ counts are measured in an interval, the best estimate of the error is $\pm \sqrt{N}$ .\myskip

Note that the magnitude of the statistical error--your uncertainty in the measurement--increases significantly for trials involving a very small number of counts. For a high counting rate in a given interval, say, 900 counts in one minute, the estimated error will be 30 counts per minute or 3.3\% error. However, for a much lower counting rate in the same interval, say 25 counts, the error of $\pm$5 counts per minute amounts to a 20\% statistical error. In order to achieve the same precision as in the first case, it would be necessary to collect 900 counts--in other words, to take a 36-minute measurement. While such a long measurement is impractical for this lab, you should keep in mind the relationship between a small number of counts and higher errors, and do your best to minimize these errors by taking longer measurements when necessary.

\subsubsection{Measuring Background Radiation}

In order to make accurate counting measurements of the sources, it is necessary to know the counting rate due to natural background radiation (mostly cosmic rays coming through the earth's atmosphere). Additionally, there will be some excess counts due to the Cs-137 gamma sources nearby in the room, whose gamma rays can pass through the side of the detector. At a distance of $30\,\mathrm{cm}$, for example, the Cs-137 source contributes roughly as many counts as the natural background radiation (doubling the distance would reduce its contribution to one-fourth this level). It's best to try to minimize these secondary effects -- by keeping your detector far from others' sources, and by shielding your own Cesium source with the lead sheets when not in use), but it's even more important to try to keep these background effects \emph{constant}. If all of your data is shifted by roughly the same constant amount, then it is possible to isolate the results you're interested in by subtracting out this constant background.

\subsection{Range of Beta Particles}
\begin{figure}[h]
\centering
\includegraphics[width=0.8\textwidth]{./Exp10/pic/betaabsorption.jpg}
\caption{Beta Absorption Curve}
\label{fig:beta}
\end{figure}
Thallium 204 beta decays to Lead 204 with a half-life of 3.9 years. The range of the most energetic of the decay electrons can be determined by placing aluminum foil absorbers between the source and the Geiger counter.\myskip

A typical absorption curve is shown in Figure {\ref{fig:beta}}. The maximum range $r$ is the point where the absorption curve meets the background. You should start by making a careful measurement of background, and you should repeat this measurement after taking the absorption curve to check for constancy.\myskip

Place the Thallium source on the second shelf below the detector, as shown in Figure {\ref{fig:absorption}}, to maximize the number of counts while leaving enough room to stack aluminum absorbers. Each of your measurements should be for a time interval of at least 30 seconds or until you get enough counts for a reasonable error.
\begin{enumerate}
  \item Begin taking measurements and adding aluminum foil absorbers until the recorded count rate has lowered to the background level. Record the number of counts for each aluminum foil sheet added. (\emph{Note that the thickness of the aluminum absorbers is marked on the foil in \textbf{mils}, or thousandths of an inch, not millimeters} (1 mil = 2.54$\times 10^{-3}$ cm).)
  \item Take another measurement of the background to verify that everything has remained constant.
  \item Make a table of the results, including the background level, and include estimates of the statistical error in each measurement. Note that you \emph{should not} subtract background from your data for this experiment.
  \item With Microsoft Excel, make a plot of the log of your count rate vs. aluminum thickness. This is your beta absorption curve!
  \item Determine the approximate value of the maximum range $r$ from your graph and use Equation ({\ref{eq:re}}) to compare your result with the value of $E = 0.765\ \mathrm{MeV}$ for the maximum beta energy for Thallium 204 as measured in a magnetic spectrometer.
\end{enumerate}

\begin{figure}[h]
\centering
\includegraphics[width=0.25\textwidth]{./Exp10/pic/image8.png}
\caption{Beta Absorption}
\label{fig:absorption}
\end{figure}

\subsection{Absorption of Gamma Rays}
The source for this part of the experiment is Cesium 137, which decays with a half-life of 30 years to Barium 137 with the emission of a $0.52\, \mathrm{MeV}$ beta ray (Figure {\ref{fig:cesium}}). The resulting Barium is in an excited state and decays by emitting a $0.662\, \mathrm{MeV}$ gamma ray almost instantaneously. Lead absorbers are used for the gamma absorption study. They are thick enough(0.062 inches) so that one absorber will stop all the betas.\myskip
\begin{figure}[h]
\centering
\includegraphics[width=0.5\textwidth]{./Exp10/pic/image9.png}
\caption{Cesium 137 Decay Scheme}
\label{fig:cesium}
\end{figure}

The gamma rays are detected by means of the same Geiger counter used in the previous part. Note that the efficiency of the Geiger counter for detecting photons is much less than for detecting the beta-particles, since it depends on a collision of the photon with the gas or wall of the counter, resulting in the emission of an electron, which in turn initiates the discharge.\myskip
\begin{figure}[h]
\centering
\includegraphics[width=0.5\textwidth]{./Exp10/pic/image10.png}
\caption{Scattering Effects: In (A), the gamma ray on the left passes outside of the detector tube. In (B), it can be seen how increasing the area of lead absorber can cause the same gamma ray to scatter \emph{into} the detector.}
\label{fig:scattering}
\end{figure}

Figure {\ref{fig:scattering}} represents one of the effects of gamma ray scattering in the lead sheets. Increasing the \emph{area} of lead through which the gamma rays pass tends to \emph{increase}, rather than decrease, the number of counts one measures, since gamma rays which otherwise would not have entered the detector may now be scattered into it.\myskip
\begin{figure}[h]
\centering
\includegraphics[width=0.4\textwidth]{./Exp10/pic/image11.png}
\caption{Absorption of Gamma Rays Set-Up.}
\label{fig:gammarays}
\end{figure}

The arrangement shown in Figure {\ref{fig:gammarays}} is designed to reduce the effect of such scattering. By keeping the lead sheets high above the source, one reduces the excess area exposed to gamma rays, and one reduces the effective difference in area between the top and bottom sheets as well.\myskip

\begin{enumerate}
  \item Take a measurement of the background count rate.
  \item Measure the number of counts per lead sheet thickness for as many sheets as you can fit in the apparatus. For your first measurement (for zero lead thickness) make sure the aluminum shelf is still present.\\ \\
  Note that gamma ray absorption differs fundamentally from beta absorption in the following way: there is no maximum range for gamma rays passing through lead; rather, one expects to lose a fixed fraction of the remaining gamma rays passing through each successive layer of lead absorber.

  \item Take another measurement of the background count rate to check for consistency. Use the average of these two count rates as your background for analysis.
  \item Make a table of data with error due to counting.
  \item With Microsoft Excel, make a plot of the log of count rate vs lead thickness with \emph{the  background count rate subtracted}. This exponential decay curve should appear as a straight line. Be sure to include error bars on your plot.
  \item Draw a line of best fit and determine error in the slope and intercept using LINEST.
  \item From Your slope, determine the absorption coefficient $\mu$ with error found by propagating uncertainties. Use your calculated value of $\mu$ and Figure \ref{fig:linear} to estimate the energy of the gamma rays.
  \item Does your calculated gamma ray energy agree with the accepted value (from Figure \ref{fig:cesium})?
  \item Discuss the main sources of error in determining gamma ray energy.
\end{enumerate}


\titleformat{\chapter}[display]
    {\bfseries\huge\filright}
    {\underline{Appendix \thechapter}}
    {0pt}
    {\huge}

\appendix
\appendixpage
\addappheadtotoc

\lhead[\rightmark]{Appendix 2-\thechapter}
\cfoot{\thepage}
\rhead[Appendx 2-\thechapter]{\leftmark}

\chapter{Review of Error Analysis}

\section{Types of Uncertainties}

Uncertainty in a measurement can arise from three possible origins: the measuring device, the procedure of how you measure, and the observed quantity itself. Usually the largest of these will determine the uncertainty in your data. \myskip

Uncertainties can be divided into two different types: systematic uncertainties and random (statistical) uncertainties\footnote{If you were to engage in further research, random uncertainty is typically referred to as statistical uncertainty.}.

\subsection{Systematic Uncertainties}


Systematic uncertainties or systematic errors always bias results in one specific direction. They will cause your measurement to consistently be higher or lower than the accepted value. \myskip

An \emph{example} of a systematic error follows. Assume you want to measure the length of a table in cm using a meter stick. However, the stick is made of metal that has contracted due to the temperature in the room, so that it is less than one meter long. Therefore, all the intervals on the stick are smaller than they should be. Your numerical value for the length of the table will then always be larger than its actual length no matter how often or how carefully you measure. Another example might be measuring temperature using a mercury thermometer in which a bubble is present in the mercury column. \myskip

Systematic errors are usually due to imperfections in the equipment, improper or biased observation, or the presence of additional physical effects not taken into account. (An example might be an experiment on forces and acceleration in which there is friction in the setup and it is not taken into account!) \myskip

In performing experiments, try to estimate the effects of as many systematic errors as you can, and then remove or correct for the most important. By being aware of the sources of systematic error beforehand, it is often possible to perform experiments with sufficient care to compensate for weaknesses in the equipment.

\subsection{Random Uncertainties}

In contrast to systematic uncertainties, random uncertainties are an unavoidable result of measurement, no matter how well designed and calibrated the tools you are using. Whenever more than one measurement is taken, the values obtained will not be equal but will exhibit a spread around a mean value, which is considered the most reliable measurement. That spread is known as the random uncertainty. Random uncertainties are unbiased -- meaning it is equally likely that an individual measurement is too high or too low. \myskip

From your everyday experience you might be thinking, ``Stop! Whenever I measure the length of a table with a meter stick I get exactly the same value no matter how often I measure it!''   This may happen if your meter stick is insensitive to random measurements, because you use a coarse scale (like $\mathrm{mm}$) and you always read the length to the nearest $\mathrm{mm}$. But if you would use a meter stick with a finer scale, or if you interpolate to fractions of a millimeter, you would definitely see the spread. As a general rule, if you do not get a spread in values, you can improve your measurements by using a finer scale or by interpolating between the finest scale marks on the ruler. \myskip

How can one reduce the effect of random uncertainties?  Consider the following \emph{example}. Ten people measure the time of a sprinter using stopwatches. It is very unlikely that each of the ten stopwatches will show exactly the same result. Even if all of the people started their watches at exactly the same time (unlikely) some of the people will have stopped the watch early, and others may have done so late. You will observe a spread in the results. If you \emph{average} the times obtained by all ten stop watches, the \emph{mean} value will be a better estimate of the true value than any individual measurement, since the uncertainty we are describing is random, the effects of the people who stop early will compensate for those who stop late. In general, making multiple measurements and averaging can reduce the effect of random uncertainty. \myskip

\emph{Remark}: We usually specify any measurement by including an estimate of the random uncertainty. (Since the random uncertainty is unbiased we note it with a $\pm$ sign). So if we measure a time of 7.6 seconds, but we expect a spread of about 0.2 seconds, we write as a result:
\begin{equation}
    t = (7.6\pm 0.2)\,\mathrm{s}
\end{equation}
indicating that the uncertainty of this measurement is $0.2\,\mathrm{s}$ or about $3\%$. \myskip
\section{Accuracy and Precision}
An important distinction in physics is the difference between the {\it{accuracy}} and the {\it{precision }} of a measurement.
Accuracy refers to the closeness of a measured value to a standard or known value. For example, if in lab you obtain a weight measurement of 3.2 kg for a given substance, but the actual or known weight is 10 kg, then your measurement is not accurate. In this case, your measurement is not close to the known value. \myskip

Precision refers to the closeness of two or more measurements to each other. Using the example above, if you weigh a given substance five times, and get 3.2 kg each time, then your measurement is very precise. Precision is independent of accuracy. You can be very precise but inaccurate, as described above. You can also be accurate but imprecise. \myskip

For example, if on average, your measurements for a given substance are close to the known value, but the measurements are far from each other, then you have accuracy without precision. \myskip

A good analogy for understanding accuracy and precision is to imagine a basketball player shooting baskets. If the player shoots with accuracy, his aim will always take the ball close to or into the basket. If the player shoots with precision, his aim will always take the ball to the same location which may or may not be close to the basket. A good player will be both accurate and precise by shooting the ball the same way each time and each time making it in the basket.
\section{Numerical Estimates of Uncertainties}

For this laboratory, we will estimate uncertainties with three approximation techniques, which we describe below. You should note which technique you are using in a particular experiment.

\subsection{Upper Bound}

Most of our measuring devices in this lab have scales that are coarser than the ability of our eyes to measure.

\begin{figure}[h]
    \begin{center}
        \includegraphics[width=0.5\textwidth]{./pic/image1.png}
    \end{center}
    \caption{Measuring Length}
    \label{fig:measure}
\end{figure}

For example in the figure above, where we are measuring the length of an object against a meter stick marked in cm, we can definitely say that our result is somewhere between $46.4\,\mathrm{cm}$ and $46.6\,\mathrm{cm}$. We assume as an \emph{upper} bound of our uncertainty, an amount equal to \emph{half} this width (in this case $0.1\,\mathrm{cm}$). The final result can be written as:
\begin{equation}
    \ell = (46.5\pm 0.1)\,\mathrm{cm}
\end{equation}

There will be many circumstances when the error is more complicated than simply the coarseness of the measuring tool. For example, if you find yourself measuring something that is very long or hard to line up properly with a meter stick. In this case, you may need to use some judgement of the best possible measurement to make and the uncertainty will be greater than the millimeter precision of your meter stick. \textbf{It is always best to slightly overestimate error and allow yourself some wiggle room if you feel that better represents your measurement!}

\subsection{Estimation from the Spread (2/3 method)} \label{ssec:twothirds}

For data in which there is random uncertainty, we usually observe individual measurements to cluster around the mean and drop in frequency as the values get further from the mean (in both directions).\footnote{There is a precise mathematical procedure to obtain uncertainties (standard deviations) from a number of measured values. Here we will apply a simple ``rule of thumb'' that avoids the more complicated mathematics of that technique. The uncertainty using the standard deviation for the group of values in our example below is 0.2.}  Find the interval around the mean that contains about 2/3 of the measured points: \emph{half} the size of this interval is a good estimate of the uncertainty in each measurement. \myskip

The reasons for choosing a range that includes 2/3 of the values come from the underlying statistics of the normal (or Gaussian) distribution (see figure \ref{fig:bellcurve}). This choice allows us to accurately add and multiply values with errors and has the advantage that the range is not affected much by outliers and occasional mistakes. A range that always includes all of the values is generally less meaningful. \myskip

\emph{Example}: You measure the following values of a specific quantity:
\begin{equation*}
    9.7,\:9.8,\:10,\:10.1,\:10.1,\:10.3
\end{equation*}
The mean of these six values is 10.0. The interval from 9.8 to 10.1 includes 4 of the 6 values; we therefore estimate the uncertainty to be 0.15. The result is that the best estimate of the quantity is 10.0 and the uncertainty of a single measurement is 0.2.\footnote{Note that about 5\% of the measured values will lie \emph{outside} $\pm$ twice the uncertainty}\footnote{While the above method for calculating uncertainty is good enough for our purposes, it oversimplifies a bit the task of calculating the uncertainty of the \emph{mean} of a quantity.  For those who are interested, please see the appendix for elaboration and clarification. }

\subsection{Square-Root Estimation in Counting}

For inherently random phenomena that involve counting individual events or occurrences, we measure only a single number $N$. This kind of measurement is relevant to counting the number of radioactive decays in a specific time interval from a sample of material, for example. It is also relevant to counting the number of left-handed people in a random sample of the population. The (absolute) uncertainty of such a single measurement, $N$, is estimated as the square root of $N$ (a counting measurement is expressed as $N \pm \sqrt{N}$). As an example, if we measure 50 radioactive decays in 1 second we should present the result as $50\pm 7$ decays per second. (The quoted uncertainty indicates that a subsequent measurement performed identically could easily result in numbers differing by 7 from 50.)

\section{Relative and Absolute Uncertainty}

There are two ways to record uncertainties: the absolute value of the uncertainty or the uncertainty relative to the mean value. So in the example above, you can write $c = (5.1 \pm 0.3)\,\mathrm{cm}$ or equally well $c = 5.1\,\mathrm{cm}\; (1.00 \pm 0.06)$. You can see that if you multiply out the second form you will obtain the first, since $5.1 \times 0.06 = 0.3$. The second form may look a bit odd, but it tells you immediately that the uncertainty is 6\% of the measured value. The number $0.3\,\mathrm{cm}$ is the absolute uncertainty and has the same units as the mean value (cm). The 0.06 (or 6\%) is the relative uncertainty and has no units since it is the ratio of two lengths. It's important to use proper notation when describing uncertainty to remove any unwanted ambiguity, so make sure it's clear when you are using relative or absolute errors.

\section{Propagation of Uncertainties}

Often, we are not directly interested in a measured value, but we want to use it in a formula to calculate another quantity. In many cases, we measure many of the quantities in the formula and each has an associated uncertainty. We deal here with how to propagate uncertainties to obtain a well-defined uncertainty on a computed quantity.

\subsection{Adding/Subtracting Quantities}

When we \textbf{add or subtract} quantities, the combined uncertainty is the \textbf{sum of the absolute uncertainties} of the constituent parts\footnote{The propagation of random uncertainties is actually slightly more complicated, but the procedure outlined here usually represents a good approximation, and it never underestimates the uncertainty. See the appendix for more information.}.

Take as an example measuring the length of a dog. We measure the distance between the left wall and the tail of the dog and subtract the distance from the wall to the dog's nose.
\begin{figure}[h]
    \begin{center}
        \includegraphics[width=0.5\textwidth]{./pic/image2.png}
    \end{center}
    \caption{Measuring a Dog}
    \label{fig:dog}
\end{figure}
So the total length of the dog is:
\begin{equation}
    \begin{split}
        \text{Length} &= (1.53\pm 0.05)\,\mathrm{m} - (0.76 \pm 0.02)\,\mathrm{m} \\
        &= \left( 1.53 - 0.76 \right)\pm\left( 0.05 + 0.02 \right)\,\mathrm{m} \\
        &= \left( 0.77 \pm 0.07 \right)\,\mathrm{m}
    \end{split}
\end{equation}

\subsection{Multiplying/Dividing Quantities}

When we \textbf{multiply or divide} quantities, the combined \textbf{relative} uncertainty is the \textbf{sum of the relative uncertainties} of the constituent parts.\footnote{Our calculation of the uncertainty actually overestimates it. The correct method does not add the absolute/relative uncertainty, but rather involves evaluating the square root of the sum of the squares. For more information please refer to the appendix of this lab manual.}

Take as an example the area of a rectangle, whose individual sides are measured to be:
\begin{align}
    a = 25.0\pm 0.5\,\mathrm{cm} = 25.0\,\mathrm{cm}\;(1.00\pm 0.02) \nonumber \\
    b = 10.0\pm 0.3\,\mathrm{cm} = 10.0\,\mathrm{cm}\;(1.00\pm 0.03)
\end{align}

The area is obtained as follows:
\begin{equation}
    \begin{split}
        \text{Area} &= \left( 25.0\pm 0.5\,\mathrm{cm} \right)\cdot\left( 10.0\pm 0.3\,\mathrm{cm} \right) \\
        &= 25.0\,\mathrm{cm}\;\left( 1.00\pm 0.02 \right)\cdot 10.0\,\mathrm{cm}\;\left( 1.00\pm 0.03 \right) \\
        &= \left( 25.0\,\mathrm{cm}\cdot 10.0\,\mathrm{cm} \right)\left( 1.00\pm \left( 0.02 + 0.03 \right) \right) \\
        &= 250.0\,\mathrm{cm}^2\;(1.00 \pm 0.05) \\
        &= 250.0\pm 12.5\,\mathrm{cm}^2 \\
        &= 250 \pm 10\,\mathrm{cm}^2
    \end{split}
\end{equation}

Note that the final step has rounded both the result and the uncertainty to an appropriate number of significant digits, given the uncertainty on the lengths of the sides. \myskip

\underline{Remarks:} Note that uncertainties on quantities used in a mathematical relationship always increase the uncertainty on the result. The quantity with the biggest uncertainty usually dominates the final result. Often one quantity will have a much bigger uncertainty than all the others. In such cases, we can simply use this main contribution.

\subsection{Multiplication by a Constant}

Multiplying a value by a constant leaves the relative error unchanged. This is equivalent to multiplying the absolute error by the same constant. For example, suppose we are trying to find the circumference of a circle knowing it's radius as $r=1.0 \pm 0.1 \hspace{1mm} \text{cm}$ with error; we would calculate the circumference with error as follows.

\begin{gather}
C = 2\pi r \nonumber \\
 C= 2\pi (1.0 \pm 0.1) \\
 C=6.3 \pm 0.6 \hspace{1mm}  \text{cm} \nonumber
\end{gather}

\subsection{Powers and Roots}

When raising a value to a certain power, its \textbf{relative uncertainty is multiplied by the exponent}. This applies to roots as well, since taking the root of a number is equivalent to raising that number to a fractional power.\myskip

Squaring a quantity involves multiplying its relative uncertainty by 2, while cubing a quantity causes its relative uncertainty to be multiplied by 3.\myskip

Taking the square root of a quantity (which is equivalent to raising the quantity to the 1/2 power) causes its relative uncertainty to be multiplied by 1/2. For example, if you know the area of a square to be:
\begin{equation}
    \text{Area} = 100\pm 8\,\mathrm{m^2} = 100\,\mathrm{m}^2\;(1.00\pm 0.08)
\end{equation}
then it follows that the side of the square is:
\begin{equation}
    \text{Side} = 10\,\mathrm{m}\;\left( 1.00\pm 0.04 \right) = 10.0\pm 0.4\,\mathrm{m}
\end{equation}
The most general rule for finding the error in powers and roots is mathematically represented as follows.
\begin{gather}
f(x) = x^n \\
\frac{\sigma_{f(x)}}{f(x)} = |n| \frac{\sigma_x}{x}
\end{gather}
Where $\sigma$ is the {\it{absolute}} uncertainty and $f(x)$ is some power or root of $x$.

\subsection{Other Functions}

If you need to calculate the error of a calculation that does not fit into one of these rules (such as trigonometric functions or logarithmic ones), here is a manual method that you can use.\myskip

Based upon the error of the quantity that you determined, you can find the maximum and minimum values of the quantity that you are calculating. The value that you found should be roughly midway between these two quantities. Then if you split the difference between the maximum and minimum you should obtain a reasonable estimate of the error. Mathematically, you would do so as follows.
\begin{gather}
\sigma_{f(x)} = \frac{f(x + \sigma_x) - f(x - \sigma_x)}{2}
\end{gather}

Here is an example: Suppose you measure an angle to be $(47.3 \pm 0.5)^\circ$ and you want to determine the error of $\sin(47.3 \pm 0.5)^\circ$. You find that $\sin(47.3) = 0.735$. Based upon your reported uncertainty, you know that your angle could be as large as $47.8^\circ$ and as small as $46.8^\circ$, and therefore you should calculate $\sin(47.8) = 0.741$ and $\sin(46.8) = 0.729$. So your calculated value is 0.735 but it can be as low as 0.729 and as high as 0.741 and therefore, if you halve the difference between 0.729 and 0.741 you get a reasonable error estimate of 0.006. So you should report your value as $0.735 \pm 0.006$.

\section{Best-Fit Line}

In most research laboratories, plotting measurements is found to be the preferred method of reviewing the data and quantitatively measuring the relationship between the experimental variables. This is effective because we often have some idea of the expected relationship between the variables {\it{a priori}}. In these labs, this expected relationship is almost always arranged to be a straight line. But even if we know that the ideal points fit on a precise straight line, experimentally measured data points will not always lie on a single line -- because the measurements always have intrinsic uncertainty. Therefore when the points are plotted, we should include error bars on both axes to indicate the uncertainties in the data. Because real measurements do not all lie on a single straight line, there are a variety of possible lines you might choose to fit the data. \myskip

\begin{figure}[h]
    \begin{center}
        \includegraphics[width=0.9\textwidth, height=0.5\textwidth]{./pic/image13.jpg}
    \end{center}
    \caption{Left: an example best-fit line. Right: the maximum and minimum possible slope from our data used to calculate uncertainty in the best-fit line. Notice how we have drawn the lines on the outer bounds of the error bars to achieve the maximum and minimum possible slope within the error bars.}
    \label{fig:bestfit}
\end{figure}

How do we know which line represents the best fit? There is an exact mathematical procedure to obtain the best-fit line, but this is usually a very tedious calculation which is outside the scope of this lab. For experiments in this course, you will be using Excel's built-in fitting function for data. The process for doing which will be explained in your next lab. However, if you are interested in learning how to approximate the technique without a computer, please see the appendix.\myskip

\underline{Remark}: Often in our experiments the data points will not look as nice as in the above examples. One or several points may not be close to any best-fit line you try. Such anomalous points may occur, for example, because of a mistake in measuring. In such cases, it is acceptable to ignore these anomalies when estimating the best-fit line (and of course you must note this fact down in your lab report).  Dropping anomalous points must be done with extreme care and only rarely (if you know the point is not physically meaningful).\footnote{More than once, data points that did not behave as theory predicted turned out to be new effects and led to Nobel prizes!}  It is better to choose a line with as many points above the line as below. If you are not sure of your measurements, it is better to re-measure or to take more data points. \myskip

\section{Numerical Statistics}
The previous discussions of uncertainty and error tell us how we can quantitatively describe our inability to make perfect single measurements. However, in real physics experiments, very rarely do we draw conclusions from a single data points. As such, it is essential that we know how to quantify error in sets of data. The 2/3 methods as discussed in Section \ref{ssec:twothirds} provides a good estimation of data statistics, but we can more rigorously calculate data set statistics. In statistics, a data set can be well described by the following four fundamental quantities: mean, median, mode, and standard deviation. The mean of a data set is the sum of all numbers in the data set divided by the number of points in the set. It is defined in the following manner.
\begin{gather}
 \text{Average} \equiv \bar x= \sum_{i} \frac{x_i}{N}
\end{gather}
The median of a data set is the middle value in a set of numbers listed in increasing order. The mode is the number that occurs the most number of times in the data set. The standard deviation describes  how the numbers in the data set are distributed around the mean. It is defined as follows.
\begin{gather}
\text{standard deviation} \equiv \sigma = \sqrt{\sum_i \frac{(x_i - \bar x)^2}{N}}
\end{gather}

These four statistical quantities give us enough information to characterize the distribution of our data set. For example, let's consider the two following Data Sets.
\myskip
\begin{center}
\begin{tabular}{c | c | c | c | c | c | c | c | c | c | c | c | c | c  }
&&&&&&&&&&Mean&Median&Mode& $\sigma$ \\
Set 1 & 9&8&11&13&10&10&12&6&9 &9.8&10&10&2.2\\
Set 2 & 11&0&10&40&2&3&10&10&4 &9.8&10&10&11.4
\end{tabular}
\end{center}
\myskip
Notice how both Data Sets have the same mean, median, and mode, which tells us the data points in each set are centered on the mean value of $9.8$. However, the standard deviations are quite different. The large standard deviation in Data Set 2 tells us there must be outliers in the data set which increase the distribution. Whereas, the relatively small standard deviation in the first data set tells us the numbers in set 1 are clustered closely together. Standard deviation is especially important because  it tells us exactly how distributed the number are around the mean value, which gives an indication of how error affects the spread of data points (for example, see figure \ref{fig:bellcurve} for the canonical "bell curve" distribution, also known as the gaussian distribution). The 2/3 method as discussed in Section \ref{ssec:twothirds} is an approximation to the standard deviation since $2/3 \sim 66\%$, which roughly corresponds to the first standard deviation (see figure \ref{fig:bellcurve}).

\begin{figure}[h]
    \begin{center}
        \includegraphics[width=0.7\textwidth]{./pic/image3.png}
    \end{center}
    \caption{An example of a Gaussian distribution, also known as a bell curve. $\sim 68\%$ of the data points are within 1 standard deviation, $\sim 95\%$ of that data points are contained within 2 standard deviations, $\sim 99.5 \%$ of the data points are contained within 3 standard deviations, etc...}
    \label{fig:bellcurve}
\end{figure}

\section{Number of Significant Digits}

The number of significant digits in a result refers to the number of digits that are relevant. The digits may occur after a string of zeroes. For example, the measurement of $2.3\,\mathrm{mm}$ has two significant digits. This does not change if you express the result in meters as $0.0023\,\mathrm{m}$. The number 100.10, by contrast, has 5 significant digits\footnote{Another way to find the number of significant digits is to convert to scientific notation, and count the number of digits in the mantissa (also significand or coefficient). For example: for $1.2\times 10^{2}$, there are two significant digits in $1.2$. }.

When you record a result, you should use the calculated error to determine how many significant digits to keep. Let's illustrate the procedure with the following example. Assume you measure the diameter of a circle to be $d = 1.6232\,\mathrm{cm}$, with an uncertainty of $0.102\,\mathrm{cm}$. You now round your uncertainty to one or two significant digits (up to you). So (using one significant digit) we initially quote $d = (1.6232 \pm 0.1)\,\mathrm{cm}$. Now we compare the mean value with the uncertainty, and keep only those digits that the uncertainty indicates are relevant. Finally, we quote the result as $d = (1.6 \pm 0.1)\,\mathrm{cm}$ for our measurement.

Suppose further that we wish to use this measurement to calculate the circumference $c$ of the circle with the relation $c = \pi\cdot d$. If we use a standard calculator, we might get a 10 digit display indicating:
\begin{equation}
    c = 5.099433195\pm 0.3204424507\,\mathrm{cm}
\end{equation}
This is not a reasonable way to write the result!  The uncertainty in the diameter had only one significant digit, so the uncertainty of the circumference calculated from the diameter cannot be substantially better. Therefore we should record the final result as:
\begin{equation}
    c = 5.1\pm 0.3\,\mathrm{cm}
\end{equation}
(If you do intermediate calculations, it is a good idea to keep as many figures as your calculator can store. The above argument applies when you \underline{record} your results!)


\chapter{Error Analysis With Excel}

\section{Plotting with Excel}
An important set of data analysis tools in Excel are plotting and linear fit functions. You will need to plot and fit data many times throughout this lab course, so make sure you are familiar with this section. Below is a walkthrough of plotting and fitting a set of data with error in excel.
\begin{enumerate}
\item Before plotting, you need to have 4 columns with data: x data, y data, x error data, and y error data. Make sure you have entered the information into excel.
\item First select your x data and y data (you can select multiple boxes in excel by holding down the ctrl button while selecting). Make sure to select your x data first or your x and y axes will be switched.
\item Choose the subheading ``insert", then ``Scatter", then ``Scatter with straight lines and markers". Now your x and y data should be plotted without error bars (see figure \ref{fig:excel1}).

\begin{figure}[h!]
\centering
\includegraphics[height=0.4\textheight, width=0.7\textwidth]{./pic/image4.png}
\caption{Selecting x and y data and creating a lined scatter plot in excel.}
\label{fig:excel1}
\end{figure}

\item To include error bars select your chart, then click the ``plus" marker on the top right of the chart. Check the box titled ``error bars". Now some basic error bars should appear on the plot. These are not based on the error bar data in your excel document, they are standard error bars.
\item To change them so they match your error bar data, select the x axis error bars on your chart, and format the error bars by clicking the ``Custom" selection, then ``specify value".
\item It should now prompt you for positive error values and negative error values. Delete ``$\{1\}$" from the two boxes, and select your error bars using the cursor. Your chart will now have the correct error bars (see figure \ref{fig:excel2}).

\item Repeat steps 5-6 for your y data.
\item To linear fit your data, right click on your data in the plot and select ``Add Trendline". Check the ``Linear", ``Display Equation on chart", and ``Display R-square value on chart" boxes.
\item Now the slope, intercept, and R squared values will be displayed on your chart. R squared is a measure of how well the line fits your data. It should be close to $1$ and at the very least greater than $0.9$.
\end{enumerate}

\begin{figure}[h!]
\centering
\includegraphics[height=0.4\textheight, width=0.7\textwidth]{./pic/image5.png}
\caption{Using your own data set to create x and y error bars in excel.}
\label{fig:excel2}
\end{figure}

\begin{figure}[h!]
\centering
\includegraphics[height=0.4\textheight, width = 0.7\textwidth]{./pic/image6.png}
\caption{Using excel to perform a linear fit and return the intercept and slope.}
\label{fig:excel3}
\end{figure}

\section{Finding Error in Slope}
\label{sec:linest}

The previous steps will help plot your data, but in order to draw conclusions, you must add uncertainty. Excel has a built in function called LINEST which finds the standard error for a linear fit. Using the same columns of data from before, the walkthrough below will help you calculate the error so you can propagate it further in the experiment.

\begin{enumerate}
\item The LINEST function is an array-type function, meaning it will output more than one number. Start, by highlighting a 2-by-3 section of empty cells (two columns, three rows).
\item With these six cells highlighted, in the input box at the top of the screen type ``= LINEST(" and add the proper arguments. The arguments should be the list of y-values, the list of x-values, TRUE, and TRUE.
For example:
$=\texttt{LINEST}(\texttt{C2:C11, A2:A11, TRUE, TRUE})$.

\item Press CONTROL+SHIFT+ENTER. \it{Note that on a Mac, this is CMD+SHIFT+ENTER.}
\end{enumerate}

Your results should have filled in that 2-by-3 section in the following way:
\begin {table}[H]
\begin{center}
\begin{tabular}{ |c | c | c |}
\hline
Slope &X Intercept \\ \hline
Error of Slope &Error of Intercept \\ \hline
$R^2$ &Error in Y \\ \hline
\end{tabular}
\caption{LINEST function output in Excel.}
\end{center}
\end{table}

\section{Helpful Commands}

Your TA will guide you through the relevant excel commands necessary for data analysis, however a list of some relevant excel commands are listed below. A list of all excel commands can be found on the \href{https://support.office.com/en-us/article/Excel-functions-alphabetical-b3944572-255d-4efb-bb96-c6d90033e188}{Microsoft Office website}\footnote{https://support.office.com/en-us/article/Excel-functions-alphabetical-b3944572-255d-4efb-bb96-c6d90033e188}

\begin{center}
\begin{tabular}{c | c}
\texttt{ABS} & Returns the absolute value of a number \\
\texttt{AVERAGE} & Computes the average of the selected data set \\
\texttt{COS} & Calculates cosine of a number\\
\texttt{DEGREES} & Converts radians to degrees \\
\texttt{EXP} & Returns e raised to the power of a given number \\
\texttt{LN} & Returns the natural logarithm of a number \\
\texttt{MEDIAN} & Finds the median of a data set \\
\texttt{MODE.SNGL} & Finds the most commonly occurring number in a data set\\
\texttt{PI} & Returns the value of pi\\
\texttt{POWER} & Returns the result of a number raised to a power\\
\texttt{SIN} & Calculates the sine of a number\\
\texttt{SQRT} & Calculates the square root of a number\\
\texttt{STDEV.P} & Calculates the standard deviation based on the entire population \\
\texttt{STDEV.S} & Estimates the standard deviation based on a sample \\
\texttt{SUM} & Calculates the sum of a data set\\
\texttt{TAN} & Calculates the tangent of a number
\end{tabular}
\end{center}


\chapter{Advanced Error Analysis}
\section{Clarification of 2/3 Rule}

To find the true uncertainty, we are really interested in the \emph{standard error of the mean}, i.e., how likely it would be for a newly measured average value to be close to our original value were we to perform the experiment again.  The proper way to figure this out would be to get say a thousand friends to perform this experiment in the same way, each using the same number of data points, and then compare the results of everyone.  Each student would calculate his or her own mean, and they would likely all be clustered around some central average.  We could then examine the spread of this cluster of means using the 2/3 rule, and we'd have a quantitative measure of the uncertainty surrounding any single student's measurement.\myskip

While it's usually impractical to get 1000 friends together to repeat an experiment a thousand times, it turns out that the uncertainty (or ``standard error'') of the mean can be estimated with the following formula:
\begin{equation}
    \text{Standard Error Of The Mean} = \frac{\text{Uncertainty Of Single Measurement}}{\sqrt{N}}
\end{equation}
where ``$N$'' is the number of data points in your sample, and ``Uncertainty Of Single Measurement'' is the uncertainty calculated via the 2/3 method.  The ``$\sqrt{N}$'' term should make sense qualitatively -- as we take more and more data points, our measured average becomes less and less uncertain as we approach what should be the ``global'' mean.

\section{The ``Correct'' Way to Add Uncertainties}

The rules we've given for propagating uncertainties through a calculation are essentially correct, and intuitively make sense.  When adding two quantities together, if one has an uncertainty of $\Delta$x and another has an uncertainty of $\Delta y$, the sum could indeed range from $(x+y) - (\Delta x +  \Delta y)$ to $(x+y) + (\Delta x + \Delta y)$.  This implies that the proper way to find the uncertainty of $(x+y)$ is to add their respective absolute uncertainties. \myskip

There is, however, a small problem -- this overestimates the uncertainty!  Since $x$ and $y$ are equally likely to be wrong by either a \emph{positive} amount or a \emph{negative} amount, there's a good chance that the respective errors of each variable will partly cancel one another out.  To account for this, a more accurate way to estimate uncertainty turns out to be to add uncertainties \emph{in quadrature}.  This means:\myskip

\textbf{Adding/Subtracting Quantities}

\begin{equation}
    (A\pm\Delta A) + (B\pm\Delta B) = (A+B)\pm\sqrt{(\Delta A)^2+(\Delta B)^2}
\end{equation}

\myskip\textbf{Multiplying/Dividing Quantities}

\begin{equation}
    A\left(1\pm\frac{\Delta A}{A}\right) \times B\left(1\pm\frac{\Delta B}{B}\right) = \left(A \times B\right) \left ( 1 \pm\sqrt{\left(\frac{\Delta A}{A}\right)^2+\left(\frac{\Delta B}{B}\right)^2} \right )
\end{equation}

While this method gives a closer approximation to what the true propagated uncertainty should be, it is clearly a more complex calculation.  In the limited time available to complete your experiment and lab report, you may use the simpler, earlier uncertainty calculation method provided, and avoid this complicated calculation.  But do remember that the simpler method \emph{overestimates} the total uncertainty.

\section{Max-Min Method for Best-fit Line}

This alternate technique will show you how to draw an approximate best-fit line for a  set of data without a computer and it is sufficiently precise for most purposes.

First, try to draw a line with as many points (with uncertainties included) lying above the line as below it. The gauge of how close the line is to a point is given by the uncertainty associated with that measured point. However, all the points at the left end should not lie on one side of the line with all the points at the right end lying on the other side. As a rule of thumb, roughly 2/3 of the points should have the line passing through the uncertainties (just as with the 2/3 rule).\myskip
%The uncertainty for the best-fit line is obtained by estimating how much one could increase and decrease the slope of the line before the fit is deemed very bad. \myskip

Clearly, this ``eyeball'' method has inherent uncertainty, so how do we estimate the uncertainty on the slope of the best-fit line? To do this we should estimate the spread of the slope, or maximum and minimum possible slopes that one can conceivably interpret from the graph. Half the difference between the minimum and maximum slopes is a good estimate of the slope uncertainty ($\sigma=\frac{m_{max}-m_{min}}{2}$).
\end{document}
